\documentclass{article}

\usepackage[margin=1in]{geometry}
\usepackage{tikz-cd}
\usepackage{amssymb}
\usepackage{amsmath}

\begin{document}

\newtheorem{theorem}{Theorem}
\newtheorem{definition}{Definition}
\newcommand{\pto}{\nrightarrow}
\newcommand{\pfrom}{\nleftarrow}
\newcommand{\vcat}{\mathcal}
\newcommand{\cat}{\mathbb}

\title{Universal Properties in Generalized Multicategories}
\author{Max S. New}

\maketitle

In this paper, we seek to develop some theory of right and left
universal properties at the abstract level of generalized
multicategories.
%
Our goal is to ease the development of new type theories with sound
and complete categorical semantics based on Martin-Lof's notion of
judgmental structure.
%
In effect we are abstracting the Curry-Howard-Lambek correspondence as
follows:

\begin{tabular}{|c|c|}
  \hline
  Syntax/Type Theory & Semantics/Category Theory\\ \hline \hline
  Judgmental Structure & Generalized Multicategory\\ \hline
  Negative Type & Representable Generic Right Universal Property\\ \hline
  Positive Type & Representable Generic Left Universal Property\\ \hline  
\end{tabular}

In the process we will provide definitions and theorems that should
free the working semanticist from proving some tedious theorems and
provide guidance for the development of new type theories.
%
In particular, our notions of generic right and left-universal
property lead to a transparent definition of the category of models in
which the type theory presents the initial model.
% TODO: actually see if we *can* prove the following LOL
Further, we provide a quite abstract ``presentation parsimony''
theorem that shows that type theories presenting functorial
constructions can \emph{derive} the functoriality from the
introduction and elimination rules.

\section{Horizontal Profunctors and Representability}

In ordinary category theory there is no need for a separate notion of
right- and left-universal property since categories satisfy perfect
duality: the opposite of a category is a category and so a
left-universal property is defined as a right-universal property in
the opposite category.
%
However multicategories are assymetric by default, the source is given
by a context and the target by a single object, so the opposite of a
multicategory is not a multicategory but a poycategory.
%
It is conceivable that the more general framework of polycategories
will restore this duality, but for now we will work with the
asymmetric notion, which will help to develop intuition in the worst
case.

Despite the lack of perfect duality, we can abstract over some of the
commonalities between right and left universal properties.
%
Right universal properties are right modules and left universal
properties are left modules, but rather than being modules between
categories or between multicategories, the modules will have one side
be a category and the other a multicategory.
%
This structure is generally modelled by a horizontal profunctor
between the virtual equipments of categories and multicategories.

\begin{definition}[Horizontal Profunctor]
  Given virtual double categories $\vcat{V},\vcat{U}$, a
  \emph{horizontal profunctor} $H : \vcat V \pfrom \vcat U$ is for
  each $v \in \vcat V, u \in \vcat U$ a set $H(v,u)$ of ``horizontal
  heteromorphisms'' from $u$ to $v$, so written $v \pfrom u$.
  Additionally a notion of ``hetero-2-cell'' with shapes:

  \begin{tikzcd}
v_n \arrow[dd, "f"] & \cdots \arrow[l, "R_n"] & v_0 \arrow[l, "R_1"] & u_0 \arrow[l, "h"] & \cdots \arrow[l, "Q_1"] & u_m \arrow[l, "Q_m"] \arrow[dd, "g"] \\
 &  &  &  &  &  \\
v' &  &  &  &  & u' \arrow[lllll, "h'"]
  \end{tikzcd}

  that can be composed with other hetero-2-cells and 2-cells in $\vcat
  V, \vcat U$ in the evident way, satisfying the evident unitality and
  associativity laws.
\end{definition}

We draw diagrams for horizontal profunctors the same way we draw them
for virtual double categories, this is justified by the following
definition, in which they are just diagrams in a virtual double category:

\begin{definition}[Cograph of a Horizontal Profunctor]
  For any horizontal profunctor $H : \vcat V \pfrom \vcat U$ we
  construct the cograph $V +_{H} U$ to have as objects $V_0 + U_0$,
  vertical arrows only within $V,U$ and horizontal morphisms defined
  as:
  \begin{align*}
    (V +_{H} U)(v,v') &= V(v,v')\\
    (V +_{H} U)(u,u') &= U(u,u')\\
    (V +_{H} U)(v,u) &= H(v,u)\\
    (V +_{H} U)(u,v) &= \emptyset
  \end{align*}
  With 2-cells inherited from $V,U,H$.
\end{definition}

I'm not sure if it's possible to define a virtual equipment of virtual
double categories, double functors and \emph{horizontal} profunctors,
but we will need at least the following very simple kind of 2-cell in
this imagined equipment of the following shape:

\begin{tikzcd}
 & W \arrow[ldd, "F_V"] \arrow[rdd, "F_U"] &  \\
 & F &  \\
V &  & U \arrow[ll, "H"]
\end{tikzcd}

\begin{definition}[Parameterized Element of a Profunctor]
  Let $\vcat {V,U}$ be virtual double categories, $\cat W$ be a
  category and let $H \vcat V \pfrom \vcat U$ be a profunctor.

  Then a $\cat W$-parameterized element of $H$ consists of
  \begin{enumerate}
  \item Functors $F_U : \cat W \to (\vcat U)_v$ and $F_{\vcat V} :
    \cat W \to (\vcat V)_v$.
  \item For each $w \in W$, a heteromorphism $F(w) : F_{\vcat V} w
    \pfrom F_{\vcat W} w$
  \item For each $f : w \to w'$, a 2-cell
    \begin{tikzcd}
F_{V} w \arrow[dd, "F_{V} f"] &  & F_{U} w \arrow[ll, "Fw"] \arrow[dd, "F_{U}f"] \\
 & Ff &  \\
F_{V} w' &  & F_{U} w' \arrow[ll, "Fw'"]
    \end{tikzcd}
    such that the assignment is functorial (preserve identity,
    composition).
  \end{enumerate}
\end{definition}

Now we can define a general notion of representability in a
profunctor.
%
Representability is a formulation of a (parameterized) type having a
universal property: the horizontal morphism is the universal property
and the 2-cell means ``maps into/out of this object look like the
universal property''.
%
In a virtual double category representability is a relationship
between horizontal and vertical morphisms with the same boundary, but
our horizontal profunctor doesn't give us a notion of vertical
morphism.
%
It might be possible to develop a more general definition of
profunctor, but instead for now we will limit to the following case:

\begin{definition}{Representable Heteromorphism}
  Let $H : \vcat V \pfrom \vcat U$ be a horizontal profunctor and
  $\bar H : \vcat U \to \vcat V$ be a functor with a $(\vcat U)_v$
  parameterized element of $H$:
  \[
  \begin{tikzcd}
 & U_v \arrow[ldd, "\bar H"'] \arrow[rdd, "id"] &  \\
 & \bar H &  \\
V &  & U \arrow[ll, "H"]
\end{tikzcd}
  \]

  Then a representation of a heteromorphism $h : u \pfrom v$ consists
  of a vertical arrow $\bar h : v \to \bar H u$ and a 2-cell
  \[
  \begin{tikzcd}
u \arrow[dd, "id"] &  & v \arrow[dd, "\bar h"] \arrow[ll, "h"] \\
 & \bar h &  \\
u &  & \bar H u \arrow[ll, "\bar H u"]
  \end{tikzcd}
  \]
  that is cartesian in $\vcat V +_{H} \vcat U$.
\end{definition}

Below, we think of all heteromorphisms as expressing
\emph{specifications} for vertical morphisms in this way, which is why
it is thought of as a ``property'' that an object may or may not have.

\section{Concrete and Generic Right Universal Properties}



\section{Left Universal Properties}


\end{document}

%% Local Variables:
%% compile-command: "pdflatex upgm.tex"
%% End:

\subsection{Generalized Algebraic Theory of VE Hyperdoctrines}

To simplify our initiality proof, we use a \emph{generalized
algebraic} presentation of our notion of
model\cite{cartmell,sterling-eat}. To describe this, we give a
``combinator'' syntax for VE hyperdoctrines, where we have explicit
sorts for contexts $\Gamma$ and $\Phi$ and explicit judgments for
substitutions $\gamma : \Delta \to \Gamma$ and $\phi : \Psi
\vdash_{f,g} \Phi$.
%
While the explicit description of the theory is quite long, we can
describe it in simple modular pieces.
%
First, we use the standard theory of a category with families, which
describes the algebraic structure of contexts $\Gamma$, types $\Gamma
\vdash A$ and terms $\Gamma \vdash M : A$.
%
Second, we add to that a theory for a slight generalization of virtual
double categories we call \emph{double categories with display
pro-arrows}.
%
The difference is that in a double category with display pro-arrows,
we have a true double category with strict vertical and horizontal
composition, among whom are the ``display'' pro-arrows.
%
The display pro-arrows model the profunctors of the syntax, which
don't have a strict composition, whereas the arbitrary pro-arrows
model the contexts $\Phi$, which do have a strict composition, given
by context concatenation $\Phi,\Psi$.
%
The difference is that in a virtual double category the contexts are
assumed to be given inductively by formal compositions of the display
pro-arrows, whereas in models of this algebraic theory, the contexts
might include other exotic objects.
%
This difference is analogous to the difference between a category with
families and a contextual category in models of dependent type theory,
and we fully expect that there is a similar relationship between the
models: virtual double categories should form a co-reflective
subcategory characterized as those double categories with display
pro-arrows where all pro-arrows are equal to some concatenation of
displays.
%
Finally, we parameterize all of the constructions of the theory of
double categories with pro-arrows by contexts $\Gamma$ and add in
equations that substitution commutes with all double category
operations.
%
This means the theory presents a category with families with an double
category with display pro-arrows internal to the category of
presheaves on the category of contexts, i.e., a hyperdoctrine of
double categories with display pro-arrows.

The main advantage of this algebraic approach is that the initiality
theorem for the combinatory syntax is trivial. Instead, the work is to
show that our type theoretic syntax, which restricts combinatory rules
to make substitution an admissible rather than primitive operation, is
equivalent to the combinatory syntax.

\begin{figure}
  \begin{mathpar}
    \inferrule
    {}
    {\Gamma : \algCtx}

    \inferrule
    {\Delta : \algCtx \and \Gamma : \algCtx}
    {\gamma : \algSubst(\Delta,\Gamma)}

    \inferrule
    {\Gamma : \algCtx}
    {C : \algCat(\Gamma)}

    \inferrule
    {\Gamma : \algCtx \and C : \algCat(\Gamma) \and D : \algCat(\Gamma) \and
    }
    {F : \algVarr(\Gamma,C,D)}

    \inferrule
    {\Gamma : \algCtx \and C : \algCat(\Gamma) \and D : \algCat(\Gamma) \and
    }
    {P : \textrm{Profunctor}(\Gamma,C,D)}

    \inferrule
    {\Gamma : \algCtx \and C : \algCat(\Gamma) \and D : \algCat(\Gamma)
    }
    {P : \textrm{ProfCtx}(\Gamma,C,D)}

    \inferrule
    {\Gamma : \textrm{Ctx}\and
      C  : \algCat(\Gamma) \and D  : \algCat(\Gamma) \and
      C' : \algCat(\Gamma) \and D' : \algCat(\Gamma) \and
      f : \algVarr(\Gamma,C,C') \and g : \algCat(\Gamma,D,D') \and
      \Phi : \textrm{ProfCtx}(\Gamma,C,D)\and
      \Psi : \textrm{ProfCtx}(\Gamma,C',D')
    }
    {\phi : \textrm{ProfSubst}(\Gamma,C,D,C',D',f,g,\Phi,\Psi)}
    
    \inferrule
    {\Gamma : \textrm{Ctx}\and
      C  : \algCat(\Gamma) \and D  : \algCat(\Gamma) \and
      C' : \algCat(\Gamma) \and D' : \algCat(\Gamma) \and
      f : \algVarr(\Gamma,C,C') \and g : \algCat(\Gamma,D,D') \and
      \Phi : \textrm{ProfCtx}(\Gamma,C,D)\and
      P : \textrm{ProfCtx}(\Gamma,C',D')
    }
    {t : \textrm{Trans}(\Gamma,C,D,C',D',f,g,\Phi,P)}
  \end{mathpar}
  \caption{Sorts of Algebraic SCT}
\end{figure}



\begin{figure}
  \begin{mathpar}
    \cdots
  \end{mathpar}
  \caption{Algebraic Dependent Type Theory}
\end{figure}

The algebraic theory of a double category with displays indexed by a
category is given in Figure~\ref{fig:alg-dc-disp}.
%
First, we give the categories and functors the theory of a category
indexed by a category.
%
Next, we give categories and \emph{profunctor contexts} the same
theory, where the identity $\id_h$ is the ``empty'' context $\alpha:C$ and the composition $\Phi \circ_h$.
%

TODO: include all of the interactions with substitutions and restrictions.

\begin{figure}
  \begin{mathpar}
    % theory of an indexed category (vertical)
    \inferrule
    {g : \algVarr(D,E) \and \algVarr(C,D)}
    {g \circ_v f : \algVarr(C,E)}

    (g \circ_v f) \circ_{vs} \gamma = (g \circ_{vs} \gamma) \circ_v (f \circ_s \gamma)
    
    (h \circ_v g) \circ_v f = h \circ_v (g \circ_v f)

    \inferrule
    {}
    {\id_v : \algVarr(C,C)}

    \id_v \circ_{vs} \gamma = \id_h

    (f \circ_v \id_v) = f\and \id_v \circ_v f = f

    % theory of an indexed category (horizontal)

    \inferrule
    {\Psi : \algHCtx(D,E) \and \algHCtx(C,D)}
    {\Psi \circ_h \Phi : \algHCtx(C,E)}

    (\Psi \circ_h \Phi) \circ_{hs} \gamma = (\Psi \circ_{hs} \gamma) \circ_h (\Phi \circ_s \gamma)
    
    (\Sigma \circ_h \Psi) \circ_h \Phi = \Sigma \circ_h (\Psi \circ_h \Phi)

    \inferrule
    {}
    {\id_h : \algHCtx(C,C)}

    \id_h \circ_{hs} \gamma = \id_h

    (\Phi \circ_h \id_h) = \Phi\and \id_h \circ_h \Phi = \Phi

    % embedding of displays

    \inferrule{P : \algHarr(C,D)}{i(P) : \algHCtx(C,D)}

    % 2-cells: substitutions
    % vertical category
    % horizontal category
    % interchange law
    \inferrule
    {\psi : \algHSubst(\Phi, f',g',\Psi)\and \phi : \algHSubst(\Sigma,f,g,\Phi)
    }
    {\psi \circ_{2v} \phi : \algHSubst(\Sigma, f' \circ f, g' \circ g, \Psi)}

    \inferrule
    {}
    {\id_{2v} : \algHSubst(\Phi,\id,\id,\Phi)}

    (\phi \circ_{2v} \psi) \circ_{2v} \sigma = \phi \circ_{2v} (\psi \circ_{2v} \sigma)\\
    \phi \circ_{2v} \id_{2v} = \phi \and \phi = \id_{2v} \circ_{2v} \phi\\

    \inferrule
    {\phi_1 : \algHSubst(\Psi_1,f,g,\Phi_1)\and \phi_2 : \algHSubst(\Psi_2,f,g,\Phi_2)
    }
    {\phi_1 \circ_{2h} \phi_2 : \algHSubst(\Psi_1 \circ_h \Psi_2, f, h, \Phi_1 \circ_h \Phi_2)}

    \inferrule
    {}
    {\id_{2h} : \algHSubst(\id_h,f,f,\id_h)}

    (\phi \circ_{2h} \psi) \circ_{2h} \sigma = \phi \circ_{2h} (\psi \circ_{2h} \sigma)\\
    \phi \circ_{2h} \id_{2h} = \phi \and \id_{2h} \circ_{2h} \phi = \phi\\

    (\phi_1 \circ_{2h} \phi_2) \circ_{2v} (\psi_1 \circ_{2h} \psi_2) = (\phi_1 \circ_{2v} \psi_1) \circ_{2h} (\phi_2 \circ_{2v} \psi_2)

  \end{mathpar}
  \caption{Double Category with Display Pro-arrows}
  \label{fig:alg-dc-disp}
\end{figure}

\begin{figure}
  \begin{mathpar}
    % Restrictions
    \inferrule
    {P : \algHarr(\Gamma,C',D')
    \and f : \algVarr(\Gamma,C,C')\and
    \and g : \algVarr(\Gamma,D,D')}
    {P \circ_r (f,g) : \algHarr(\Gamma,C,D)}

    (P \circ_r (f,g) \circ_s \gamma) = (P \circ_s \gamma) \circ_r (f \circ_s \gamma, g \circ_s \gamma)

    P \circ_r (f,g) \circ_r (f',g') = P \circ_r (f \circ_r f', g \circ_r g')

    P \circ_r (\id,\id) = P

    % Prof Multiplicatives
    % LHom
    \inferrule
    {P : \algHarr(C, D) \and Q : \algHarr(C,E)}
    {\algLHom(P,Q) : \algHarr(D,E)}
    
    \algLHom(P,Q) \circ_{hs} \gamma = \algLHom(P \circ_{hs} \gamma, Q \circ_{hs} \gamma)\and
    \algLHom(P,Q) \circ_{r} (f,g) = \algLHom(P \circ_r (\id,f), Q \circ_r(\id,g))

    \inferrule
    {}
    {\algLHomE(P,Q) : \algHSubst(i(P)\circ_h i(\algLHom(P,Q)),\id_v,\id_v,i(Q))}

    \inferrule
    {t : \algHSubst(i(P) \circ_h \Phi, \id,g, i(Q))}
    {\algLHomI(\Phi,g,t,P,Q) : \algHSubst(\Phi,\id,g,i(\algLHom(P,Q)))}

    \algLHomE \circ_{2v} (\id_v(P) \circ_h \algLHomI(t)) = t\and
    \theta = \algLHomI(\algLHomE \circ_{2v} (\id_v(P) \circ_h \theta))\and
        
    % RHom
    \inferrule
    {P : \algHarr(C, E) \and Q : \algHarr(D,E)}
    {\algRHom(P,Q) : \algHarr(C, D)}

    \algRHom(P,Q) \circ_{hs} \gamma = \algRHom(P \circ_{hs} \gamma, Q \circ_{hs} \gamma)\and
    \algRHom(P,Q) \circ_{r} (f,g) = \algRHom(P \circ_r (f, \id), Q \circ_r(g,\id))

    \inferrule
    {}
    {\algRHomE(P,Q) : \algHSubst(i(\algRHom(P,Q)) \id_h i(Q),\id_v,\id_v,i(P))}

    \inferrule
    {t : \algHSubst(\Phi \circ_h i(Q), f, \id, i(P))}
    {\algRHomI(\Phi,f,t,P,Q) : \algHSubst(\Phi,\id,g,i(\algLHom(P,Q)))}

    \algRHomE \circ_{2v} (\algRHomI(t) \circ_h \id(Q)) = t\and
    \theta = \algRHomI(\algRHomE \circ_{2v} (\theta \circ_h \id(Q)))\and

    % Unit
    \inferrule
    {C : \algCat}
    {\algUnit(C) : \algHarr(C,C)}

    \algUnit(C) \circ_{hs} \gamma = \algUnit(C \circ_{cs} \gamma)

    \inferrule
    {}
    {\algUnitI(C) : \algSubst(\id_h(C),\id,\id,i(\algUnit(C)))}

    \inferrule
    {t : \algHSubst(\Phi \id_h \Psi,f,g,i(R))}
    {\algUnitE(t) : \algHSubst((\Phi \circ_h i(\algUnit(C)) \circ_h \Psi), f, g, i(R))}

    \algUnitE(t) \circ_{2v} (\id_{2v}(\Phi) \circ_h \algUnitI(C) \circ_h \id_{2v}(\Psi)) = t\and
    s = \algUnitE(s \circ_{2v} (\id_{2v}(\Phi) \circ_h \algUnitI(C) \circ_h \id_{2v}(\Psi)))
    
    % Tensor
    \inferrule
    {P : \algHarr(C,D) \and Q : \algHarr(D,E)}
    {\algTensor(P,Q) : \algHarr(C,E)}

    \algTensor(P,Q) \circ_{hs} \gamma = \algTensor(P \circ_{hs} \gamma, Q \circ_{hs} \gamma)\and
    \algTensor(P,Q) \circ_{r} (f,g) = \algTensor(P \circ_r (f,\id), Q \circ_r (\id,g))

    \algTensorI(P,Q) : \algHSubst(i(P) \circ_h i(Q),\id,\id,i(\algTensor(P,Q)))

    \inferrule
    {t : \algHSubst(\Phi \circ_h i(P) \circ_h i(Q) \circ_h \Psi, f,g,i(R))}
    {\algTensorE(t) : \algHSubst(\Phi \circ_h i(\algTensor(P,Q)) \circ_h \Psi, f,g,i(R))}

    \algTensorE(t) \circ_{2v} (\id_{2v}(\Phi) \circ_{2h} \algTensorI(P,Q) \circ_{2h} \id_{2v}(\Psi)) = t\and
    s = \algTensorE(s \circ_{2v} (\id_{2v}(\Phi) \circ_{2h} \algTensorI(P,Q) \circ_{2h} \id_{2v}(\Psi)))
    
    % Cat Multiplicatives
    % Cat of Elements
    \inferrule
    {P : \algHarr(\Gamma,C,D)}
    {\algElts(C,D,P) : \algCat(\Gamma)}

    \inferrule
    {}
    {\pi^- : \algVarr(\algElts(C,D,P),C)}

    \inferrule
    {}
    {\pi^+ : \algVarr(\algElts(C,D,P),D)}

    \inferrule
    {}
    {\pi^p : \algHSubst(\id_h(\algElts(C,D,P)),\pi^-,\pi^+,P)}

    \inferrule
    {f : \algVarr(E,C) \and g : \algVarr(E,D)\and
      t : \algHSubst(\id_h(E), f, g, P)
    }
    {\algEltsI(t,f,g) : \algVarr(E,\algElts(C,D,P))}

    \pi^- \circ_v \algEltsI(t,f,g) = f\and
    \pi^+ \circ_v \algEltsI(t,f,g) = g\and
    \pi^p \circ_{2v} \algEltsI(t,f,g) = \alpha

    \inferrule
    {h : \algVarr(E,\algElts(C,D,P))}
    {h = \algEltsI(\pi^-\circ_v h,\pi^+\circ_vh,\pi^p \circ_{2v} \id_{2h}(h))}
    
    % Presheaf cats
    
    % Cat Additives
    % Product
    % Terminal

    % Prof Additives
    % Product
    % Terminal
  \end{mathpar}
  \caption{Multiplicatives}
\end{figure}

\begin{figure}
  \begin{mathpar}
    % Internalization
    % Cats
    \inferrule
    {}
    {\algCatTy : \algTy(1)}

    \inferrule
    {C : \algCat(\Gamma)}
    {\algCatQt(C) : \algTm(\Gamma,\algCatTy \circ !)}

    \algCatQt(C) \circ_{tms} \gamma = \algCatQt(C \circ_{cs} \gamma)\and

    \inferrule
    {M : \algTm(\Gamma,\algCatTy \circ !)}
    {\algCatUnqt(M) : \algCat(\Gamma)}

    \algCatUnqt(M) \circ_{cs} \gamma = \algCatUnqt(M \circ_{tms} \gamma)\and
    \algCatUnqt(\algCatQt(C)) = C\and
    M = \algCatQt(\algCatUnqt(M))
    
    % Functor
    % Prof
    % ProfCtx?
    % Trans
    \inferrule
    {P : \algHarr(\Gamma,C,C)}
    {\algTransTy(P) : \algTy(\Gamma)}

    \inferrule
    {t : \algHSubst(\id_h,\id_v,\id_v,i(P))}
    {\algTransQt(t) : \algTm(\Gamma,\algTransTy(P))}

    \algTransQt(t) \circ_{tms} \gamma = \algTransQt(t \circ_{2s} \gamma)
    
    \inferrule
    {}
    {\algTransUnqt : \algHSubst(ext(\Gamma,\algTransTy(P)),\id_h,\id_v,\id_v,i(P \circ_{hs} \pi_1))}

    \algTransUnqt \circ_{2s} (\gamma,\algTransQt(t)) = t\\

    % M : Tm(G,∀ α. P α α)
    % unquote : Subst(G,t : ∀ α:C. P α α; α:C, id,id, P α α)
    % M = unquote()
    M = \algTransQt(\algTransUnqt \circ_{2s} ext_s(\id,M))
  \end{mathpar}
  \caption{Internalization}
\end{figure}

\subsection{Equivalence between Natural Deduction and Combinatory Syntaxes}

Now we aim to prove that our natural deduction syntax of SCT is sound
and complete for the algebraic notion of hyperdoctines of double
categories with displays.
%
In category-theoretic terms, we want to show that the syntax of SCT
presents an \emph{initial object} in the category of models.
%
This is proved in two steps: first, we show that SCT is a model

\begin{corollary}
  There is a unique homomorphism of HDCDs
  \[ \eta : Comb \to SCT \]
\end{corollary}

\begin{construction}{Soundness}
  We construct a homomorphism of HDCDs
  \[ i : SCT \to Comb \]
  such that the composite
  \[ \eta \circ i : SCT \to SCT \]
  is equal to the identity.
\end{construction}
\begin{proof}
  \begin{align*}
    i(\cdot) &= 1\\
    i(\Gamma,x:A)  &= ext(i(\Gamma),ext(A))\\
    \cdots
    i(\Gamma \pipe \alpha:\cat C \vdash \alpha:\cat C) &= \id_v\\
    i(\Gamma \pipe \alpha:\cat C \vdash M a :\cat C) &= (\algVarrUnqt \circ_{vs} (\id, (i(M)))) \circ_v i(a)\\
    % Presheaf intro
    % elements intro
    i(\Gamma \pipe \alpha:\cat C \vdash (a,t,b) : \Sigma_{\alpha:\cat C;\beta:\cat D} R) &= \algEltsI(i(a),i(t),i(b))\\
    %
  \end{align*}
\end{proof}

\begin{construction}{Completeness}
  SCT presents a hyperdoctrine of double categories with displays.
\end{construction}
\begin{proof}
  First, the sorts.
  \begin{align*}
    \eta(\algCtx) &= \{ \Gamma | \Gamma \isadtctx \}\\
    \eta(\algSubst(\Delta,\Gamma)) &= \{ \gamma | \eta(\Delta) \vdash \gamma : \eta(\Gamma) \}\\
    \eta(\algTy(\Gamma)) &= \{ A |  \eta(\Gamma) \vdash A \isaTy \}\\
    \eta(\algTm(\Gamma,A)) &= \{ M | \eta(\Gamma) \vdash M : \eta(A) \}\\
    \eta(\algCat(\Gamma)) &= \{ \cat C | \eta(\Gamma) \vdash \cat C \isaCat \}\\
    \eta(\algVarr(\Gamma,C,D)) &= \{ a | \eta(\Gamma)|\alpha:\eta(C) \vdash a : \eta(D) \}\\
    \eta(\algHarr(\Gamma,C,D)) &= \{ R | \eta(\Gamma)|\alpha:\eta(C);\beta:\eta(D) \vdash R \isaSet \}\\
    \eta(\algHCtx(\Gamma,C,D)) &= \{ \Phi | \eta(\Gamma) \vdash \Phi \isavectx \wedge d^-(\Phi) = \eta(C) \and d^+(\Phi) = \eta(D) \}\\
    \eta(\algHSubst(\Gamma, \Psi, f, g, \Phi)) &= \{ \phi | \eta(\gamma)\pipe \eta(\Psi) \vdash \phi : \eta(\Phi) \wedge d^-\phi = \eta(f) \wedge d^+\phi = \eta(g) \}\\
    1 &= \cdot\\
    ! &= \cdot\\
    ext(\Gamma,A) = \Gamma, x:A \\
    \cdots\\
    \Phi \circ_h \Psi &= \Phi \jnctx \Psi\\
    \id_h(\cat C) &= \alpha:\cat C\\
    i(\cat C, \cat D, R) &= \alpha:\cat C, x: R, \beta:\cat D\\
    \phi \circ_{2v} \psi &= \phi[\psi]\\
    \id_{2v}(\Phi) &= \id_\Phi\\
    \id_{2h}(\alpha:\cat C \vdash a : \cat D) &= a/\beta\\
    \phi \circ_{2h} \psi &= \phi \jnctx \psi
  \end{align*}
\end{proof}

\documentclass[a4paper,UKenglish,cleveref, autoref, thm-restate]{lipics-v2021}

\usepackage{tikz-cd}
\usepackage{amssymb}
\usepackage{amsmath}
\usepackage{amsthm}
\usepackage{mathtools}
\usepackage{stmaryrd}
\usepackage{mathpartir}
\usepackage{bbm}
\usepackage{cleveref}

\DeclareMathOperator*{\colim}{colim}

%This is a template for producing LIPIcs articles. 
%See lipics-v2021-authors-guidelines.pdf for further information.
%for A4 paper format use option "a4paper", for US-letter use option "letterpaper"
%for british hyphenation rules use option "UKenglish", for american hyphenation rules use option "USenglish"
%for section-numbered lemmas etc., use "numberwithinsect"
%for enabling cleveref support, use "cleveref"
%for enabling autoref support, use "autoref"
%for anonymousing the authors (e.g. for double-blind review), add "anonymous"
%for enabling thm-restate support, use "thm-restate"
%for enabling a two-column layout for the author/affilation part (only applicable for > 6 authors), use "authorcolumns"
%for producing a PDF according the PDF/A standard, add "pdfa"

%\pdfoutput=1 %uncomment to ensure pdflatex processing (mandatatory e.g. to submit to arXiv)
%\hideLIPIcs  %uncomment to remove references to LIPIcs series (logo, DOI, ...), e.g. when preparing a pre-final version to be uploaded to arXiv or another public repository

%\graphicspath{{./graphics/}}%helpful if your graphic files are in another directory

\bibliographystyle{plainurl}% the mandatory bibstyle

\title{A Simple Type Theory for Category Theory} %TODO Please add

%\titlerunning{Dummy short title} %TODO optional, please use if title is longer than one line

\author{Max S. New}
       {University of Michigan Computer Science \& Engineering, Ann Arbor, Michigan, USA \and \url{http://www.maxsnew.com} }
       {maxsnew@umich.edu}
       {}
       {}
       %% {https://orcid.org/0000-0002-1825-0097}
       %% {(Optional) author-specific funding acknowledgements}%TODO mandatory, please use full name
       %% ; only 1 author per \author macro; first two parameters are mandatory, other parameters ca
       %% n be empty. Please provide at least the name of the affiliation and the country. The full address is optional. Use additional curly braces to indicate the correct name splitting when the last name consists of multiple name parts.

\author{Daniel R. Licata}{Wesleyan University, Middletown, Connecticut, USA}
       {dlicata@wesleyan.edu}{}{}

\authorrunning{M.S. New and D.R. Licata} %TODO mandatory. First: Use abbreviated first/middle names. Second (only in severe cases): Use first author plus 'et al.'

\Copyright{Max S. New and Daniel R. Licata} %TODO mandatory, please use full first names. LIPIcs license is "CC-BY";  http://creativecommons.org/licenses/by/3.0/

\ccsdesc[100]{\textcolor{red}{Replace ccsdesc macro with valid one}} %TODO mandatory: Please choose ACM 2012 classifications from https://dl.acm.org/ccs/ccs_flat.cfm 

\keywords{Dummy keyword} %TODO mandatory; please add comma-separated list of keywords

\category{} %optional, e.g. invited paper

\relatedversion{} %optional, e.g. full version hosted on arXiv, HAL, or other respository/website
%\relatedversiondetails[linktext={opt. text shown instead of the URL}, cite=DBLP:books/mk/GrayR93]{Classification (e.g. Full Version, Extended Version, Previous Version}{URL to related version} %linktext and cite are optional

%\supplement{}%optional, e.g. related research data, source code, ... hosted on a repository like zenodo, figshare, GitHub, ...
%\supplementdetails[linktext={opt. text shown instead of the URL}, cite=DBLP:books/mk/GrayR93, subcategory={Description, Subcategory}, swhid={Software Heritage Identifier}]{General Classification (e.g. Software, Dataset, Model, ...)}{URL to related version} %linktext, cite, and subcategory are optional

%\funding{(Optional) general funding statement \dots}%optional, to capture a funding statement, which applies to all authors. Please enter author specific funding statements as fifth argument of the \author macro.

%% \acknowledgements{I want to thank \dots}%optional

%\nolinenumbers %uncomment to disable line numbering

%Editor-only macros:: begin (do not touch as author)%%%%%%%%%%%%%%%%%%%%%%%%%%%%%%%%%%
\EventEditors{John Q. Open and Joan R. Access}
\EventNoEds{2}
\EventLongTitle{42nd Conference on Very Important Topics (CVIT 2016)}
\EventShortTitle{CVIT 2016}
\EventAcronym{CVIT}
\EventYear{2016}
\EventDate{December 24--27, 2016}
\EventLocation{Little Whinging, United Kingdom}
\EventLogo{}
\SeriesVolume{42}
\ArticleNo{23}
%%%%%%%%%%%%%%%%%%%%%%%%%%%%%%%%%%%%%%%%%%%%%%%%%%%%%%

\begin{document}
%% \newtheorem{theorem}{Theorem}
%% \newtheorem{lemma}{Lemma}
%% \newtheorem{construction}{Construction}
%% \newtheorem{definition}{Definition}
\newcommand{\pto}{\nrightarrow}
\newcommand{\pfrom}{\nleftarrow}
\newcommand{\vcat}{\mathcal}
\newcommand{\cat}{\mathbbm}
\newcommand{\isaSet}{\,\,\textrm{Set}}
\newcommand{\isaTy}{\,\,\textrm{Type}}
\newcommand{\isaCat}{\,\,\textrm{Cat}}
\newcommand{\Set}{\textrm{Set}}
\newcommand{\vtkmnd}{\mathbb{K}\text{Mod} (\vcat{V},T)}
\newcommand{\rmod}{\text{RMod}}
\newcommand{\lmod}{\text{LMod}}
\newcommand{\varr}[2]{\text{Fun}{#1}{#2}}
\newcommand{\harr}[2]{\text{Rel}{#1}{#2}}
\newcommand{\harrapp}[3]{#1(#2;#3)}

\newcommand{\jnctx}{\curlyveedownarrow}

\newcommand{\id}{\textrm{id}}
\newcommand{\for}{\textrm{for}\,}
\newcommand{\when}{\textrm{when}\,}
\newcommand{\lett}{\textrm{let}\,}
%% \newcommand{\sort}{\,\,\textrm{sort}}
\newcommand{\Sort}{\textrm{Sort}}
\newcommand{\isadtctx}{\,\,\textrm{type context}}
\newcommand{\isavectx}{\,\,\textrm{trans. context}}
%% \newcommand{\obctx}{\,\,\textrm{object context}}
\newcommand{\boundary}{\,\,\textrm{set context}}
\newcommand{\prof}{\,\,\textrm{span}}
\newcommand{\subst}{\,\,\textrm{subst}}
\newcommand{\sigctx}{\,\,\textrm{sig-ctx}}
\newcommand{\sig}{\,\,\textrm{sig}}
\newcommand{\pipe}{\mathrel{|}}

\newcommand{\punitinXfromYtoZ}[3]{#2 \mathop{\to_{#1}} #3}
\newcommand{\punitrefl}[1]{\textrm{id}_{#1}}
\newcommand{\punitelimtoYwithkontZ}[2]{\textrm{ind}_{\to}(#1,#2)}

\newcommand{\odotexists}[1]{\mathop{\overset{\exists #1}\odot}}
\newcommand{\tensorexistsXwithYandZ}[3]{#2 \odotexists{#1} #3}
\newcommand{\tensorintroofXandY}[2]{\textrm{pair}_\odot(#1,#2)}
\newcommand{\tensorintroatXwithYandZ}[3]{(#2,#1,#3)}
\newcommand{\tensorelimXtoYwithkontZ}[3]{\textrm{ind}_{\to}(#1,#2,#3)}
\newcommand{\tensorelimfromWXtoYwithkontZ}[4]{\textrm{ind}_{\odot}(#1,#2,#3,#4)}
\newcommand{\tensorelimVatWtoXtoYwithkontZ}[5]{\textrm{ind}_{\to}(#4,#2,#1,#3,#5)}

\newcommand{\tlwith}[1]{\mathop{\prescript{#1}{}\triangleleft}}
\newcommand{\tlforall}[1]{\tlwith{\forall #1}}

\newcommand{\trwith}[1]{\mathop{\triangleright^{#1}}}
\newcommand{\trforall}[1]{\trwith {\forall #1}}

\newcommand{\homrallXYtoZ}[3]{#2 \trforall {#1} #3}

\newcommand{\homlallXYtoZ}[3]{#3 \tlforall {#1} #2}

\newcommand{\homrlambdaXatYdotZ}[3]{\lambda^\triangleright (#1, #2). #3}
\newcommand{\homllambdaXatYdotZ}[3]{\lambda^\triangleleft (#2, #1). #3}

\newcommand{\homrappXtoYatZ}[3]{#1 \trwith{#3} #2}
\newcommand{\homlappXtoYatZ}[3]{#2 \tlwith{#3} #1}

\newcommand{\homunary}[4]{\pendallXdotY {#1} {\homrallXYtoZ {#3} {#2} {#4}}}
\newcommand{\hombinary}[6]{\pendallXdotY {#1} {\homrallXYtoZ {#3} {#2} \homrallXYtoZ {#5} {#4} {#6}}}

\newcommand{\lambdaunary}[4]{\pendlambdaXdotY {#1} {\homrlambdaXatYdotZ {#2} {#3} {#4}}}
\newcommand{\lambdabinary}[6]{\lambdaunary{#1}{#2}{#3}{\homrlambdaXatYdotZ{#4}{#5}{#6}}}
\newcommand{\lambdatrinary}[8]{\lambdabinary{#1}{#2}{#3}{#4}{#5}{\homrlambdaXatYdotZ{#6}{#7}{#8}}}

\newcommand{\appunary}[4]{\homrappXtoYatZ {\pendappXtoY {#1} {#2}} {#3} {#4}}
\newcommand{\appbinary}[6]{\homrappXtoYatZ {\appunary{#1}{#2}{#3}{#4}}{#5}{#6}}
\newcommand{\apptrinary}[8]{\homrappXtoYatZ {\appbinary{#1}{#2}{#3}{#4}{#5}{#6}}{#7}{#8}}

\newcommand{\pendallXdotY}[2]{\forall #1. #2}
\newcommand{\pendlambdaXdotY}[2]{\lambda #1. #2}
\newcommand{\pendappXtoY}[2]{#1^{#2}}

\newcommand{\Cat}{\textrm{Cat}}
\newcommand{\Id}[3]{\textrm{Id} #1\,#2\,#3}
\newcommand{\vlambda}[1]{\lambda^{F}{#1}.}
\newcommand{\hlambda}[2]{\lambda^{P}{#1};{#2}.}

\newcommand{\equalizeVofWbyXeqYatZ}[5]{\{ {#1} : {#2} \pipe {#3} = {#4} : {#5} \}}

\maketitle

%TODO mandatory: add short abstract of the document
\begin{abstract}
  We present a domain-specific type theory for constructions and
  proofs in category theory. The type theory axiomatizes notions of
  category, functor, profunctor and a generalized form of natural
  transformations. By restricting the type theory to a form of ordered
  logic, we guarantee that all mappings between categories are
  functorial and all transformations are natural by construction, with
  no separate proof necessary. Important category theoretic proofs
  such as the Yoneda lemma and Co-yoneda lemma become simple type
  theoretic proofs about the relationship between unit, tensor and
  (ordered) function types.

  We show that the type theory is sound for models called
  \emph{virtual equipments}, a kind of double category which is known
  to model not just ordinary categories but enriched, internal and
  generalized multi-categories as well. This means that while proofs
  and constructions in our theory read like typical category-theoretic
  constructions, the proofs automatically apply to these generalized
  categories. Our type theory simplifies arguments over the algebraic
  description of virtual equipments by modeling the algebraic
  operation of restriction of profunctors along functors as instead
  the strict syntactic substitution.
\end{abstract}

\section{Introduction}

Category theory is a branch of mathematics that formalizes the
algebraic structure of function composition. The high level of
abstraction allows for succinct proofs relevant to structures across a
broad variety of mathematical domains. Category theory is of
particular interest in the theory of programming languages and proof
assistants, where it is commonly used to aid in the design of typed
languages, proof assistants and functional programming abstractions.

Novices in category theory are often overwhelmed by the number of
naturality and functoriality side conditions that abound in otherwise
simple proofs in category theory. For instance consider the proof of
the Yoneda lemma, often the first ``big'' proof in an introductory
category theory text. ... Here the reader is left to themselves to
show that the ``evaluation at identity'' morphism and its inverse are
actually natural transformations. They will then work through a
diagram chase to determine that this naturality condition indeed
holds.

More experienced users of category theory in programming language
theory find that for modeling computational effects such as general
recursion or indexed-state, or modeling logics for parametricity,
gradual typing, or step-indexed logical relations involves not just
ordinary category theory, but more general notions such as enriched,
internal or multi-categories. However, model construction theorems and
especially soundness and completeness theorems in these areas will
require their own, even more complex versions of the same common
lemmas of category theory, in addition to the domain-specific results
such as existence of fixed points.

At this level of generality, the proofs can be difficult to
understand, as they often involve some combination of the original
category theoretic proof with some additional side-conditions that
everythinng is in fact appropriately step-indexed, continuous etc.
%
The solution to taming this complexity is to prove these general
theorems at a higher level of generality, for instance, proving them
in some abstract 2-category of enriched/internal categories, functors
and transformations.
%
For proofs in generalized category theory specifically, the notion of
a \emph{virtual equipment} has been shown to be occupy an algebraic
sweet-spot of being complex enough to express most basic constructions
(functors, profunctors, natural transformations, adjunctions, kan
extensions, etc) but also general enough to encompass internal,
enriched and ``generalized multi-categories'', which include ordinary
flavors of multi-categories like the cartesian multi-categories that
model sequents in natural deduction, but even structures such as
topological spaces.
%
However, there are two difficulties with this approach.
%
First, the abstract arguments can be difficult to understand, and can
seem quite hard to connect to the original argument for ordinary
category theory.
%
Second, the algebraic structure of a virtual equipment is quite
unwieldy to manipulate. The equivalent of a 2-cell is a kind of
``multi-2-cell'' with many inputs and a single output, in
diagrammatic form these are rendered as

\[ % https://tikzcd.yichuanshen.de/#N4Igdg9gJgpgziAXAbVABwnAlgFyxMJZABgBpiBdUkANwEMAbAVxiRAGEB9YkAX1PSZc+QigCM5KrUYs2XMXwEgM2PASIAmSdXrNWiDpw2LBqkUQDM26XrYAdOwygQcCfqeHqUAFmu7ZBlzAYADUYrwmykJqoiSkYlL++iCRKp6xvgk6Msl8UjBQAObwRKAAZgBOEAC2SBIgOBBIWjYBIABKnAruIJU1zdSNSFatyZ3GPX21iCNDiL6jbJ2Ek1XTZA1NiPVJS9yRU0gbcwCs2bYGAGIHa0hnm0gAbOdtAMo3-fODW8+LBgDieV4QA
\begin{tikzcd}
C_0 \arrow[r, "R_0"] \arrow[d, "F"] & C_1 \arrow[r, "R_1"] & C_2 \arrow[r, "R_2"] & \ldots \arrow[r, "R_n"] & C_{n+1} \arrow[d, "G"] \\
{} \arrow[rrrr, "S"]                &                      &                      &                         & {}                    
\end{tikzcd} \]

This 2-cell has one output pro-arrow, many input pro-arrows which must
match the source from one to the target object of the next and a left
and right ``frame'' arrow. These can be composed in a manner similar
to multi-categories, where you can compose this with many input
2-cells, one for each input pro-arrow.
%
These then have complex associativity and unit rules and side
conditions that say the framing arrows compose alongside the 2-cells.

We solve both of these issues by developing an alternative \emph{type
theoretic syntax} for the constructions in a virtual equipment.
%
In this syntax the ``multi-2-cell'' above instead becomes a natural
transformation judgment
\[ \Phi \vdash t : R[F(x);G(y)] \]
We write natural transformations in a natural deduction style: a
transformation is written as a ``parameterized'' element of the output
profunctor. Furthermore the restriction operators for vertical arrows
into horizontal arrows is naturally modeled by a syntactic
substitution.
%
This interpretation of restriction as substitution both matches the
semantic intuition for the restriction operation as well as greatly
simplifies reasoning about composition of transformations.


%% Experienced category theorists have no trouble with this, as they are
%% familiar with a great many principles for constructing transformations
%% that are guaranteed to be natural. Consider an analogy with
%% calculus. When a mathematician writes a formula such as

%% $$\sin(f(x) - g(x))$$

%% They do not need to spend time with an explicit $\epsilon,\delta$
%% proof that the function is continuous or furthermore smooth. Instead,
%% they follow syntactic principles that compositions of continuous
%% functions are continuous.

%% There are similar syntactic notions for guaranteeing that
%% transformations are natural.

\subsection{Category Theory with Profunctors}

The syntax of Judgment Theory gives not just a syntax for the familiar
trio of category, functor and natural transformation, but additionally
a concept that is not usually highlighted in introductory category
theory texts: the profunctor. A simple intuition is that profunctors
are to functors as relations are to functions. Indeed, the syntax for
judgment theory has a straightforward semantics in the category of
sets and relations.

The reason profunctors are not commonly introduced is that they are
seen as an auxiliary notion: a profunctor $R : C \pto D$ is
typically defined as a functor $C^{\textrm{op}}\times D \to
\textrm{Set}$. However this definition requires two concepts that are
somewhat problematic: the opposite category $C^\textrm{op}$ and the
category of (small) sets $\textrm{Set}$. However, these notions do not
necessarily exist in all of our intended applications. First, we are
interested in semantics where are objects are \emph{multicategories}
where morphisms have multiple inputs but only a single output. In this
case the opposite of a multicategory is not a multicategory at all but
instead some kind of ``co-multicategory'' with multiple outputs but
one input. Second, having a universe object like $\textrm{Set}$ is also

\subsection{Applications to Enriched, Internal and Multicategory Theory}

The syntax of judgment theory is an internal language for
\emph{virtual equipments}. These are a kind of joint generalization of
double categories and multicategories.

Notably, the definition of a category in \cref{sec:internal} can be
applied as an extremely general notion of category, including internal
categories and enriched categories. This general notion is often
called a ``generalized multi-category'' and encompasses an astonishing
range of mathematical structures, including categories, topological
spaces, metric spaces, etc.

\subsection{Paper Overview}

To demonstrate the utility of SCT, we proceed as follows
\begin{enumerate}
\item In \cref{sec:syntax} we introduce the syntax of Simple Category
  Theory.
\item In \cref{sec:synthetic} we show how common theorems of category
  theory such as the (co)-Yoneda lemma, equivalence of different
  formulations of adjunctions, and preservation of limits can all be
  cleanly reproduced in SCT.
\item In \cref{sec:internal}, we show how we can use SCT to define
  \emph{internal categories}, which allows us to prove additional
  theorems that are not formalizable as synthetic theorems.
\item In \cref{sec:semantics}, we show that the syntax of SCT is sound
  for models in an arbitrary virtual double category with
  corresponding universal constructions.
\item In \cref{sec:discussion}, we discuss related type theories and
  potential extensions to model multi-variable and higher-order
  categorical constructions.
\end{enumerate}

\section{Syntax of SCT}

The syntactic forms of Simple Category Theory (SCT) are given in
\cref{fig:syntax}.
%
First, we have the types and terms of the logic over our category
theory. Types include the usual dependent sum, product and identity
types of dependent type theory, as well as several forms that allow us
to abstract over the category theoretic constructions: a type $\Cat$
of categories, a type $\varr{\cat C}{\cat D}$ for functors from $\cat
C$ to $\cat D$, a type $\harr{\cat C}{\cat D}$ for profunctors
contravariant in $\cat C$ and covariant in $\cat D$, and finally a
type $\pendallXdotY {\alpha:{\cat C}} R$ that, when combined with
other constructions on profunctors, allows for abstraction over
arbitrary natural transformations.
%
Terms $L,M,N$ describe the elements of our logic's types, and include
the usual constructions as well as forms for defining categories,
functors, profunctors and natural transformations.
%
Next, we have a syntax for categories, which are written using a
blackboard font $\cat C, \cat D, \cat E$. Categories include terms $M$
of the category type (allowing for abstraction), in addition to
primitives for product and unit categories.
%
Next, we have a syntax for objects of a category, which, because they
are typed with respect to an input variable, provide a syntax for
functors.
%
The objects we can define are object variables, application of a
functor $M a$, and pairs of objects (for the product category) and the
unique object of the unit category.
%
Next, $P,Q,R$ stand for \emph{sets}, but similar to objects, they are
always typed with respect to a contravariant and covariant variable,
and so denote profunctors.
%
These include first the multiplicatives: the hom set
$\punitinXfromYtoZ {\cat C} a b$ representing the morphisms from $a$
to $b$ in $\cat C$, a tensor product $\tensorexistsXwithYandZ
{\beta:\cat D} P Q$, and two kinds of ``ordered function types''
$\homrallXYtoZ \beta P Q$ and $\homlallXYtoZ \alpha R S$. 
%
We also include application of a profunctor term $M$ to two objects,
and cartesian product and unit sets.
%
Next, we have transformations $s,t,u$, which are typed with respect
to (0 or more) input and (one) output set.

Finally, we have three forms of contexts, type contexts $\Gamma$, set
contexts $\Xi$ and transformation contexts $\Phi$. A type context
$\Gamma$ is the familiar notion of context from dependent type
theory. We use uppercase $X,Y,Z$ for term variables to distinguish
them from set variables. Every set expression $P$ must depend on an
object of one category covariantly and an object of one category
contravariantly, however these are allowed to be the same
object. These two scenarios are captured by the notion of set context
$\Xi$. The set context $\alpha:\cat C; \beta: \cat D$ gives the common
case where the contravariant variable $\alpha$ is different from the
covariant variable $\beta$, and the set context $\alpha: \cat C$ alone
gives the case where the profunctor depends on the same variable both
covariantly and contravariantly. Note that we write object variables
as greek letters $\alpha, \beta$. Transformation contexts $\Phi$
describe the inputs to a natural transformation. The input variables
of a natural transformation $x: R$ stand for elements of sets, but as
just described, sets always have covariant and contravariant
dependencies. For this reason, transformation contexts include
category variables $\alpha: \cat C$ interleaved with the set variables
$x : R$, a typical context $\Phi$ would look like $\alpha_1:\cat C_1,
x_1:R_1, \alpha_2: \cat C_2, x_2 : R_2, \alpha_3: \cat C_3$, where
$R_1$ is parameterized by $\alpha_1$ contravariantly and $\alpha_2$
covariantly and $R_2$ is similarly parameterized by
$\alpha_2;\alpha_3$. Note that we consider transformation contexts up
to associativity in that $(\Phi, x:R, \Psi), y : S, \Sigma$ should be
considered equal to $\Phi, x:R, (\Psi, y: S, \Sigma)$.  In this
calculus there is no single ``empty'' transformation context, but
rather the empty context is one that consists of a single category
variable $\alpha:\cat C$ and no set variables. Each form of context
has a corresponding form of substitution, denoted using the lower-case
form of the same greek letters.

\begin{figure}
  \begin{mathpar}
    \begin{array}{rccl}
      \textrm{Type} & A,B,C & ::= &
      \sum_{X:A} B \pipe \prod_{X:A} B \pipe \Id A M N \\
      &&& \Cat \pipe \varr{\cat C}{\cat D}\pipe \harr {\cat C} {\cat D}\pipe \pendallXdotY {\alpha:{\cat C}} R\\
      \textrm{Term} & L,M,N & ::= & X \pipe \lambda X:M. N \pipe M\,N \pipe (M,N) \pipe \pi_1 M \pipe \pi_2 M\\
      &&&\lceil \cat C \rceil \pipe \vlambda {\alpha:\cat C} a \pipe \hlambda {\alpha:\cat C}{\beta: \cat D} R \ldots \\
      \textrm{Categories} &\cat C, \cat D, \cat E & ::= & M \pipe \cat C \times \cat D \pipe \cat 1\\
      \textrm{Objects} & a, b, c & ::= & \alpha \pipe M a \pipe (a,b) \pipe () \ldots\\
      \textrm{Sets} & P,Q,R & ::= & \punitinXfromYtoZ {\cat C} a b \pipe \tensorexistsXwithYandZ {\beta} P Q \pipe \homrallXYtoZ \beta P Q \pipe \homlallXYtoZ \alpha R S \\
      &&& M(a,b) \pipe 1 \pipe P \times Q\\
      \textrm{Transformations} & s,t,u & ::= & \ldots \\
      \\
      \textrm{Type Context} & \Gamma, \Delta & ::= & \cdot \pipe \Gamma, X : A\\
      \textrm{Category Context} & \Xi, Z & ::= & \alpha:\cat C \pipe \alpha:\cat C;\beta: \cat D\\
      \textrm{Set Context} & \Phi,\Psi & ::= & \alpha : \cat C \pipe \Phi , x : P , \Psi\\
      \textrm{Term Substitution} & \gamma, \delta & ::= & \cdot \pipe \gamma, M/X\\
      \textrm{Object Substitution} & \xi, \zeta & ::= & a/\alpha \pipe a/\alpha;b/\beta\\
      \textrm{Transformation Substitution} & \phi, \psi & ::= & a/\alpha \pipe \phi,s/x, \psi      
    \end{array}
  \end{mathpar}
  \caption{Simple Category Theory Syntactic Forms}
  \label{fig:syntax}
\end{figure}

Next, \cref{fig:formation} shows the \emph{formation} rules for these
various syntactic forms. All other forms are typed with respect to a
well-formed type context $\Gamma$. Objects are typed additionally with
an input object variable $\alpha : \cat C$ and an output category
$\cat D$; in the semantics these objects are modeled as functors. A
set $S$ is typed with respect to a set context $\Xi$ to describe its
covariant/contravariant dependence on some input objects. These are
semantically modeled as profunctors. A transformation $s$ has a
context of transformation variable and a single set output. To be well
formed, the context and set must be parameterized by the same
contravariant and covariant object variables. To ensure this, we use a
coercion operation $\underline \Phi$ that erases everything in the
context but the left-most and right-most object variables.

\begin{figure}
  \begin{mathpar}
    \inferrule*[right=TyCtxForm]
        {}
        {\Gamma \isadtctx}

    \inferrule*[right=Cat-form]
    {\Gamma \isadtctx}
    {\Gamma\vdash \cat C \isaCat}

    \inferrule*[right=Ob-form]
    {\Gamma \vdash \cat C\isaCat\and \Gamma \vdash \cat D \isaCat}
    {\Gamma \pipe \alpha : \cat C \vdash a : \cat D}\\

    \inferrule*[right=SetCtxForm]
    {\Gamma \isadtctx}
    {\Gamma \vdash \Xi \boundary}

    \inferrule*[right=SetForm]
    {\Gamma \isadtctx \and \Gamma \vdash \Xi \boundary}
    {\Gamma \pipe \Xi \vdash S \isaSet}

    \inferrule*[right=TransCtxForm]
    {\Gamma \isadtctx}
     {\Gamma \vdash \Phi \isavectx}

    \inferrule*[right=TransFormation]
    {\Gamma \vdash \Phi \isavectx\and
     \Gamma\pipe \underline \Phi \vdash R \isaSet}
    {\Gamma \pipe \Phi \vdash s : R}

    \inferrule
    {\Delta \isadtctx \and \Gamma \isadtctx}
    {\gamma : \Delta \to \Gamma}
    
    \inferrule*[right=Boundary Map formation]
    {\Gamma \vdash Z \boundary \and \Gamma \vdash \Xi \boundary}
    {\Gamma \vdash \xi : Z \to \Xi}

    \inferrule*[right=Cat Subst formation]
    {\Gamma \vdash \Psi \isavectx \and \Gamma \vdash \Phi \isavectx}
    {\Gamma \vdash \phi : \Psi \to \Phi}
  \end{mathpar}
  \caption{Formation Rules}
  \label{fig:formation}
\end{figure}

%% The following figures present the basic formation, primitive and
%% admissible equality rules of our type theory, an internal language for
%% virtual double categories with restrictions.
%% %
%% The type theory has 4 central concepts, which we will refer to
%% sometimes using type theoretic/set theoretic and at other times
%% category-theoretic terminology.
%% %
%% In type theoretic terminology, judgment theory has \emph{sorts} $\cat
%% C, \cat D, \cat E$, and \emph{types} $A,B,C$ which have a given sort
%% and are also \emph{parameterized} by an object of some other sort, and
%% so can also be thought of as \emph{functions} with input and output
%% sorts.
%% %
%% Next there are \emph{judgments} $P,Q,R,S$ which are parameterized by
%% two variables.
%% %
%% In $\alpha^o:\cat C;\beta : \cat D \pipe R$, we say $R$ depends on
%% $\alpha$ \emph{contravariantly} and in $\beta$ \emph{covariantly}.
%% %
%% Finally, we have \emph{terms} $s,t,u$ that are elements of some
%% judgment $R$ but are also parameterized by elements of some other
%% judgments which are given by a \emph{context} $\Phi$, which is a
%% ``compatible string'' of judgments where the covariant variable of one
%% judgment is the same as the contravariant variable of the next.
%% %
%% Since these terms are parameterized by other terms, we can see these
%% as \emph{inference rules} for the judgments.

%% At other times we will want to think of these using set-theoretic
%% terminology, in which case we have sets $\cat C,\cat D, \cat E$,
%% functions/elements $A,B,C$, \emph{spans} $R,\ldots$ and morphisms of
%% spans $s,t,u$.
%% %
%% Finally when we develop multicategory theory we will use
%% category-theoretic terminology, in which the structure is most
%% familiar and we have categories, objects/functors,
%% \emph{profunctors}/bimodules and 2-cells/homomorphisms of bimodules.

%% Figure\ref{fig:formation} introduces the formation rules for the basic
%% constructs. Sorts give a syntax for objects of the double
%% category. Types have an output sort and a sorted input variable, and
%% model vertical arrows. Next, judgments have two sorted variables, one
%% contravariant and one covariant, which can be the same
%% variable. Judgments model the horizontal arrows. Next are contexts,
%% whose formation rule is the same as judgments, and which give a syntax
%% for ``compatible strings'' of judgments. Then terms have an input
%% context and an output judgment, which must have the same covariant and
%% contravaiant variables, modeling ``globular'' 2-cells. Finally, we
%% have substitutions for terms, which are typed with respect to an input
%% context, an output context and two types, forming a kind of
%% square. These model the ``compatible string'' of 2-cells that arise in
%% the composition laws.



%% Next, Figure~\ref{fig:basic-rules} gives the basic structural rules of
%% judgment theory, which model composition and restrictions in a virtual
%% double category. First, we have varialbes and substitution for types,
%% modeling identity and composition of vertical arrows. Next, we have
%% restriction of a judgment to more specific types, which is given by
%% substitution of types for the judgment's parameter variables. Next, we
%% have rules for constructing an ``empty'' context, which is just a
In typical natural-deduction style, all syntactic forms typed in
context admit an action of substitution. For terms, this is the
ordinary type theoretic substitution. Naturally objects $\alpha:\cat C
\vdash a : \cat D$ can be substituted for object variables $\beta :
\cat D$ in other objects. We can also substitute objects into
\emph{sets}. For instance, if we have a hom set in a category
$\punitinXfromYtoZ {\cat C} \alpha {\alpha'}$, and we can define
objects $a$ and $a'$ in the category $\cat C$, then we can substitute
those in to form the hom set between $a$ and $a'$: $\punitinXfromYtoZ
{\cat C} a {a'}$. This semantically is a ``restriction'' of a
profunctor along two functors, and modeling this operation simply as a
substitution considerably simplifies reasoning using profunctors.
%
Finally we have the action of substitution on natural
transformations. A natural transformation is indexed by set variables
$x:R$, whose sets are each parameterized by a contravariant and a
covariant category variable $\alpha;\beta$. We can think of natural
transformations as \emph{polymorphic} in the categories involved, and
so when we make a transformation substitution, we substitute not just
the set variables, but we also \emph{instantiate} the polymorphic
category variables with objects.

\cref{fig:contexts} details the typing rules for contexts and the
typing for variables and the admissible actions of substitution (the
definition of substitution is straightforward and found in the
appendix). the rules for term and object variables are as usual.
%
An object substitution into a single object context $\alpha:\cat C$
consists of an object $a : \cat C$ parameterized by an object of
another category.
%
An object substitution into a pair of co- and contra-variant variables
$\alpha:\cat C;\beta:\cat D$ is a pair of an object substituted for
$\alpha$ and one for $\beta$, each typed with respect to the
appropriate variable of an object context $\Xi$.
%
Note that this allows for $\Xi$ to consist of either two variables or
a single variable.
%
When
constructing a transformation context $\Phi, x : R, \Psi$, we need to
ensure that the contravariant dependence of $R$ matches the covariant
dependence of $\Phi$ and vice-versa for $\Psi$. We define this with
auxiliary metafunctions $d^+,d^-$ which extract the covariant and
contravariant object variable of a transformation context.
%
The rules for transformation variables and substitutions define the
use of variables to be \emph{linear} (each variables is used exactly
once) and \emph{ordered}, they occur in the term in the same order
that they appear in the context.
%
The rule for a transformation variables says a variable $x$ can only
be used when it is the sole variable in the context.
%
A transformation subsitution into an ``empty'' transformation context
$\alpha: \cat C$ is the same as an object substitution into
$\alpha:\cat C$.
%
The rule for constructing a transformation substitution into a
non-empty context $\Phi_1,x:R,\Phi_2$ says we can construct a
substitution by constructing substitutions $\phi_1,\phi_2$ for
$\Phi_1$ and $\Phi_2$ as well as a transformation $s$ into an
instantiation of $R$ where the input contexts come from splitting the
input context $\Psi$ into three components, and ensuring that the
instantiation of $R$ agrees with how $\phi_1,\phi_2$ instantiate the
variables. This uses two new metafunctions. First, $\Phi \jnctx \Psi$
is an operation for concatenating two contexts when $d^+(\Phi) =
d^-(\Psi)$. Next, we extend the boundary maps $d^\pm$ to substitutions
to extract their object substitutions.

\begin{figure}
  \begin{mathpar}
    \inferrule{}{\cdot \isadtctx}

    \inferrule
    {\Gamma \isadtctx \and \Gamma \vdash A \isaTy}
    {\Gamma , X : A \isadtctx}\\

    \inferrule
    {X : A \in \Gamma}
    {\Gamma \vdash X : A}

    \inferrule
    {}
    {\alpha: \cat C \vdash \alpha: \cat C}

    \inferrule{\Gamma \vdash \cat C : \isaCat}{\Gamma \vdash \alpha: \cat C \boundary}

    \inferrule{\Gamma \vdash \cat C : \isaCat\and
      \Gamma\vdash \cat D : \isaCat}{\Gamma \vdash \alpha: \cat C;\beta:\cat D \boundary}\\


    \inferrule*
        {\Gamma \pipe \alpha':\cat C' \vdash A : \cat C}
        {\Gamma \vdash a/\alpha : \alpha':\cat C' \to \alpha:\cat C}

    \inferrule*
    {\Gamma \pipe d^-\Xi \vdash a : \cat C\and
     \Gamma \pipe d^+\Xi \vdash b : \cat D
    }
    {\Gamma \vdash (a/\alpha;b/\beta) : \Xi \to (\alpha:\cat C;\beta:\cat D)}\\
    
    \inferrule
        {\Gamma \vdash \cat C : \isaCat}
        {\Gamma \vdash \alpha: \cat C \isavectx}

    \inferrule
    {\Gamma \vdash \Phi \isavectx\and
    \Gamma \vdash \Psi \isavectx \and
    \Gamma \pipe d^+\Phi; d^- \Psi \vdash R \isaSet}
    {\Gamma \vdash \Phi, x:R, \Psi \isavectx}

    \inferrule
    {}
    {\Gamma \pipe \alpha:\cat C, x : R, \beta: \cat D \vdash x : R}

    \inferrule
    {\Psi = \Psi_1 \jnctx \Psi_M \jnctx \Psi_2\and
      \Gamma \vdash \phi_1 : \Psi_1 \to \Phi_1\and
      \Gamma \pipe \Psi_s \vdash s : R[d^+\phi_1;d^-\phi_2] \and
      \Gamma \vdash \phi_2 : \Psi_2 \to \Phi_2
      }
    {\Gamma \vdash \phi_1,t/x,\phi_2 : \Psi \to \Phi_1,x:R,\Phi_2}
  \end{mathpar}
  \begin{align*}
    \underline{\alpha:C} &= \alpha:C\\
    \underline{\Phi,x:R,\Psi} &= d^-\Phi; d^+{\Psi}
  \end{align*}
  \begin{align*}
    d^{\pm}(\alpha:\cat C) &= \alpha : \cat C\\
    d^{-}(\Phi,x:R,\Psi) &= d^-\Phi\\
    d^{+}(\Phi,x:R,\Psi) &= d^-\Psi
  \end{align*}
    \begin{align*}
    d^{\pm}(a/\alpha) &= a/\alpha\\
    d^{-}(\phi,t/x,\psi) &= d^-\phi\\
    d^{+}(\phi,t/x,\psi) &= d^+\psi\\
  \end{align*}
  \caption{Basic Rules of Judgment Theory}
  \label{fig:contexts}
\end{figure}

Now that we have established a judgmental structure, in typical type
theoretic fashion, our type constructs follow naturally as
internalizing the judgments. Though the positive multiplicatives are
more familiar, we start by describing the negative multiplicatives,
since they are better behaved, and are useful to have already defined
when discussing the positives.

\subsection{Types}

Finally, we describe the \emph{end}, $\pendallXdotY {\alpha} P$, which
when combined with the left or right hom types described in the next
section allows for us to internalize arbitrary transformation sequents
$\Phi \vdash P$ as a type.
%
This type is well formed when $\alpha$ is the covariant \emph{and}
contravariant variable of $P$.
%
An element of the type is a polymorphic element of $P$, varying in
$\alpha$.
%
The introduction form is naturally a $\lambda$ abstracting over
$\alpha:\cat C$ and the elimination form instantiates $\alpha$ with an
arbitrary object $a : \cat C$.
%
We also include the necessary $\beta\eta$ equalities.

\begin{figure}
  \begin{mathpar}
    \inferrule
    {\Gamma \pipe \alpha:\cat C\vdash P \isaSet}
    {\Gamma \vdash \pendallXdotY {\alpha:{\cat C}} P \isaTy}

    \inferrule
    {\Gamma \pipe \alpha:\cat C\vdash t : P}
    {\Gamma \vdash \pendlambdaXdotY \alpha t : \pendallXdotY \alpha P}

    \inferrule
    {\Gamma \vdash M : \pendallXdotY \alpha P\and
     \Gamma \pipe \beta:\cat D \vdash a : \cat C}
    {\Gamma \pipe \beta:\cat D \vdash \pendappXtoY M a : P[a/\alpha]}

    \pendappXtoY {(\pendlambdaXdotY \alpha t)} a = t[a/\alpha]

    \inferrule*[right=2Cell$\eta$]
    {\Gamma \vdash M : \pendallXdotY \alpha P }
    {\Gamma \vdash M = \pendlambdaXdotY \alpha {\pendappXtoY M \alpha} : \pendallXdotY \alpha P}
  \end{mathpar}
  \caption{End}
  \label{fig:end}
\end{figure}

Semantically, this type corresponds to taking the end of a profunctor
$P : {\cat C}^{op} \times {\cat C} \to \Set$, typically written
\[ \int^{\alpha:\cat C} P \alpha \alpha \]

\subsection{Negative Multiplicatives}

Since our calculus of natural transformations is ordered linear, we
have two distinct ``function sets'', internalizing concatenation on
the left or right of the context. The typing rules for these are given
in \cref{fig:cotensor}. The two types are the \emph{right hom}
$\homrallXYtoZ \alpha R P$ and the \emph{left hom} $\homlallXYtoZ
\alpha R P$. As a type, we can think of these both as a kind of
polymorphic function type in that in addition to parameterizing an
element of the set $P$ by an element of the set $R$, they also
quantify over an object $\alpha$ that is in $P$ and $R$.

These behave like function types, but their typing is somewhat
involved. First consider the right hom $\homrallXYtoZ \alpha R
P$. This acts very much like a polymorphic function type $\forall
\alpha. R \to P$, but with restrictions on the object
variables. First, the object variable $\alpha$ must occur covariantly
in $P$ and contravariantly in $R$. Next, the contravariant variable of
the type is the contravariant variable of $P$ and the covariant
variable of the type is the \emph{contravariant} variable of $R$, with
the flip in variance intuitively justified since $R$ occurs in the
negative position of the function. The placement of the $\forall
\alpha$ on the right side of the triangle is intended to be suggestive
of the fact that $\alpha$ occurs \emph{covariantly} in the input and
output sets $R,P$ and also that the corresponding $\lambda$ introduces
variables on the right.
%
As you would expect from a polymorphic function type, the introduction
form is a $\lambda$ that abstracts over both a transformation variable
and an object variable, which are concatenated on the right of the
context, aligning with the condition that the contravariant variable
of $R$ match the covariant variable of the original context.
%
The elimination rule is a combined instantiation and application form,
$\homrappXtoYatZ s t a$ which instantiates the object $\alpha$ to $a$
and passes an element $t : R[a/\alpha]$ as input, yielding an element
$P[a/\alpha]$ as output. Finally, we have the expected $\beta$ and
$\eta$ reasoning principles that we would have for a function type.

The left hom $\homlallXYtoZ \alpha R P$ dually abstracts over context
extension on the left. This time, the variable $\alpha$ occurs
\emph{contravariantly} in the input and output sets, and the covariant
dependence of the function set matches the covariant dependence of the
ouput set, while the contravariant dependence is the covariant
dependence of the input. In our notation, we place the $\forall\alpha$
on the left to be suggestive of the fact that it is contravariant in
$R$ and $P$. Next, the introduction form is again a polymorphic
$\lambda$, but this time we write the object variable $\alpha$ first,
and then the set variable $x$, to match the structure of the
context. The elimination form is similarly an application, but to
agree with the ordered use of variables, this time we place the
argument to the left and the function to the right.

\begin{figure}
  \begin{mathpar}
      \inferrule*[right=Covariant Hom formation]
      {\Gamma \pipe d^+\Xi; \alpha : \cat C \vdash  R \and \Gamma \pipe  d^-\Xi; \alpha : \cat C \vdash P}
      {\Gamma \pipe \Xi \vdash \homrallXYtoZ \alpha R P}

      \inferrule*[right=Covariant Hom introduction]
      {\Gamma \pipe \Phi, x : R, \alpha : \cat C \vdash t : P}
      {\Gamma \pipe \Phi \vdash
        \homrlambdaXatYdotZ {x : R} {\alpha : \cat C} t : \homrallXYtoZ \alpha R P}

      %% \inferrule*[right=Covar Hom elim simple]
      %% {}
      %% {\alpha: \cat C, f:\forall \gamma. P \triangleleft R, \beta : \cat D, x : R, \gamma : \cat E \vdash f\triangleleft (x, \gamma) : P}

      \inferrule*[right=Covariant Hom elimination]
      {\Phi = \Phi_f \jnctx \Phi_a\and
       \Gamma\pipe\Phi_f \vdash s : \homrallXYtoZ {\alpha:\cat C} R P \and
       d^+\Phi_a \vdash a : \cat C\and
       \Phi_a \vdash t : R[a/\alpha]
        }
      {\Gamma\pipe \Phi \vdash \homrappXtoYatZ s t a : P[a/\alpha]}\\

      \homrappXtoYatZ {(\homrlambdaXatYdotZ x \alpha s)} t a = s[t/x,a/\alpha] (\beta)
      %% (\lambda^\triangleleft (x, \gamma). M)\triangleleft (x, \gamma) = M (\beta)

      \inferrule
      {\Gamma\pipe\Phi \vdash s : \homrallXYtoZ \alpha R P}
      {s = \homrlambdaXatYdotZ {x}{\alpha} \homrappXtoYatZ s x \alpha}
      
      %% \inferrule
      %% {\Gamma\pipe\Phi, x:R, \alpha : \cat C \vdash \homrappXtoYatZ s x \alpha = \homrappXtoYatZ t x \alpha : P}
      %% {\Gamma\pipe\Phi \vdash s = t : \homrallXYtoZ \alpha R P}

      \inferrule*[right=Contra Hom formation]
      {\Gamma \pipe \alpha:\cat C; d^-\Xi\vdash R \and \Gamma \pipe \alpha : \cat C; d^+\Xi\vdash P}
      {\Gamma \pipe \Xi \vdash \homlallXYtoZ \alpha R P}

      \inferrule*[right=Contra Hom introduction]
      {\Gamma\pipe\alpha:\cat C, x : R, \Phi \vdash t : P}
      {\Gamma\pipe\Phi \vdash \homllambdaXatYdotZ {x : R} {\alpha : \cat C} t : \homlallXYtoZ \alpha R P}

      \inferrule*[right=Contra Hom elimination]
      {\Phi = \Phi_a \jnctx \Phi_f\and
        \Gamma \pipe d^-\Phi_a \vdash a : \cat C\and
       \Gamma \pipe \Phi_a \vdash s : R[a/\alpha] \and
       \Gamma \pipe \Phi_f \vdash t : \homlallXYtoZ {\alpha:\cat C} R P}
      {\Gamma \pipe \Phi \vdash \homlappXtoYatZ t s a : P[a/\alpha]}\\

      \inferrule*[Right=$\beta$]
                 {}
                 {\homlappXtoYatZ {(\homllambdaXatYdotZ x \alpha t)} s a = t[a/\alpha,s/x]}

      \inferrule*[right=$\eta$]
      {\Gamma \pipe \Phi \vdash t : \homlallXYtoZ \alpha R P}
      {\Gamma \pipe \Phi \vdash t = \homllambdaXatYdotZ x \alpha \homlappXtoYatZ t x \alpha : \homlallXYtoZ \alpha R P}

      %% \inferrule*[right=$\eta$]
      %% {\Gamma \pipe \alpha:\cat C, x:R, \Phi \vdash \homlappXtoYatZ M x \alpha = \homlappXtoYatZ N x \alpha : P}
      %% {\Gamma \pipe \Phi \vdash M = N : \homlallXYtoZ \alpha R P}
  \end{mathpar}
  \caption{Hom of Relations/Profunctors}
  \label{fig:hom}
\end{figure}

In category theory, these profunctors can be constructed using an
\emph{end} and the hom of functor categories\cite{somecats}. Given two
profunctors $P : C^o \times E \to \Set$ and $R : D^o \times E \to
Set$, their hom is given as
\[ \int_{e \in E} R(-,e) \Rightarrow P(e,=) \]

%% The syntax is given in Figure~\ref{fig:cotensor}.
%% %
%% First, the formation rule for covariant cotensor formation requires
%% that the covariant variables for $R,P$ are the same, and the
%% contravariant variables match the boundary of the overall
%% judgment. Note that we use $d^+\Xi,d^-\Xi$ to allow for the case that
%% the boundary is a single variable $\Xi = \alpha:\cat C$.
%% %
%% Next, the substitution rule is straightforward.
%% %
%% Then we have the elimination form, first written in a simple form with
%% no substitution built in.
%% %
%% This says that given a variable of the cotensor judgment, and another
%% variable of judgment $R$, we can apply these to get a term of judgment
%% $P$. As a diagram this is written as follows:

%% \[ % https://tikzcd.yichuanshen.de/#N4Igdg9gJgpgziAXAbVABwnAlgFyxMJZABgBpiBdUkANwEMAbAVxiRAGEQBfU9TXfIRQBGclVqMWbACLdeIDNjwEiAJjHV6zVohABROXyWCiZYeK1TdnHkYEqU685sk793cTCgBzeEVAAZgBOEAC2SGQgOBBIohLabAA6iQEQQYwMAATJ3nShoXQAdJkACtmJOEFYdGDeDDD1ATiZAEqGIMFhsdTRSOrxViBtth0h4YgAzD0xiAAsLgm6Je2d45G9k9QMWGBuUBBMAEb1INQAFjB0UGyQu6dRdFgMNwSsI6t900jzIBdXL3ctjs9gdjqweo9nrpbm8KFwgA
%% \begin{tikzcd}
%% C \arrow[r, "\forall \gamma. P \triangleleft R"] \arrow[d, no head, Rightarrow] & D \arrow[r, "R"] & E \arrow[d, no head, Rightarrow] \\
%% C \arrow[rr, "P"]                                                               &                  & E                               
%% \end{tikzcd}\]

%% The universal property of this judgment is that this is the universal
%% 2-cell of this type, in that any 2-cell


%% \[ % https://tikzcd.yichuanshen.de/#N4Igdg9gJgpgziAXAbVABwnAlgFyxMJZABgBpiBdUkANwEMAbAVxiRAGEByEAX1PUy58hFACZyVWoxZsAIr34gM2PASIBmCdXrNWiEAFEFAlcKJkAjJJ0z97Y0sGqRyTVe3S9hh8qFqUFlpSumy8kjBQAObwRKAAZgBOEAC2SIEgOBBI4sG2IABKDokpSJoZWYgALB4h+gAKRUmpiGTlpdQMWGBecBCdUCDUABYwdAOIYEwMDNQ4dFgMbJDdjSWIOZlI1SAjY0sErB1dXlAQTABGDIcZ84v6y6x88U1IrZuIAKxPIMXNH7MVdKdFb6KB0OAjAbDUbjcAHMI8IA
%% \begin{tikzcd}
%% C' \arrow[d] \arrow[r] & {} \arrow[r, no head, dashed] & D \arrow[r, "R"] & E \arrow[d, no head, Rightarrow] \\
%% C \arrow[rrr, "P"]     &                               &                  & E                               
%% \end{tikzcd}\]

%% Exhibits a unique factorization:

%% \[ % https://tikzcd.yichuanshen.de/#N4Igdg9gJgpgziAXAbVABwnAlgFyxMJZABgBoBGAXVJADcBDAGwFcYkQBhEAX1PU1z5CKAEwVqdJq3YARHnxAZseAkQDM4mgxZtEIAKLz+yoUTIiJ26Xq69jg1Sg0WtU3QaOKBK4SVLFLN3YOAHJPJQdfcn9AnXZw71NRGNc4vTk7LxNHZA0A1OsPbgkYKABzeCJQADMAJwgAWyRokBwIJDFJNJAAJU86xqQNVvbEABYC9wAFfvqmxDIRoZpGLDB3KAhmACNGNhoACxh6KHZIdZAaHHosRjOCNkyB+c62pAmQI5P7i5W1ja2u32rRudz050eCmeSAArFdRgA2J5zJAI+FIADsfwueig9DgR1Oh2Op3BD1mg0QWKWiBaqxxIE2Oz2l0+JJ+wOutw5FPmcJpxGRlMWb1pk3YAB0JdUILUmIwAARSsr0BoNegAOgVUyVEpwtSw9DAZT2e2qOAVfSF82pooAHOK9FaoSjEA6aZ16QDmcCvqTwOSrqCecVuEA
%% \begin{tikzcd}
%% C' \arrow[r] \arrow[d]                                                           & {} \arrow[r, no head, dashed] & D \arrow[d, no head, Rightarrow] \arrow[r, "R"] & E \arrow[d, no head, Rightarrow] \\
%% C \arrow[d, no head, Rightarrow] \arrow[rr, "\forall \gamma. P \triangleleft R"] &                               & D \arrow[r, "R"]                                & E \arrow[d, no head, Rightarrow] \\
%% C \arrow[rrr, "P"]                                                               &                               &                                                 & E                               
%% \end{tikzcd}\]

%% We give elimination rules and introduction rules in a style similar to
%% $\lambda$-calculus, with corresponding $\beta\eta$ rules.

These hom types are useful both for category-theoretic applications
(Kan extensions, the Yoneda lemma), but also for purely syntactic
reasons.  The universal property of these function types allows us, by
iterated hom to express any sequent $\Phi \vdash R$ as an equivalent
one with an ``empty'' context, i.e., a single object variable $d^\pm
\Phi$. Then using the end type defined in the previous section, we can
internalize that judgment as a type in our ambient dependent type
theory.

As an example, any sequent $\Gamma \pipe \alpha, x: R, \beta, y : S, \gamma \vdash
P$ is equivalent to either of
\[ \Gamma \vdash \pendallXdotY\alpha \homrallXYtoZ \beta R \homrallXYtoZ \gamma S P 
\qquad\qquad\qquad \Gamma \vdash \pendallXdotY \gamma \homlallXYtoZ \beta S {\homlallXYtoZ \alpha R P}\]
This property is useful in giving a simple syntax to positive types.

\subsection{Unit and Tensor of Profunctors}

While the hom sets of the previous section work quite smoothly since
they are negative connectives in a natural deduction system, the positive
connectives, the unit and tensor, are trickier to define.

First, the multiplicative unit is the set $\punitinXfromYtoZ {\cat C}
{a_1} {a_2}$, semantically, the hom set of the category $\cat C$. The
rules for this type act like an identity type in a directed type
theory. The introduction rule defines an identity morphism of an
object in any category. The elimination rule is then the appropriate
analogue of the $J$ rule of the identity type. Given any profunctor
$Q$ from $\cat C$ to itself, we can construct an element $\harrapp Q
{\alpha_1}{\alpha_2}$ for any objects in $\cat C$ if we specify a
natural element $\harrapp Q \alpha \alpha$ for any $\alpha$ in $\cat
C$. The reason this is sufficient, semantically, is that all
profunctors must be functorial in their two indices. In fact, as we
will see in the section on synthetic category theory, this induction
principle is how we can show internally that profunctors are
functorial in their indices. We choose to express this elimination
form using the right hom type as its syntax more closely matches the
usual syntax of function types, but this choice is arbitrary, and we
would get an equivalent calculus had we used the left hom, or added
both.

Next, we continue with the tensor product $\tensorexistsXwithYandZ
\beta P Q$. Adjoint to the right/left hom sets, this provides a form
of \emph{existentially quantified pair}. The formation rule states the
set is well-formed when $\beta$ is the covariant variable of $P$ and
the contravariant variable of $Q$, and, its contravariant dependence
is the same as $P$, while its covariant dependence is the same as $Q$.
%
The introduction rule is then a tuple $\tensorintroatXwithYandZ b s t$
with an instantiation $b$ of the quantified variable $\beta$. Note
that the use of $d^+\Psi_s$ in the rule could be replaced by
$d^-\Psi_t$ since they must be equal by the side condition $\Phi =
\jnctx \Psi_s \Psi_t$.
%
The elimination rule says to construct a transformation taking a pair
as input it is sufficient to construct one that takes in a $P$ and $Q$
for any $\beta$, directly corresponding to the adjoint relationship
between the tensor and the right hom.
%
Finally, we have the expected $\beta\eta$ equations.

\begin{figure}
  \begin{mathpar}
    \inferrule*[right=Unit-Formation]
    {\Gamma \vdash \cat C \isaCat\and
     \Gamma \pipe d^-\Xi \vdash a_1 : \cat C\and \Gamma \pipe d^+ \Xi \vdash a_2 : \cat C}
    {\Gamma \pipe \Xi \vdash \punitinXfromYtoZ {\cat C} {a_1} {a_2}}

    %% (A \to A')[\xi] = A[d^-\xi] \to A'[d^+\xi]

    \inferrule*[right=Unit-Introduction]
    {\Gamma\pipe\beta: \cat D \vdash a : \cat C}
    {\Gamma \pipe\cdot \vdash \punitrefl a : \punitinXfromYtoZ {\cat C} a a}

    \inferrule*[right=Unit Elim]{
      \Gamma\vdash Q : \harr {\cat C} {\cat C}\and
      \Gamma\vdash t : \pendallXdotY {\alpha:\cat C} {\harrapp Q \alpha \alpha}}
    {\Gamma \vdash \punitelimtoYwithkontZ Q N :
      \pendallXdotY {\alpha_1} {\homrallXYtoZ {\alpha_2} {\punitinXfromYtoZ {\cat C} {\alpha_1}{\alpha_2}} {\harrapp Q {\alpha_1}{\alpha_2}}}
      }

    %% J_\to(\alpha_1.\alpha_2. Q, \alpha. N, A_1, M, A_2)[\phi] = J_\to(\alpha_1.\alpha_2. Q, \alpha. N, A_1[d^-\phi], M[\phi], A_2[d^+\phi])

    %% \inferrule*[right=Unit$\beta$]{\beta:\cat D \vdash A : \cat C\and
    %%   \alpha:\cat C \vdash N : Q[\alpha/\alpha_1,\alpha/\alpha_2]
    %% }{\beta : \cat D \vdash J_\to(\alpha_1.\alpha_2. Q, \alpha. N, A, \id(A), A) = N[A/\alpha] : Q[A/\alpha_1,A/\alpha_2]}
    %%

    \appunary {\punitelimtoYwithkontZ Q N} a {\punitrefl a} a = \pendappXtoY N a

    \inferrule*[right=Unit$\eta$]
    {\Gamma \vdash M : \pendallXdotY {\alpha_1} \homrallXYtoZ {\alpha_2} {\punitinXfromYtoZ {\cat C} {\alpha_1} {\alpha_2}} {\harrapp Q {\alpha_1} {\alpha_2}}}
    {M = {\punitelimtoYwithkontZ Q {\appunary M \alpha {\punitrefl \alpha} \alpha}}}
    %% {}
    %% {\alpha_1 : \cat C, f : \alpha_1 \to_{\cat C} \alpha_2, \alpha_2:\cat C \vdash M : \harrapp Q {\alpha_1} {\alpha_2}}
    %% {\alpha_1 : \cat C, f : \alpha_1 \to_{\cat C} \alpha_2, \alpha_2:\cat C \vdash M =
    %%   \homrappXtoYatZ {\pendappXtoY {\punitelimtoYwithkontZ Q {\pendlambdaXdotY \alpha M[\alpha/\alpha_1,\pendappXtoY {\punitrefl C} \alpha/f,\alpha/\alpha_2]}} {\alpha_1}} f {\alpha_2}
    %% : \harrapp Q {\alpha_1}{\alpha_2}}

    \inferrule*[right=Tensor formation]
    {\Gamma \pipe d^-\Xi; \beta:D\vdash P \isaSet\and
     \Gamma \pipe \beta:D;d^+\Xi \vdash Q \isaSet}
    {\Gamma \pipe \Xi \vdash \tensorexistsXwithYandZ {\beta:D} P Q}

  %% \inferrule*[right=Tensor intro function]
  %% {}
  %% {\alpha : \cat C, q : Q, \beta : \cat D, r : R, \gamma : \cat E \vdash (q, \beta,r) : \exists \beta : \cat D. Q \odot R}

  %% \inferrule*[right=Tensor introduction]
  %% {\Gamma \vdash P : \harr C D \and
  %%   \Gamma \vdash Q : \harr D E}
  %% {\Gamma\pipe\cdot\vdash \tensorintroofXandY P Q : \pendallXdotY {\alpha} \homrallXYtoZ \beta {\harrapp P \alpha \beta} {\homrallXYtoZ \gamma {\harrapp Q \beta \gamma} {(\tensorexistsXwithYandZ \beta {\harrapp P \alpha \beta}{\harrapp Q \beta \gamma})}}}

  \inferrule*[right=Tensor introduction]
  {\Phi = \Psi_s \jnctx \Psi_t \and
   \Gamma \pipe d^+\Psi_s \vdash b : \cat D\and
   \Gamma \pipe \Psi_s \vdash s : P[b/\beta] \and
   \Gamma \pipe \Psi_t \vdash t : Q[b/\beta]}
  {\Gamma\pipe\Phi\vdash \tensorintroatXwithYandZ b s t : \tensorexistsXwithYandZ {\beta : \cat D} P Q}

  %% \tensorintroatXwithYandZ B M N [\phi,B'/\beta',\psi] = \tensorintroatXwithYandZ {B[B'\beta]} {M[\phi]} {N[\psi]}

  %% \inferrule*[right=Tensor simpl]
  %% {\Gamma\pipe\cdot \vdash R : \harr {C} {E}\and
  %%  \Gamma\pipe\cdot \vdash N : \pendallXdotY {\alpha:C} {\homrallXYtoZ {\beta:D} P {\homrallXYtoZ {\gamma:E} Q {\harrapp R \alpha\gamma}}}}
  %% {\Gamma\pipe\alpha:C,p:\tensorexistsXwithYandZ {\beta:D} P Q,\gamma:E \vdash\tensorelimVatWtoXtoYwithkontZ p \alpha \gamma R N : \harrapp R A G}

  \inferrule*[right=Tensor-elim]
  {\Gamma \pipe \vdash P : \harr {\cat C} {\cat D}\and\Gamma \vdash Q : \harr {\cat D}{\cat E}\and
   \Gamma \vdash R : \harr {\cat C}{\cat E}\and
   \Gamma \vdash N : \hombinary \alpha {\harrapp P \alpha \beta} \beta {\harrapp Q \beta \gamma} \gamma {\harrapp R \alpha \gamma}}
  {\Gamma \vdash \tensorelimfromWXtoYwithkontZ P Q R N : \homunary \alpha {(\tensorexistsXwithYandZ \beta {\harrapp P \alpha \beta} {\harrapp Q \beta \gamma})} {\gamma} {\harrapp R \alpha \gamma}}
  
  %% \inferrule*[right=Tensor elim]
  %% {\Gamma\pipe\cdot \vdash R : \harr {C} {E}\and
  %%  \Gamma\pipe\Phi \vdash M : \tensorexistsXwithYandZ {\beta:D} {P[A/\alpha;]} {Q[;G/\gamma]}\and
  %%  \Gamma\pipe\cdot \vdash N : \pendallXdotY {\alpha:C} {\homrallXYtoZ {\beta:D} P {\homrallXYtoZ {\gamma:E} Q {\harrapp R \alpha\gamma}}}}
  %% {\Gamma\pipe\Phi \vdash\tensorelimVatWtoXtoYwithkontZ M A G R N : \harrapp R A G}

  %% J_\odot(\alpha. \gamma. S, \alpha.q.\beta.r.\gamma.N, A, M, G)[\phi] =
  %% J_\odot(\alpha. \gamma. S, \alpha.q.\beta.r.\gamma.N, A[d^-\phi], M[\phi], G[d^+\phi])

  %% \inferrule*[right=Tensor elim Sequent]
  %% {\Phi,\alpha : \cat C, q : Q, \beta : \cat D, r: R, \gamma : \cat E,\Psi \vdash N : S}
  %% {\Phi, \alpha : \cat C, x : \exists \beta. Q \odot R, \gamma : \cat E,\Psi \vdash J_\odot(x;q.\beta.r. N) : S}

  %% \inferrule*[right=Tensor elim Subst]
  %% {\Phi,\alpha : \cat C, q : Q, \beta : \cat D, r: R, \gamma : \cat E,\Phi' \vdash N : S
  %% \and
  %% \phi : \Psi \to \Phi,\alpha : \cat C, q : Q, \beta : \cat D, r: R, \gamma : \cat E,\Phi'}
  %% {\Psi  \vdash J_\odot(x;q.\beta.r. N)[\phi] : S}


  
  \inferrule*[right=Tensor $\beta$]
  {}
  {\appunary {\tensorelimfromWXtoYwithkontZ P Q R N} a {\tensorintroatXwithYandZ b s t} c = \appbinary N a s b t c}
  %% {\homrappXtoYatZ {\pendappXtoY{\tensorelimfromWXtoYwithkontZ P Q R N} A} {(\homrappXtoYatZ{\homrappXtoYatZ{\pendappXtoY {\tensorintroofXandY P Q} A} f B} g G)} G =
  %%   \homrappXtoYatZ {\homrappXtoYatZ {\pendappXtoY N A} f B} g G}
  
  %% \inferrule*[right=Tensor $\beta$]
  %% {}{J_\odot(\alpha.\gamma.S, \alpha.q.\beta.r.\gamma.N,A,(M,B,N),G) = N[A/\alpha,M/q,B/\beta,N/r,G/\gamma]}

  \inferrule*[right=Tensor$\eta$]
  {\Gamma \vdash N : \homunary \alpha {(\tensorexistsXwithYandZ \beta {\harrapp P \alpha \beta}{\harrapp Q \beta \gamma})} \gamma {\harrapp R \alpha \gamma}}
  {N = \tensorelimfromWXtoYwithkontZ P Q R {\lambdabinary \alpha x \beta y \gamma \appunary N \alpha {\tensorintroatXwithYandZ \beta x y} \gamma}}

  %% \inferrule*[right=Tensor$\eta$]
  %% {\alpha : \cat C, x : \tensorexistsXwithYandZ \beta P Q, \gamma : \cat E \vdash M : R}
  %% {\alpha : \cat C, x : \exists \beta. Q \odot R, \gamma : \cat E \vdash M =
  %%   \homrappXtoYatZ {\pendappXtoY{\tensorelimfromWXtoYwithkontZ P Q R
  %%       {\pendlambdaXdotY \alpha {\homrlambdaXatYdotZ p \beta {\homrlambdaXatYdotZ q \gamma {M[\homrappXtoYatZ {\homrappXtoYatZ {\pendappXtoY {\tensorintroofXandY P Q} \alpha} p \beta} q \gamma/x]}}}}
  %%     } \alpha} x \gamma : S}
  \end{mathpar}  
  \caption{Unit, Tensor of Profunctors}
\end{figure}

%% Next, we can internalize the $,\alpha,$ as a positive connective as a
%% kind of ``existentially quantified tensor product'', abstracting the
%% composition of profunctors.

%% The ordinary formula for this is that for profunctors $R : C^o \times
%% D \to \Set$ and $S : D^o \times E \to \Set$, the composite is given by
%% a coend:

%% \[ \int^{d \in D} R(-,d) \otimes S(d,=) \]

%% We will instead use slightly more logical notation, writing the coend as
%% an existential quantifier, justified by the analogy between the two
%% and the structure of the introduction/elimination forms in our syntax
%% in particular.
%% \begin{figure}
%% \begin{mathpar}
%% \end{mathpar}
%% \caption{Tensor Primitive Rules}
%% \label{fig:tensor}
%% \end{figure}

Semantically, the tensor product is the composition of profunctors,
which is typically written as a coend
\[ \int_{\beta} P(-,\beta) \times Q(\beta,=) \]

\subsection{Dependent Types}

Next, we cover the basic dependent type theory structure: $\Pi,\Sigma$
and identity types.

\begin{figure}
  \begin{mathpar}
  \inferrule
  {\Gamma \pipe\cdot \vdash S \isaSet\and \Gamma \pipe\cdot \vdash T \isaSet}
  {\Gamma\pipe\cdot \vdash \prod_{x:S} T \isaSet}

  \inferrule
  {\Gamma \pipe \cdot \vdash M : \prod_{x:S} T\and
    \Gamma \pipe \cdot \vdash N : S}
  {\Gamma \pipe \cdot \vdash M N : T[N/x]}

  \inferrule
  {\Gamma,x:S \pipe \cdot \vdash M : T}
  {\Gamma \pipe \cdot \vdash \lambda x:S. M : \prod_{x:S} T}

  \inferrule
  {\Gamma \pipe\cdot \vdash S \isaSet\and \Gamma \pipe\cdot \vdash T \isaSet}
  {\Gamma\pipe\cdot \vdash \sum_{x:S} T \isaSet}
  
  \inferrule
  {~}
  {\Gamma\pipe\cdot \vdash \top \isaSet}

  \inferrule
  {\Gamma \pipe \cdot \vdash M : S\and \Gamma \pipe \cdot \vdash N : S}
  {\Gamma \pipe \cdot \vdash \textrm{Id}_S M N \isaSet}

  \inferrule
  {\Gamma \pipe \cdot \vdash M : S}
  {\Gamma \pipe \cdot \vdash \textrm{refl}(M) : \textrm{Id}_S M M}

  \inferrule
  {\Gamma,x:S,y:S,p:\textrm{Id}_S x y \pipe \cdot \vdash C \isaSet\and
   \Gamma \pipe \cdot \vdash M_k : \prod_{x:S} C[x/y,\textrm{refl}(x)/p]}
  {\Gamma \pipe \cdot \vdash \textrm{J}(x,y,p.C,M) : \prod_{x:S,y:S,p:\textrm{Id}_S x y} C}
  \end{mathpar}
  \caption{Ordinary Dependent Type Structure}
\end{figure}

\subsection{Internalizing Virtual Equipment Structure}

\begin{figure}
  \begin{mathpar}
    \inferrule
    {\Gamma \vdash \cat C : \Sort \and\Gamma \vdash \cat D : \Sort}
    {\Gamma\pipe\cdot \vdash \varr{\cat C}{\cat D} \isaSet}

    \inferrule
    {\Gamma \pipe \alpha:\cat C \vdash A : \cat D}
    {\Gamma \vdash \lambda \alpha:\cat C. A : \varr{\cat C}{\cat D}}

    \inferrule
    {\Gamma \pipe \alpha:\cat C \vdash A : \cat D\and
      \Gamma\pipe\cdot \vdash M : \varr{\cat D}{\cat E}}
    {\Gamma\pipe \alpha:\cat C \vdash M A : \cat E}

    (\lambda\alpha:C. A) B = A[B/\alpha]

    \inferrule*[right=VArr $\eta$]
    {\Gamma \pipe\cdot \vdash M : \varr C D}
    {\Gamma \pipe\cdot \vdash M = (\lambda \alpha:C. M \alpha) : \varr C D}
    \\

    \inferrule
    {\Gamma \vdash \cat C : \Sort \and\Gamma \vdash \cat D : \Sort}
    {\Gamma\pipe\cdot \vdash \harr{\cat C}{\cat D} \isaSet}

    \inferrule
    {\Gamma \pipe \alpha:\cat C;\beta:\cat D \vdash R \,\isaSet}
    {\Gamma \vdash \lambda \alpha:\cat C;\beta:\cat D. R : \harr{\cat C}{\cat D}}

    \inferrule
    {\Gamma \vdash M : \harr{\cat C}{\cat D}\and
     \Gamma \pipe d^- \Xi \vdash A : \cat C\and
     \Gamma \pipe d^+ \Xi \vdash B : \cat C}
    {\Gamma \pipe \Xi \vdash M A; B \isaSet}

    (\lambda \alpha:C;\beta:D. R)A;B = R[A/\alpha;B/\beta]

    \inferrule*[right=HArr $\eta$]
    {\Gamma \pipe \cdot \vdash M : \harr C D}
    {\Gamma \pipe \cdot \vdash M = (\lambda \alpha;\beta. M \alpha;\beta) : \harr C D}\\

  \end{mathpar}
  \caption{Internalizing VE Structure}
\end{figure}

%% \subsection{Equalizers}

%% To encode the concept of internal hom of profunctors, kan extensions
%% and (co-)limits, we need to define the \emph{equalizer} of horizontal
%% arrows. Specifically, the universal property that we need is an
%% equalizer in the category whose objects are horizontal arrows (for
%% fixed endpoints) and whose arrows are the 2-cells.

%% \begin{figure}
%%   \begin{mathpar}
%%     \inferrule*[right=Equalizer-formation]
%%     {\Gamma \pipe \cdot \vdash P : \harr\, C\,D\and
%%      \Gamma \pipe \cdot \vdash Q : \harr\, C \, D\\\\
%%      \Gamma \pipe \cdot \vdash \phi : \homunary \alpha {\harrapp P \alpha \beta} \beta {\harrapp Q \alpha \beta}\and
%%      \Gamma \pipe \cdot \vdash \psi : \homunary \alpha {\harrapp P \alpha \beta} \beta {\harrapp Q \alpha \beta}\and}
%%     {\Gamma \pipe \cdot \vdash \textrm{Equalizer}(C,D,P,\phi,\psi) : \harr\, C\, D}\and

%%     \inferrule
%%     {\Gamma \pipe \Phi \vdash M : \harrapp{\textrm{Equalizer}(C,D,P,\phi,\psi)} A B}
%%     {\Gamma \pipe \Phi \vdash \textrm{forget}(M) : \harrapp P A B}

%%     \inferrule
%%     {\Gamma \pipe \Phi \vdash M : \harrapp P A B \and \appunary \phi A M B = \appunary \psi A M B}
%%     {\Gamma \pipe \Phi \vdash \textrm{equalizing}_{\phi,\psi}(M) : \harrapp{\textrm{Equalizer}(C,D,P,\phi,\psi)} A B}

%%     TODO: obvious equations
%%   \end{mathpar}
%%   \caption{Equalizers (Logical Framework style)}
%% \end{figure}


%% \begin{figure}
%%   \begin{mathpar}
%%     \inferrule*[right=Equalizer-formation]
%%     {\Gamma \pipe \Xi \vdash P \isaSet\and\Gamma \pipe \Xi \vdash Q \isaSet\and
%%       \Gamma\pipe d^-\Xi, p : P, d^+ \Xi \vdash \phi : Q\and
%%       \Gamma\pipe d^-\Xi, p : P, d^+ \Xi \vdash \psi : Q }
%%     {\Gamma \pipe \Xi \vdash \equalizeVofWbyXeqYatZ p P \phi \psi Q : \harr\, C\, D}\and

%%     \inferrule*[right=Equalizer-formation]
%%     {\Gamma \pipe \alpha:C;\beta:D \vdash P \isaSet\and\Gamma \pipe \alpha:C;\beta:D \vdash Q \isaSet\and
%%       \Gamma \pipe d^-\Xi \vdash A : C\and
%%       \Gamma \pipe d^+\Xi \vdash B : D\\\\
%%       \Gamma\pipe \cdot \vdash \phi : \homunary \alpha P \beta Q\and
%%       \Gamma\pipe \cdot \vdash \psi : \homunary \alpha P \beta Q
%%     }
%%     {\Gamma \pipe \Xi \vdash \equalizeVofWbyXeqYatZ p {P[A/\alpha;B/\beta]} {\appunary \phi A p B} {\appunary \psi A p B} {Q[A/\alpha;B/\beta]} : \harr\, C\, D}\and

%%     \inferrule
%%     {\Gamma \pipe \Phi \vdash M : \equalizeVofWbyXeqYatZ p P \phi \psi Q}
%%     {\Gamma \pipe \Phi \vdash \pi_{eq}\,M : P}

%%     \inferrule
%%     {\Gamma \pipe \Phi \vdash M : \harrapp P A B \and \appunary \phi A M B = \appunary \psi A M B}
%%     {\Gamma \pipe \Phi \vdash \textrm{equalizing}_{\phi,\psi}(M) : \harrapp{\textrm{Equalizer}(C,D,P,\phi,\psi)} A B}

%%     TODO: equations
%%   \end{mathpar}
%%   \caption{Equalizers (Natural Deduction Style)}
%% \end{figure}

%% \subsection{Products}

%% We add in parallel two different notions of product: products of sorts
%% and products of spans, which have the universal property in their
%% respective categories.

%% \begin{figure}
%%   \begin{mathpar}
%%     \inferrule{}{\Gamma \vdash 1 : \Sort}\and
%%     \inferrule*{}{\Gamma\pipe\alpha:C \vdash () : 1}\and
%%     \inferrule*{\Gamma\pipe\alpha:C \vdash A : 1}{A = ()}
%%     \inferrule{}{\Gamma\pipe \Xi \vdash 1 \isaSet}\and
%%     \inferrule{}{\Gamma\pipe \Phi \vdash () :  1}\and
%%     \inferrule{\Gamma\pipe \Phi \vdash M :  1}{M = ()}\\\\

%%     \inferrule{\Gamma \vdash C : \Sort \and \Gamma \vdash D : \Sort}
%%               {\Gamma \vdash C \times D : \Sort}\and

%%     \inferrule{\forall i \in \{1,2\}.~ \Gamma \pipe \alpha:C \vdash A_i : D_i}{\Gamma \pipe \alpha:C \vdash (A_1,A_2) : D_1 \times D_2}

%%     \inferrule{\Gamma \pipe \alpha:C \vdash A : D_1 \times D_2}{\Gamma \pipe \alpha:C \vdash \pi_i A : D_i}

%%     \pi_i(A_1,A_2) = A_i\\
%%     \inferrule{\Gamma\pipe \alpha:C \vdash A :  D_1\times D_2}{A = (\pi_1 A,\pi_2 A)}\\\\

%%     \inferrule{\Gamma\pipe\Xi \vdash R \isaSet \and \Gamma\pipe\Xi \vdash S\isaSet}
%%               {\Gamma\pipe\Xi \vdash R \times S\isaSet}\and

%%     \inferrule{\forall i \in \{1,2\}.~ \Gamma \pipe\Phi \vdash M_i : R_i}{\Gamma \pipe\Phi \vdash (M_1,M_2) : R_1 \times R_2}

%%     \inferrule{\Gamma \pipe\Phi\vdash M : R_1 \times R_2}{\Gamma \pipe\Phi \vdash \pi_i M : R_i}

%%     \pi_i(M_1,M_2) = M_i\\
%%     \inferrule{\Gamma\pipe\Phi \vdash M :  R_1\times R_2}{M = (\pi_1 M,\pi_2 M)}
%%   \end{mathpar}
%%   \caption{Products}
%%   \label{fig:products}
%% \end{figure}

%% One additional connective we might desire is a ``parallel'' product of
%% sets $R \times' S : \harr {C_1 \times C_2} {D_1 \times D_2}$ where $R
%% : \harr {C_1} {D_1}$ and $R : \harr {C_2} {D_2}$. We don't need to
%% define this as a primitive though because it is expressible in terms
%% of the two kinds of products we have above:

%% \[
%% \inferrule
%% {\Gamma \pipe \alpha : C_i; \beta : D_i \vdash R_i \isaSet}
%% {\Gamma \pipe \alpha : C_1\times C_2; \beta : D_1 \times D_2
%%   R_1[\pi_1\alpha/\alpha;\pi_1\beta/\beta] \times R_2[\pi_2\alpha/\alpha;\pi_2\beta/\beta]\isaSet}
%% \]

%% The empty products give us access now to several traditional notions
%% of category theory that we couldn't before define, such as objects and
%% presheaves of either variance:

%% $$\textrm{Object}\,C = \varr 1 C$$
%% $$\textrm{Presheaf}\,C = \harr 1 C$$
%% $$\textrm{coPresheaf}\,C = \harr C 1$$

%% However, we find that in practice we don't have much use for these
%% concepts as most theorems about them easily generalize to more general
%% theorems about arbitrary functors or profunctors.

%% %% To define internal ends and coends, we need to go beyond the usual
%% %% simple types and add equalizers (a form of subsets) and co-equalizers
%% %% (a form of quotients) of judgments. Here's a sketch of the syntax for
%% %% each. These make type checking very difficult as they result in
%% %% obligations for proving equality of certain terms, which may or may
%% %% not be decidable.

%% %% \begin{mathpar}
%% %%   \inferrule*[right=Equalizer Formation]
%% %%   {\alpha;\beta\,|\, R \and
%% %%   x : R \vdash t : S x:R \vdash u : S}
%% %%   {\alpha;\beta\,|\,\{ x : R \,|\, t = u \}}

%% %%   \inferrule*[right=Equalizer Intro]
%% %%   {\Phi \vdash r : R \and t[r/x] = u[r/x]}
%% %%   {\Phi \vdash \textrm{assert}(x.t = u)(r) : \{ x : R \,|\, t = u \}}

%% %%   \inferrule*[right=Equalizer Elim]
%% %%   {\Phi \vdash r : \{ x : R \,|\, t = u \}}
%% %%   {\Phi \vdash \textrm{forget}(r) : R}

%% %%   t[\textrm{forget}(r)/x] = u[\textrm{forget(r)}/x]

%% %%   \textrm{assert}(x.t = u)(\textrm{forget}(r)) = r
%% %%   \textrm{forget}(\textrm{assert}(x.t = u)(r)) = r
%% %% \end{mathpar}

%% %% \begin{mathpar}
%% %%   \inferrule*[right=Co-equalizer Formation]
%% %%   {\alpha;\beta\,|\, R \and
%% %%   x : S \vdash t : R x:S \vdash u : R}
%% %%   {\alpha;\beta\,|\,R/(\forall x:S. t = u)}

%% %%   \inferrule*[right=Coequalizer Intro]
%% %%   {\Phi \vdash r : R}
%% %%   {\Phi \vdash \textrm{equate}(x:S.t=u)(r) : R/(\forall x.t=u)}

%% %%   \inferrule*[right=Coequalizer Elim]
%% %%   {\Psi \vdash r : R/(x.t = u) \and
%% %%     \Phi,y:R,\Phi' \vdash s : S \and
%% %%     s[t/y] = s[u/y]}
%% %%   {\Phi,\Psi,\Phi' \vdash \textrm{lett} \textrm{equate}(y) = r.~s : S}

%% %%   \textrm{equate}(x:S.t = u)(t[\gamma]) = \textrm{equate}(x:S.t = u)(u[\gamma])

%% %%   \cdots
%% %% \end{mathpar}

%% %% \subsection{Positive Multiplicatives}

%% %% \subsection{Negative Multiplicatives}

%% %% \subsection{Additives}

\section{Synthetic Category Theory}

Next, we look at synthetic category theory constructions in SCT.

\subsection{Profunctor Isomorphisms}

We start our study of synthetic category theory constructions with
some isomorphisms that show the interaction between the 5
multiplicative connectives: right/left hom, end, unit and tensor.

First, we fix some terminology
\begin{lemma}
  Let $P, Q : \harr {\cat C} {\cat D}$.
  A homomorphism from $P$ to $Q$ is a term of type
  \[ \homunary {\alpha} {\harrapp P \alpha \beta} {\beta} {\harrapp Q \alpha \beta} \]
  Given homomorphisms $M : P \multimap Q$, $N : Q \multimap R$ we can define their composite as
  \[ (N \circ M) = \lambdaunary {\alpha} {x:\harrapp P \alpha \beta} \beta {\appunary \alpha N \beta {\appunary \alpha M \beta x}} \]
  This notion of composition is associative and unital using the
  $\beta\eta$ equations for the end/right hom connectives, with unit
  $\lambdaunary {\alpha} x \beta x$.

  An isomorphism between $P, Q : \harr {\cat C}{\cat D}$ is then a
  homomorphism $P \multimap Q$ with an inverse $Q \multimap P$.
\end{lemma}

Our first synthetic theorem is that we have that the unit, i.e., the
hom set, is the unit for tensor and hom.
%
The theorem for hom is the synthetic \emph{Yoneda lemma} and the
theorem for tensor is the synthetic \emph{Co-Yoneda lemma}.
\begin{lemma}[{Yoneda and Co-Yoneda}]
  \label{lem:yo-and-co-yo}
  Let $P : \harr {\cat C} {\cat D}$.

  For any $\gamma : \cat E \vdash b : \cat D$, the following are naturally isomorphic to $\harrapp P \alpha b$:
  \begin{enumerate}
  \item $\homrallXYtoZ {\beta} {(\punitinXfromYtoZ {D} {\beta} {b})} {\harrapp P {\alpha} {\beta}}$
  \item $\tensorexistsXwithYandZ {\beta} {\harrapp P \alpha {\beta}} {(\punitinXfromYtoZ {\cat D} {\beta} {b})}$
  \end{enumerate}

  For any $\gamma : \cat E \vdash a : \cat C$, the following are naturally isomorphic to $\harrapp P a \beta$:
  \begin{enumerate}
  \item $\homlallXYtoZ {{\alpha}} {(\punitinXfromYtoZ {C} {\alpha} {a})} {\harrapp P {{\alpha}} {\beta}}$
  \item $\tensorexistsXwithYandZ {\alpha} {(\punitinXfromYtoZ {C} {a} {\alpha})} {\harrapp P {\alpha}{\beta}}$
  \end{enumerate}
\end{lemma}
\begin{proof}
  TODO: what's better, build the substitution in or prove the
  substitution can be built in afterwards? The one building in the
  substitution obscures the connection to the normal tensor/hom/unit
  relationship, but the one with the built in subst is obviously more
  general.
\end{proof}

Next, we have various associativities between the homs, tensor and end
constructions. Semantically, these are related to so-called
\emph{Fubini} theorems about interchange of (co)-ends.
\begin{lemma}[{Fubini}]
  \label{lem:fubini}
  The following isomorphisms hold for any $P : \harr {\cat C}{\cat D},
  P' : \harr {\cat C}{\cat D} Q :\harr {\cat D}{\cat E}, R : \harr
  {\cat E} {\cat F}, S : \harr {\cat C} {\cat E}$. where $\alpha:\cat
  C, \beta:\cat D, \gamma : \cat E, \delta: \cat F$
  \begin{enumerate}
  \item $\tensorexistsXwithYandZ {\beta} {\harrapp P \alpha \beta} {(\tensorexistsXwithYandZ \gamma {\harrapp Q \beta \gamma} {\harrapp R \gamma \delta})} \cong \tensorexistsXwithYandZ {\gamma} {(\tensorexistsXwithYandZ \beta {\harrapp P \alpha \beta} {\harrapp Q \beta \gamma})} {\harrapp R \gamma \delta}$
  \item $\homrallXYtoZ {\gamma} {(\tensorexistsXwithYandZ \beta {\harrapp P \alpha \beta} {\harrapp Q \beta \gamma})} {\harrapp S \alpha \gamma} \cong \homrallXYtoZ \beta {\harrapp P \alpha \beta} {\homrallXYtoZ \gamma {\harrapp Q \beta \gamma} {\harrapp S \alpha \gamma}}$
  \item $\homlallXYtoZ {\alpha} {(\tensorexistsXwithYandZ \beta {\harrapp P \alpha \beta} {\harrapp Q \beta \gamma})} {\harrapp S \alpha \gamma} \cong \homlallXYtoZ \beta {\harrapp Q \beta \gamma} {\homlallXYtoZ \alpha {\harrapp P \alpha \beta} {\harrapp S \alpha \gamma}}$
  \item $\homrallXYtoZ \gamma {\harrapp Q \beta \gamma} {(\homlallXYtoZ \alpha {\harrapp P \alpha \beta} {\harrapp S \alpha \gamma})} \cong \homlallXYtoZ \alpha {\harrapp P \alpha \beta} {(\homrallXYtoZ \gamma {\harrapp Q \beta \gamma} {\harrapp S \alpha \gamma})}$
  \item $\pendallXdotY \alpha \homrallXYtoZ \beta P {P'} \cong \pendallXdotY \beta \homlallXYtoZ \alpha P {P'}$
  \end{enumerate}  
\end{lemma}
\begin{proof}
  TODO
\end{proof}
\subsection{Composition, Functoriality, Profunctoriality, Naturality}

Hom sets don't provide primitive syntax for composition of morphisms
in a category. As our first synthetic construction, we can
\emph{define} composition using the elimination form for the unit.
%
\begin{definition}[Composition of Morphisms]
  We define composition of morphisms as the following term of type $\Pi_{\cat C:\Cat}\hombinary {\alpha_1} {\punitinXfromYtoZ {\cat C}{\alpha_1} {\alpha_2}} {\alpha_2} {\punitinXfromYtoZ {\cat C}{\alpha_2} {\alpha_3}} {\alpha_3} {\punitinXfromYtoZ {\cat C}{\alpha_1} {\alpha_3}}$:
  \[
  \lambda \cat C. \punitelimtoYwithkontZ {\hlambda {\alpha_1}{\alpha_2} \homrallXYtoZ {\alpha_3} {\punitinXfromYtoZ {\cat C} {\alpha_2} {\alpha_3}} {\punitinXfromYtoZ {\cat C} {\alpha_1} {\alpha_3}}} {\lambdaunary {\alpha} {f : \punitinXfromYtoZ {\cat C}{\alpha} {\alpha_3}} {\alpha_3} f}
  \]
  We call this term $\textrm{comp}$
\end{definition}
This has the same flavor as a directed type theory: to construct the
composite it is sufficient to define the action on the identity
morphism, which is of course the identity for composition.
%
Note that we gave a slight bias to the definition to follow by
induction on the \emph{first} function, whereas we could make the
following alternative definition, naturally of type
$\pendallXdotY {\alpha_2} \homrallXYtoZ {\alpha_3} {\punitinXfromYtoZ {\cat C}{\alpha_2} {\alpha_3}}
{(\homlallXYtoZ {{\alpha_1}} {\punitinXfromYtoZ {\cat C}{\alpha_1} {\alpha_2}}{\punitinXfromYtoZ {\cat C}{\alpha_1} {\alpha_3}})}$:
\[
  \punitelimtoYwithkontZ {\hlambda {\alpha_2}{\alpha_3} \homlallXYtoZ {{\alpha_1}} {\punitinXfromYtoZ {\cat C}{\alpha_1} {\alpha_2}}{\punitinXfromYtoZ {\cat C}{\alpha_1} {\alpha_3}} } {\lambdaunary {\alpha} {g : \punitinXfromYtoZ {\cat C}{\alpha_1}{\alpha}} {\alpha} g}
\]

\begin{lemma}[Equivalence of Composition Definitions]
  TODO
\end{lemma}

It should follow by $\beta\eta$ that this is associative and unital.

Next, let's prove synthetically that every functor $F : \varr{\cat
  C}{\cat D}$ is functorial in this notion of composition.
\begin{definition}
  For any $F : \varr {\cat C} {\cat D}$, we can define a morphism $F_1
  : \homunary {\alpha_1} {\punitinXfromYtoZ {\cat C} {\alpha_1}
    {\alpha_2}} {\alpha_2} {\punitinXfromYtoZ {\cat D} {F \alpha_1}
    {F \alpha_2}}$ as
  \[ F_1 = \punitelimtoYwithkontZ {\hlambda {\alpha_1}{\alpha_2} \punitinXfromYtoZ {\cat D} {F \alpha_1}{F \alpha_2}}
  {\pendlambdaXdotY \alpha \punitrefl {\cat D} {F \alpha}}
  \]
  Note that
  \[ A[\id_\alpha/\alpha] = \id_{A} \]
  \[ A[f;g] = A[f];A[g] \]
\end{definition}
Note the similarity to the proof that all functions are congruences in
MLTT/HoTT.

And similarly, every profunctor $R : \harr {\cat C}{\cat D}$ is
\emph{pro-functorial}.
\begin{definition}
  For any $R : \varr {\cat C} {\cat D}$, we can define left and right
  actions of composition:
  \[ \homunary {\alpha_1} {\tensorexistsXwithYandZ {\alpha_2} {(\punitinXfromYtoZ {\cat C} {\alpha_1}{\alpha_2})} {\tensorexistsXwithYandZ {\alpha_3} {(\harrapp R {\alpha_2} {\alpha_3})}}{(\punitinXfromYtoZ {\cat C} {\alpha_3}{\alpha_4})}} {\alpha_4} {\harrapp R {\alpha_1}{\alpha_4}}
  \]
  Furthermore we can show that they are functorial in that the induced functions
  TODO
  are equal
\end{definition}

Next we can define natural transformations as specific kinds of terms
whose type is $\pendallXdotY {\alpha} \punitinXfromYtoZ {\cat D} {F
  \alpha}{G \alpha}$. In fact we can generalize this to any
endo-profunctor $P$:
\begin{definition}[Natural element]
  A natural element of an endo-profunctor $P : \harr {\cat C}{\cat C}$
  is a term of type $\pendallXdotY {\alpha} \harrapp P \alpha \alpha$.

  A natural transformation is the special case where $P =
  \punitinXfromYtoZ {\cat D} {F \alpha}{G \alpha}$ for functors $F, G
  : \varr {\cat C}{\cat D}$.

  This is natural in the sense that the incuded terms
  \[ \homunary {\alpha_1}{\punitinXfromYtoZ {} {\alpha_1}{\alpha_2}} {\alpha_2} {\harrapp P {\alpha_1}{\alpha_2}}  \]
  and
  \[ \pendallXdotY {\alpha_2} \homlallXYtoZ {\alpha_1} {\punitinXfromYtoZ {} {\alpha_1}{\alpha_2}} {\harrapp P {\alpha_1}{\alpha_2}}\]

  are equal under the associativity equivalence between these two
  types.
\end{definition}

\subsection{Adjoint Functors}

Next, we demonstrate that the two common definitions of adjoint
functors are equivalent: isomorphism of hom-sets and units and
co-units.
%
The proof goes through most easily, and utilizes duality most
effectively, if we factorize the construction through an intermediate
notion of when a functor \emph{represents} a profunctor.

\begin{definition}[Functor Represents a Profunctor]
  A functor $G : \varr{\cat D}{\cat C}$ right-represents a profunctor $R :
  \harr {\cat C}{\cat D}$ when there is an isomorphism of profunctors:
  \[ \punitinXfromYtoZ {\cat C} {\alpha} {G \beta} \cong \harrapp R {\alpha}{\beta}\]

  We say a functor $F : \varr{\cat C}{\cat D}$ \emph{left-represents} a profunctor $R : \harr {\cat C}{\cat D}$ when there is an isomorphism of profunctors
  \[ \punitinXfromYtoZ {\cat D} {F \alpha} {\beta} \cong \harrapp R {\alpha}{\beta}\]
\end{definition}

This is a parameterized generalization of the notion of an object
representing a functor $R : C^{op} \to Set$.
These notions are precisely dual. $F$ left-represents a profunctor $R$
precisely if it right-represents it in the opposite SCT.

Note that trivially any functor $G$ right-represents
$\hlambda\alpha\beta \punitinXfromYtoZ {\cat D}{\alpha} {G \beta}$ and
dually, any functor $F$ left-represents
$\hlambda\alpha\beta\punitinXfromYtoZ {\cat C}{F\alpha}{\beta}$.

Then we can show that any right-representation isomorphism is
determined by a co-unit:
\begin{lemma}[(Co)-unit of a Representation]
  If $G: \varr {\cat D}{\cat C}$ right-represents $R : \harr {\cat C}{\cat D}$, then there is a natural element

  \[ \pendallXdotY \beta \harrapp R {G\beta} \beta \]

  Such that the half of the isomorphism 

  \[ \punitinXfromYtoZ {\cat C} {\alpha} {G \beta} \multimap \harrapp R {\alpha}{\beta} \]

  is defined by composition with the natural element.

  Dually, any $F$ left-representing isomorphism with $R$ is determined
  by a term of type $\pendallXdotY \alpha \harrapp R {\alpha}
  {F\alpha}$.
\end{lemma}
\begin{proof}
  A homomorphism $\punitinXfromYtoZ {\cat C} {\alpha} {G \beta} \multimap \harrapp R {\alpha}{\beta}$ is by definition a term of type
  \[ \pendallXdotY \alpha \homrallXYtoZ {\beta} {(\punitinXfromYtoZ {\cat C} \alpha {G \beta})} {\harrapp R \alpha \beta} \]
  By \cref{lem:fubini}, this type is isomorphic to 
  \[ \pendallXdotY \beta \homlallXYtoZ {\alpha} {(\punitinXfromYtoZ {\cat C} \alpha {G \beta})} {\harrapp R \alpha \beta} \]
  By \cref{lem:yo-and-co-yo}, this type is isomorphic to
  \[ \pendallXdotY \beta {{\harrapp R {(G \beta)} \beta}} \]

  Therefore this half of the isomorphism is determined by a natural
  element. The construction in the proof of \cref{lem:yo-and-co-yo} is
  indeed by composition (i.e., unit elimination).

  The $F$ case is precisely dual.
\end{proof}

Combining these two, we get the unit-counit definition for adjoint
functors

\begin{definition}[Adjunction by unit/co-unit]
  We say $G : \varr {\cat D}{\cat C}$ is right adjoint to $F : \varr
  {\cat C}{\cat D}$ if there is a unit
  \[ \eta : \pendallXdotY \alpha \punitinXfromYtoZ \alpha {G(F(\alpha))}\]
  and a co-unit
  \[ \epsilon : \pendallXdotY \beta \punitinXfromYtoZ {F(G(\beta))} \beta \]

  satisfying the triangle identities...
  \[ \id_{R[\beta]} = R[\epsilon_{\beta}] \circ \eta_{R[\beta]} \]
  \[ \id_{L[\alpha]} = \epsilon_{L[\alpha]} \circ L[\eta_{\alpha}] \]
\end{definition}

\begin{theorem}[Adjunctions]
  $G$ is right adjoint to $F$ if and only if there is a natural isomorphism
  $\punitinXfromYtoZ {\cat D} {(F \alpha)} \beta \cong \punitinXfromYtoZ {\cat D} \alpha {G \beta}$
\end{theorem}
\begin{proof}
  First, by \cref{lem:co-unit}, the forward direction of the
  isomorphism is equivalent to providing a term
  \[ \eta : \pendallXdotY \alpha \punitinXfromYtoZ \alpha {G(F(\alpha)) } \]

  Second, the backward direction of the isomorphism is equivalent to
  providing a term
  \[ \epsilon : \pendallXdotY \beta \punitinXfromYtoZ {F(G(\beta))} \beta \]

  And the triangle laws state exactly that the induced functions are
  isomorphisms.
\end{proof}

\subsection{Kan Extensions/Lifts}

As a further example of the synthetic approach to universal
properties, we consider right and left \emph{Kan extensions}, which
are generalizations of limits and colimits, respectively.

The right Kan extension of $F$ along $J$ is the ``best'' (maximal) way to
fill in the following square:
\[% https://tikzcd.yichuanshen.de/#N4Igdg9gJgpgziAXAbVABwnAlgFyxMJZABgBpiBdUkANwEMAbAVxiRAGEQBfU9TXfIRQBGclVqMWbACLdeIDNjwEiZYePrNWiDgHJu4mFADm8IqABmAJwgBbJGRA4ISAEzVNUnQCk5lm-aIjs5IohJabABiBlxAA
\begin{tikzcd}
\cat C \arrow[d, "J"] \arrow[r, "F"] & \cat D \\
\cat C'                              &  
\end{tikzcd}\]
I.e. a functor $\textrm{Ran}_J F : \cat C' \to \cat D$ with a
universal natural transformation $\textrm{Ran}_J F \circ J \Rightarrow
F$. A left kan extension $\textrm{Lan}_J F$ is the same but with a
natural transformation $F \Rightarrow \textrm{Lan}_J F \circ J $.

We can express the concept in SCT as representing a certain profunctor
constructed using the profunctor hom:
\begin{definition}[Kan Extensions]
  Given functors $F : \varr {\cat C}{\cat D}$ and $J : \varr {\cat
    C}{\cat C'}$, a right Kan extension is a functor $\textrm{Ran}_J F
  : \varr {\cat C'}{\cat D} $that right-represents the profunctor
  \[ \hlambda {\beta:\cat D}{\alpha': \cat C} \homrallXYtoZ {\alpha}{(\punitinXfromYtoZ {\cat C'}{\alpha'}{J\alpha})} {\punitinXfromYtoZ {\cat D}{\beta}{F \alpha}} \]

  A left Kan extensions is a functor $\textrm{Lan}_J F : \varr {\cat
    C'}{\cat D}$ is a functor satisfying the dual property: it
  \emph{left}-represents the profunctor
  \[\hlambda {\alpha':\cat C}{\beta:\cat D} \homlallXYtoZ {\alpha}{(\punitinXfromYtoZ {\cat C'}{J\alpha}{\alpha'})} {\punitinXfromYtoZ {\cat D}{F \alpha}{\beta}}
  \]
\end{definition}

The limit of a functor $F : \varr{\cat C}{\cat D}$ is a special case
of a Kan extension where $J : \varr{\cat C}{1}$ is the unique functor
to the terminal category. Similarly, the colimit of $F$ is the left
Kan extension with respect to the same $J$.
%
Specializing to this case, the representability for limits means:
\[ \punitinXfromYtoZ {\cat D}{\beta}{\textrm{lim} F} \cong \homrallXYtoZ {\alpha}{(\punitinXfromYtoZ {\cat 1}{\alpha'}{()})}{\punitinXfromYtoZ {\cat D}{\beta}{F \alpha}} \]
And if we view a morphism in $\cat 1$ as being trivial, then this
essentially states that mapping $\beta$ to $\textrm{lim} F$ is the
same thing as constructing a cone: for all $\alpha :\cat C$, a
morphism from $\beta$ to $F \alpha$, naturally in $\alpha$.

The usual formulation of Kan extensions is in terms of a universal
morphism.
%
We can reproduce this formulation using our established theory of
equivalences for homs of profunctors.
\begin{lemma}
  One half of the representability isomorphism for a right Kan
  extension $\textrm{Ran}_J F$ is determined by a natural
  transformation
  \[ \pendallXdotY \alpha \punitinXfromYtoZ {\cat D}{(\textrm{Ran}_JF)(J\alpha)}{F\alpha} \]

  Dually, one half of the representability isomorphism for a left Kan
  extension $\textrm{Lan}_J F$ is determined by a natural
  transformation
  \[ \pendallXdotY \alpha \punitinXfromYtoZ {\cat D}{F\alpha}{(\textrm{Lan}_JF)(J\alpha)} \]
\end{lemma}
\begin{proof}
  By \cref{lem:unit-repr}, one half of representability is determined by a natural element
  \[ \pendallXdotY {\alpha'} \homrallXYtoZ {\alpha}{(\punitinXfromYtoZ {\cat C'}{\alpha'}{J\alpha})} {\punitinXfromYtoZ {\cat D}{(\textrm{Ran}_J F)\alpha'}{F \alpha}} \]
  which, by \cref{lem:fubini} is isomorphic to
  \[
  \pendallXdotY {\alpha} \homlallXYtoZ {\alpha'}{(\punitinXfromYtoZ {\cat C'}{\alpha'}{J\alpha})}{\punitinXfromYtoZ {\cat D}{(\textrm{Ran}_J F)\alpha'}{F \alpha}}
  \]
  which, by \cref{lem:yo-and-co-yo}, is isomorphic to 
  \[ \pendallXdotY \alpha \punitinXfromYtoZ {\cat D}{(\textrm{Ran}_JF)(J\alpha)}{F\alpha} \]
  by taking $P = \hlambda {\alpha'}{\alpha} {\punitinXfromYtoZ {\cat D}{(\textrm{Ran}_J F)\alpha'}{F \alpha}}$.

  The proof for left Kan extensions follows by \cref{lem:duality}.
\end{proof}

With the definition based on representability, we can easily prove the
generalization of the famous theorem, that right adjoints preserve not
just limits but all right Kan extensions

\begin{theorem}[Right/Left Adjoints preserve Right/Left Kan Extensions]
  Let $K : \varr{\cat C}{\cat D}$ and $J : \varr{\cat C}{\cat C'}$.

  If the right Kan extension $\textrm{Ran}_J K$ exists, and if $G :
  \varr {\cat D}{\cat D'}$ has a left adjoint $F : \varr{\cat D'}{\cat
    D}$, then the right Kan extension $\textrm{Ran}_J G\circ K$ exists
  and is given by $G \circ \textrm{Ran}_J K$.

  Similarly, if the left Kan extension $\textrm{Lan}_J K$ exists and
  if $F : \varr{\cat D}{\cat D'}$ has a right adjoint $G : \varr{\cat
    D'}{\cat D}$, then the left Kan extension $\textrm{Lan}_J F\circ
  K$ exists and is given by $F \circ \textrm{Ran}_J K$.
\end{theorem}
\begin{proof}
  By the composition of the following natural isomorphisms:
  \begin{align*}
    \punitinXfromYtoZ {\cat D'} {\beta'} {G ((\textrm{Ran}_J K) \alpha')} &\cong \punitinXfromYtoZ {\cat D}{F \beta'} {(\textrm{Ran}_J K) \alpha'} \tag{Adjunction}\\
    &\cong \homrallXYtoZ {\alpha}{(\punitinXfromYtoZ {\cat C'}{\alpha'}{J\alpha})} {\punitinXfromYtoZ {\cat D}{F \beta'}{K \alpha}} \tag{Ran assumption, \cref{lem:natural-restriction}}\\
    &\cong \homrallXYtoZ {\alpha}{(\punitinXfromYtoZ {\cat C'}{\alpha'}{J\alpha})} {\punitinXfromYtoZ {\cat D}{\beta'}{G(K \alpha)}} \tag{Adjunction}\\
  \end{align*}
\end{proof}

\section{Internal Category Theory}

A downside of working synthetically is that since everything we define
must be functorial/natural, we cannot prove any theorems about
functions that are not assumed to be functorial or transformations
that are not assumed ot be natural.
%
These kinds of constructions are surprisingly common in basic category theory.
%
For instance, when defining a left adjoint $F$ to a given right
adjoint functor $U$, the functoriality of $F$ is often not assumed,
rather we are given that each $F a$ has a certain universal mapping
property and this can then be used to show the action of $F$ on
objects extends to an action on morphisms.
%
This definition is quite important for applications of category theory
to type theory: in type theory when a type constructor is defined it
is given introduction, elimination and $\beta\eta$ rules, but no
primitive syntax is given for a functorial action.
%
The fact that functoriality is determined from the universal mapping
property of the type is then a key step in proving the type theory is
sound and complete with respect to models which are expected to have
certain functors on them.

As another example, a common lemma states that an \emph{possibly
unnatural} transformation is in fact natural if it is the inverse of a
natural transformation.
%
Neither of these examples can be stated as synthetic theorems, as it
is impossible to quantify over ``possibly unnatural transformations''
or ``functions on objects''.
%
Inside the calculus, we cannot ``see'' that categories have an
underlying set of objects.
%
Fortunately there is a way to use our calculus in a different manner
to study these constructions: by defining categories \emph{internal
to} SCT.
%
This reproduces classical constructions, since there is a model where
a category $\cat C$ is interpreted simply as a set, functors as
functions and profunctors $\harr {\cat C}{\cat D}$ as \emph{spans},
i.e. sets $R$ with (source and target) projections $\cat C \leftarrow
R \rightarrow \cat D$.
%
The proofs read like the classical proofs (with some care additional
attention on the ordering of variable usage), but by virtue of being
syntactic automatically apply to any notion of category internal to a
virtual equipment, which includes classical internal categories,
enriched categories and a broad range of generalized multicategory.

We note that a synthetic approach to model this situation may be
possible were we to extend our calculus with a \emph{modality} $|\cat
C|$ that takes a category to the discrete category on the set of
objects. We discuss this in \cref{sec:discussion}.



Next, we introduce the notion of category internal to our syntactic
virtual equipment and the \emph{monoids and modules} translation that
shows that, with the inclusion of hom sets, our syntax is already
suitable for doing syntactic category theory.

The benefits of our syntax become clear when we define internal
categories, functors, profunctors and transformations: the definitions
read exactly as the classical definitions.
\begin{definition}[Internal Category]
  An \emph{(internal) category} consists of\footnote{the following is
    easily interpreted as a $\Sigma$ type}:
  \begin{enumerate}
  \item A sort $C_0 \isaCat$
  \item A span $C_1 : \harr{C_0}{ C_0}$
  \item Identity arrows
    $C_r : \pendallXdotY {\alpha :  C_0} {\harrapp{C_1} \alpha \alpha}$
  \item Composition of arrows
    \[ C_t : \pendallXdotY \alpha {\homrallXYtoZ \beta {\harrapp{C_1} \alpha \beta} {\homrallXYtoZ \gamma {\harrapp{C_1} \beta \gamma} {\harrapp{C_1} \alpha \gamma}}}\]
    
    %% \[ C_t^\triangleright : \forall \alpha :  C_0. \forall \beta. ({C_1} \beta \alpha) \triangleright \forall \gamma. ({C_1} \gamma \beta) \triangleright {C_1} \gamma \alpha \]

    %% or

    %% \[ C_t^\triangleleft : \forall \alpha. \forall \beta. (\forall \gamma. {C_1} \alpha \gamma \triangleleft {C_1} \beta \gamma) \triangleleft {C_1} \alpha \beta
    %% \]

    %% or using tensors and covariant homs

    %% \[ C_t^\odot : \forall \alpha. \forall \gamma. (\exists \beta. {C_1}\alpha\beta\odot {C_1}\beta\gamma)\triangleright {C_1}\alpha \gamma \]

    %% or using tensors and contravariant homs

    %% \[ C_t^\odot : \forall \gamma. \forall \alpha. {C_1}\alpha\gamma \triangleleft (\exists \beta. {C_1}\alpha\beta\odot {C_1}\beta\gamma)\]
  \item Satisfying right unitality, i.e., for any $\Phi \vdash f :
    \harrapp {C_1} A B$, we'd like $f; r(B) = f$. We can formalize this
    using an equality of two transformations of type $\pendallXdotY
    \alpha {\homrallXYtoZ \beta {\harrapp{C_1} \alpha \beta} {\harrapp{C_1} \alpha \beta}}$

    \[ \pendlambdaXdotY \alpha {\homrlambdaXatYdotZ f \beta {\homrappXtoYatZ {\homrappXtoYatZ {\pendappXtoY {C_t} \alpha} {f} \beta} {\pendappXtoY {C_r} \beta} \beta}}
    = \pendlambdaXdotY \alpha {\homrlambdaXatYdotZ f \beta f} \]

    These kinds of equalities between 2-cells are very common in
    internal category theory, so to save space we will suppress the
    sequence of $\lambda$s that start such equalities. So the above can simply be written as

    \[ { {\homrappXtoYatZ {\homrappXtoYatZ {\pendappXtoY {C_t} \alpha} {f} \beta} {\pendappXtoY {C_r} \beta} \beta}} = { f}\]

  \item and left unitality (an equality of the same type as right unitality):


    \[ { {\homrappXtoYatZ {\homrappXtoYatZ {\pendappXtoY {C_t} \alpha} {\pendappXtoY {C_r} \alpha} \alpha} f \beta}} = { f}\]
  \item and associativity, which is an equality at type
    \[ \pendallXdotY \alpha {\homrallXYtoZ \beta {\harrapp{C_1} \alpha \beta} {\homrallXYtoZ \gamma {\harrapp{C_1} \beta \gamma} {\homrallXYtoZ \delta {\harrapp{C_1} \gamma \delta} {\harrapp{C_1} \alpha \delta}}}}\]

    \[ \homrappXtoYatZ {\homrappXtoYatZ {\pendappXtoY {C_t}\alpha} f \beta} {(\homrappXtoYatZ {\homrappXtoYatZ {\pendappXtoY {C_t}\beta} g \gamma} h \delta)} \delta
    =
    \homrappXtoYatZ {\homrappXtoYatZ {\pendappXtoY {C_t}\alpha} {(\homrappXtoYatZ {\homrappXtoYatZ {\pendappXtoY {C_t}\alpha} f \beta} g \gamma)} \gamma} h \delta \]
  \end{enumerate}
\end{definition}

We can define an internal functor to be a pair of an object function
and a function on arrows that preserves identity and 
\begin{definition}[Internal Functor]
  A functor $F$ between internal categories $C$ and $D$ consists of
  \begin{enumerate}
  \item a map of objects $F_0 : \varr {C_0} {D_0}$
  \item a map of arrows
    $F_1 : \pendallXdotY {\alpha}
    {\homrallXYtoZ \beta {\harrapp{C_1} \alpha \beta} {\harrapp{D_1} {F_0 \alpha}{F_0 \beta}}}$
  \item That preserves identity. This is given by equality between the
    following terms of type $\forall \alpha:{{C_0}}. D_1 {(F_0 \alpha)}{(F_0\alpha)}$:
    \[ \homrappXtoYatZ {\pendappXtoY {F_1}\alpha} {\pendappXtoY {C_r} \alpha} \alpha = \pendappXtoY {D_r}{F_0\alpha} \]
  \item And preserves composition.
    This is given by an equality between the following terms of type

    \[ \pendallXdotY \alpha
    \homrallXYtoZ {\beta} {\harrapp{C_1} \alpha \beta} {\homrallXYtoZ \gamma {\harrapp{C_1} \beta \gamma} {\harrapp{D_1} {F_0 \alpha} {F_0 \gamma}}}\]

    First, we can compose in $C$, and then map using $F$:
    \[ \homrappXtoYatZ {\pendappXtoY{F_1} \alpha} {(\homrappXtoYatZ {\homrappXtoYatZ {\pendappXtoY {C_t} \alpha} f \beta} g \gamma)}\gamma\]

    
    or we can map the morphisms with $F$, and then compose them in $D$:
    \[ \homrappXtoYatZ {\homrappXtoYatZ {\pendappXtoY {D_t} {(F_0 \alpha)}} {(\homrappXtoYatZ {\pendappXtoY {F_1} \alpha} f \beta)} {(F_0 \beta)}} {(\homrappXtoYatZ {\pendappXtoY {F_1} \beta} g \gamma)} {{F_0} \gamma}\]
  \end{enumerate}
\end{definition}

We can immediately define identity and composition of functors.

\begin{definition}
  For any $C : \textrm{Cat}$, we can define an identity functor
  $\textrm{Id}_C : \textrm{Functor} C C$ by
  \begin{enumerate}
  \item $\textrm{Id}_0 = \lambda \alpha:C_0. \alpha$
  \item $\textrm{Id}_1 = \pendlambdaXdotY \alpha
    {\homrlambdaXatYdotZ f {\alpha'} f}$
  \item Preservation of identity requires proving
    \[
    \lambda \alpha. \pendappXtoY {C_r}{\textrm{Id}_0\alpha} \]
    is equal to
    \[ \lambda \alpha.
    \homrappXtoYatZ {\pendappXtoY {\textrm{Id}_1} \alpha} {(\pendappXtoY {C_r}\alpha)} \alpha\]
    Both sides simply $\beta$ reduce to $ \lambda \alpha. \pendappXtoY {C_r} \alpha$
    and the result follows by reflexivity.
  \item Preservation of composition requires proving (under a $\lambda \alpha. \homrlambdaXatYdotZ f \beta \homrlambdaXatYdotZ g \gamma$)
    \[
    \homrappXtoYatZ {\pendappXtoY {{\textrm{Id}}_1} \alpha} {(\homrappXtoYatZ {\homrappXtoYatZ {\pendappXtoY {C_t} \alpha} f \beta} g \gamma)}\gamma
    = \homrappXtoYatZ {\homrappXtoYatZ {\pendappXtoY {C_t} {{\textrm{Id}}_0 \alpha}} {(\homrappXtoYatZ {\pendappXtoY {{\textrm{Id}}_1} \alpha} f \beta)} {{\textrm{Id}}_0 \beta}} {(\homrappXtoYatZ {\pendappXtoY{{\textrm{Id}}_1} \beta} g \gamma)} {{\textrm{Id}}_0 \gamma}\]
    Both sides $\beta$ reduce to

    \[ \homrappXtoYatZ {\homrappXtoYatZ {\pendappXtoY {C_t} \alpha} f \beta} g \gamma\]
    so the result holds by reflexivity.
  \end{enumerate}
\end{definition}

\begin{definition}
  Given categories $C,D,E$ and functors $F : \textrm{Functor} C D$ and
  $G : \textrm{Functor} D E$, we define the composition $G \circ F :
  \textrm{Functor} C E$ as
  \begin{enumerate}
  \item $(G \circ F)_0 = \lambda \alpha. {G_0 ({F_0} \alpha)}$
  \item $(G \circ F)_1 =
    \pendlambdaXdotY \alpha {
    \homrlambdaXatYdotZ f \beta
    {\homrappXtoYatZ {\pendappXtoY {G_1} {F_0 \alpha}} {\homrappXtoYatZ {\pendappXtoY {F_1} \alpha} f \beta} {F_0 \beta}}}$
  \item Preservation of identity requires proving
    \[ \lambda \alpha. \pendappXtoY {E_r} {(G \circ F)_0\alpha} \]
    is equal to
    \[ \lambda \alpha.
    \homrappXtoYatZ {\pendappXtoY {(G \circ F)_1} \alpha} {\pendappXtoY {C_r}\alpha} \alpha \]

    Expanding definitions and $\beta$ reducing we need to show

    \[ \pendlambdaXdotY\alpha {\pendappXtoY {E_r}{G_0 (F_0 \alpha)}} \]
    is equal to
    \[ \pendlambdaXdotY \alpha {\homrappXtoYatZ {\pendappXtoY {G_1} {F_0 \alpha}} {(\homrappXtoYatZ {\pendappXtoY {F_1} \alpha} {\pendappXtoY {C_r} \alpha} \alpha)} {F_0 \alpha}}\]

    Proceeding from right to left applying transitivity:
    \begin{align*}
      &\pendlambdaXdotY \alpha {\homrappXtoYatZ {\pendappXtoY {G_1} {F_0 \alpha}} {(\homrappXtoYatZ {\pendappXtoY {F_1} \alpha} {\pendappXtoY {C_r} \alpha} \alpha)} {F_0 \alpha}}\\
      &=\pendlambdaXdotY \alpha {\homrappXtoYatZ {\pendappXtoY {G_1} {F_0 \alpha}} {(\pendappXtoY{(\pendlambdaXdotY \alpha {\homrappXtoYatZ {\pendappXtoY {F_1} \alpha} {\pendappXtoY {C_r} \alpha} \alpha})}\alpha)}} {F_0 \alpha}\\
      &=\pendlambdaXdotY \alpha {\homrappXtoYatZ {\pendappXtoY {G_1} {F_0 \alpha}} {(\pendappXtoY{(\pendlambdaXdotY \alpha {{\pendappXtoY {D_r} {F_0 \alpha}}})}\alpha)}} {F_0 \alpha}\tag{*}\\
      &=\pendlambdaXdotY \alpha {\homrappXtoYatZ {\pendappXtoY {G_1} {F_0 \alpha}} {\pendappXtoY {D_r} {F_0 \alpha}}} {F_0 \alpha}\\
      &= \pendlambdaXdotY \alpha
      \pendappXtoY {(\pendlambdaXdotY \beta {{\homrappXtoYatZ {\pendappXtoY {G_1} \beta} {\pendappXtoY {D_r} \beta} \beta}})} {F_0 \alpha}\\
      &= \pendlambdaXdotY \alpha
      \pendappXtoY {(\pendlambdaXdotY \beta \pendappXtoY {E_r} {G_0 \beta})} {F_0 \alpha}\tag{*}\\
      &= \pendlambdaXdotY \alpha \pendappXtoY {E_r} {G_0 (F_0 \beta)}\\
    \end{align*}

    Where the two marked lines follow from congruence using the
    assumptions about $F,G$.

  \item Preservation of Composition.
    
    \[ \homrappXtoYatZ {\pendappXtoY{(G \circ F)_1} \alpha} {(\homrappXtoYatZ {\homrappXtoYatZ {\pendappXtoY {C_t} \alpha} f \beta} g \gamma)}\gamma\]
    equals

    \[ \homrappXtoYatZ {\homrappXtoYatZ {\pendappXtoY {E_t} {(G \circ F)_0 \alpha}} {(\homrappXtoYatZ {\pendappXtoY {(G\circ F)_1} \alpha} f \beta)} {(G\circ F)_0 \beta}} {(\homrappXtoYatZ {\pendappXtoY {(G\circ F)_1} \beta} g \gamma)} {{F_0} \gamma}\]

    Reducing, we need to show
    \[ \homrappXtoYatZ {\pendappXtoY{G_1} {F_0 \alpha}} {(\homrappXtoYatZ {\pendappXtoY {F_1} \alpha} {(\homrappXtoYatZ {\homrappXtoYatZ {\pendappXtoY {C_t}\alpha} f {\beta}} g {\gamma})} \gamma)} {F_0 \gamma} \]

    is equal to
    \[ \homrappXtoYatZ {\homrappXtoYatZ {\pendappXtoY {E_t} {G_0(F_0\alpha)}}
      {({\homrappXtoYatZ {\pendappXtoY {G_1} {F_0 \alpha}} {(\homrappXtoYatZ {\pendappXtoY {F_1} \alpha} f \beta)} {F_0 \beta}})}
      {G_0(F_0\beta)}}
       {({\homrappXtoYatZ {\pendappXtoY {G_1} {F_0 \beta}} {(\homrappXtoYatZ {\pendappXtoY {F_1} \beta} g \gamma)} {F_0 \gamma}})}
       {{F_0} \gamma}\]

    Going from left to right:
    \begin{align*}
      &\homrappXtoYatZ {\pendappXtoY{G_1} {F_0 \alpha}} {(\homrappXtoYatZ {\pendappXtoY {F_1} \alpha} {(\homrappXtoYatZ {\homrappXtoYatZ {\pendappXtoY {C_t}\alpha} f {\beta}} g {\gamma})} \gamma)} {F_0 \gamma}\\
      &=
      \homrappXtoYatZ
      {\pendappXtoY{G_1} {F_0 \alpha}}
      {(({\homrappXtoYatZ {\homrappXtoYatZ{\pendappXtoY{(\pendlambdaXdotY \alpha {\homrlambdaXatYdotZ f \beta {\homrlambdaXatYdotZ g \gamma {\homrappXtoYatZ {\pendappXtoY {F_1} \alpha} {(\homrappXtoYatZ {\homrappXtoYatZ {\pendappXtoY {C_t}\alpha} f {\beta}} g {\gamma})} \gamma}}})}\alpha} f \beta} g \gamma}))}
      {F_0 \gamma}\\
      &=
      \homrappXtoYatZ
      {\pendappXtoY{G_1} {F_0 \alpha}}
      {(({\homrappXtoYatZ {\homrappXtoYatZ{\pendappXtoY{(\pendlambdaXdotY \alpha {\homrlambdaXatYdotZ f \beta {\homrlambdaXatYdotZ g \gamma {\homrappXtoYatZ {\homrappXtoYatZ {\pendappXtoY {D_t} {(F_0 \alpha)}} {(\homrappXtoYatZ {\pendappXtoY {F_1} \alpha} f \beta)} {(F_0 \beta)}} {(\homrappXtoYatZ {\pendappXtoY {F_1} \beta} g \gamma)} {{F_0} \gamma}}}})}\alpha} f \beta} g \gamma}))}
      {F_0 \gamma}\tag{*}\\
      &=
      \homrappXtoYatZ
      {\pendappXtoY{G_1} {F_0 \alpha}}
      {\homrappXtoYatZ {\homrappXtoYatZ {\pendappXtoY {D_t} {(F_0 \alpha)}} {(\homrappXtoYatZ {\pendappXtoY {F_1} \alpha} f \beta)} {(F_0 \beta)}} {(\homrappXtoYatZ {\pendappXtoY {F_1} \beta} g \gamma)} {{F_0} \gamma}}      
      {F_0 \gamma}\\
      &=
      {(({\homrappXtoYatZ {\homrappXtoYatZ{\pendappXtoY{(\pendlambdaXdotY {\alpha'} {\homrlambdaXatYdotZ {f'} {\beta'} {\homrlambdaXatYdotZ {g'} {\gamma'} {{\homrappXtoYatZ {\pendappXtoY {G_1} {\alpha'}} {(\homrappXtoYatZ {\homrappXtoYatZ {\pendappXtoY {D_t}{\alpha'}} {f'} {\beta'}} {g'} {\gamma'})} {\gamma'}}}}})}{F_0\alpha}} {(\homrappXtoYatZ {\pendappXtoY {F_1} \alpha} f \beta)} {F_0\beta}} {(\homrappXtoYatZ {\pendappXtoY {F_1} \beta} g \gamma)} {F_0\gamma}}))}\\
      &=
      {(({\homrappXtoYatZ {\homrappXtoYatZ{\pendappXtoY{(\pendlambdaXdotY {\alpha'} {\homrlambdaXatYdotZ {f'} {\beta'} {\homrlambdaXatYdotZ {g'} {\gamma'} {{\homrappXtoYatZ {\homrappXtoYatZ {\pendappXtoY {E_t} {(G_0 {\alpha'})}} {(\homrappXtoYatZ {\pendappXtoY {G_1} {\alpha'}} {f'} {\beta'})} {(G_0 \beta')}} {(\homrappXtoYatZ {\pendappXtoY {G_1} {\beta'}} {g'} {\gamma'})} {{G_0} \gamma'}}}}})}{F_0\alpha}} {(\homrappXtoYatZ {\pendappXtoY {F_1} \alpha} f \beta)} {F_0\beta}} {(\homrappXtoYatZ {\pendappXtoY {F_1} \beta} g \gamma)} {F_0\gamma}}))}\\
      &= \homrappXtoYatZ {\homrappXtoYatZ {\pendappXtoY {E_t} {G_0(F_0\alpha)}}
      {({\homrappXtoYatZ {\pendappXtoY {G_1} {F_0 \alpha}} {(\homrappXtoYatZ {\pendappXtoY {F_1} \alpha} f \beta)} {F_0 \beta}})}
      {G_0(F_0\beta)}}
       {({\homrappXtoYatZ {\pendappXtoY {G_1} {F_0 \beta}} {(\homrappXtoYatZ {\pendappXtoY {F_1} \beta} g \gamma)} {F_0 \gamma}})}
       {{F_0} \gamma}
    \end{align*}
  \end{enumerate}
\end{definition}

\begin{definition}[Internal Profunctor]
  A profunctor $R$ between internal categories $\cat C$ and $\cat D$
  consists of
  \begin{enumerate}
  \item A span $R_s : \harr {C_0} {D_0}$.
  \item A ternary ``mixed composition'' action:
    $R_t :\pendallXdotY \alpha {\homrallXYtoZ {\alpha'} {\harrapp {C_1} {\alpha}{\alpha'}} {\homrallXYtoZ {\beta'} {\harrapp {R_s} {\alpha'}{\beta'}} {\homrallXYtoZ {\beta} {\harrapp {D_1} {\beta'} {\beta}} {\harrapp {R_s} {\alpha}{\beta}}}}}$

  \item Satisfying unitality (under a $\pendlambdaXdotY \alpha {\homrlambdaXatYdotZ f \beta {}}$):
    \[
    f = \homrappXtoYatZ {\homrappXtoYatZ {\homrappXtoYatZ {\pendappXtoY {R_t} \alpha} {\pendappXtoY {C_r} \alpha} \alpha}f \beta} {\pendappXtoY {D_r} \beta} \beta
    \]
    
  \item And associativity (both with a large implicit
    $\pendlambdaXdotY \alpha {\homrlambdaXatYdotZ g \beta \homrlambdaXatYdotZ {g'} {\beta'}
    \homrlambdaXatYdotZ {r} {\beta''}\homrlambdaXatYdotZ {f'} {\alpha''}\homrlambdaXatYdotZ {f} {\alpha'}}$)

    % f : a -> a'
    % f' : a' -> a''
    % r : a'' -> b''
    % g' : b'' -> b'
    % g : b' -> b
    %%
    %
    % f o (f' o r o g') o g
    %% and
    % (f o f') o r o (g' o g)

    \[ \homrappXtoYatZ {\homrappXtoYatZ {\homrappXtoYatZ {\pendappXtoY {R_t} {\alpha}} {f} {\alpha'}}{({\homrappXtoYatZ {\homrappXtoYatZ {\homrappXtoYatZ {\pendappXtoY {R_t} {\alpha'}} {f'} {\alpha''}}r {\beta''}} {g'} {\beta'}})} {\beta'}} {g} {\beta}\]
    is equal to
    \[ \homrappXtoYatZ
       {\homrappXtoYatZ {\homrappXtoYatZ {\pendappXtoY {R_t} {\alpha}} {(\homrappXtoYatZ {\homrappXtoYatZ {\pendappXtoY {C_t} {\alpha}} {f} {\alpha'}} {f'} {\alpha''})} {\alpha''}} r {\beta''}}
       {(\homrappXtoYatZ {\homrappXtoYatZ {\pendappXtoY {D_t} {\beta''}} {g'} {\beta'}} g {\beta})} {\beta}\]
  \end{enumerate}
\end{definition}

We can immediately define the Hom profunctor for any category $C$ as
$(\text{Hom}_C R)_s = C_1$ with actions given by $C_t$.

Next, we show that Profunctors can be restricted by functors.

\begin{definition}[Restriction of Profunctors]
  If $C, D, C', D' : \text{Cat}$, and $F : \text{Functor} C C', G : \text{Functor} D D'$ and
  $R : \text{Prof} C' D'$, then we can define a profunctor $R[F;G] : \text{Prof} C D$ by
  \begin{enumerate}
  \item $R[F;G]_s = \lambda (\alpha:C,\beta:D). R_s (F_0 \alpha) (G_0 \beta)$
  \item with composition action given by

    \[ R[F;G]_t =
    \pendlambdaXdotY \alpha {\homrlambdaXatYdotZ f {\alpha'} {\homrlambdaXatYdotZ r {\beta'} {\homrlambdaXatYdotZ g {\beta} {\homrappXtoYatZ {\homrappXtoYatZ {\homrappXtoYatZ {\pendappXtoY {R_t} {F_0 \alpha}} {(\homrappXtoYatZ {\pendappXtoY {F_1} {\alpha}} f {\alpha'})} {F_0 \alpha'}} r {G_0 \beta'}} {(\homrappXtoYatZ {\pendappXtoY {G_1} {\beta'}} g \beta)} {G_0 \beta}}}} }
 \]
  %% \item $\cat C'$ composition:
  %%   \[ \lambda \alpha. \lambda^\triangleleft (\beta, (f : C_1 \alpha \alpha',\alpha',\alpha', r : R_s \alpha' \beta)). R_l(F_1 f, r) \]
  %% \item $\cat D'$ composition:
  %%   \[ \lambda \alpha. \lambda^\triangleleft ('\beta, (r : R_s \alpha \alpha',\alpha', \beta',h : D_1 \beta' \beta)). R_r(r, G_1 h) \]
  \end{enumerate}
  Identity and associativity properties should follow. TODO
\end{definition}

\begin{definition}[Homomorphism]
  Given $R : \text{Prof} C D$ and $S : \text{Prof} C D$, a
  homomorphism $m$ from $R$ to $S$ consists of
  \begin{enumerate}
  \item $m_f : \pendallXdotY \alpha {\homrallXYtoZ \beta {\harrapp R \alpha \beta}{\harrapp S \alpha \beta}}$
    % m_f (R_t f r g) = S_t f (m_f r) g
  \item Satisfying naturality (under a $\ldots$)

    \[
    \homrappXtoYatZ {\pendappXtoY{m_f} \alpha} {(\homrappXtoYatZ {\homrappXtoYatZ {\homrappXtoYatZ {\pendappXtoY {R_t} \alpha} f {\alpha'}} r {\beta'}} g \beta)} \beta
    =
    \homrappXtoYatZ {\homrappXtoYatZ {\homrappXtoYatZ {\pendappXtoY {S_t} \alpha} {f} {\alpha'}} {(\homrappXtoYatZ {\pendappXtoY {m_f}{\alpha'}} r {\beta'})} {\beta'}} g \beta
    \]
  \end{enumerate}
\end{definition}

\begin{definition}
  An isomorphism between profunctors $R$ and $S$ is a mutually inverse
  pair of homomorphisms.
\end{definition}

Additionally, we can define natural transformations between functors,
but more generally, we can define a natural element of any
endo-profunctor, where natural transformations from $F$ to $G$ are the
same as natural elements of $\text{Hom}(F-,G=)$.

\begin{definition}[Natural Element]
  A natural element $\phi$ of a profunctor $R : \text{Prof}\,C C$ is a term
  \[ \phi : \pendallXdotY \alpha {\harrapp R \alpha \alpha} \]
  that is natural in that
  \[
  \homrappXtoYatZ {\homrappXtoYatZ {\pendappXtoY {C_t} \alpha} {\pendappXtoY \phi \alpha} \alpha} f {\alpha'}
  = \homrappXtoYatZ {\homrappXtoYatZ {\pendappXtoY {C_t } \alpha} f {\alpha'}} {\pendappXtoY \phi {\alpha'}}{\alpha'}
  \]
\end{definition}

\begin{definition}[Natural Transformation]
  A natural transformation from $F : \text{Functor} \,C\, D$ and $G :
  \text{Functor}\,C\,D$ is a natural element of $\text{Hom}(D)[F;G]$
\end{definition}

Composition of natural transformations is then given by $C_t$.

Natural elements are closely related to homomorphisms: a natural
element equivalent to a homomorphism out of a Hom profunctor. This is
our first instance of a ``Yoneda'' principle that the behavior of some
structure that varies naturally with the morphisms of a category is
fully determined by its action on the identity morphisms.

\begin{theorem}
  Natural elements of $R : \text{Prof}\, C C$ are equivalent to
  homomorphisms from $\text{Hom}_C$ to $R$.
\end{theorem}
\begin{enumerate}
\item Given a natural element $\phi$ of $R$, we can define a
  homomorphism by
  \[ \pendlambdaXdotY \alpha {\homrlambdaXatYdotZ f {\alpha'} \homrappXtoYatZ {\homrappXtoYatZ {\homrappXtoYatZ {\pendappXtoY {R_t} \alpha} {\pendappXtoY {C_r} \alpha} \alpha}{\pendappXtoY \phi \alpha} {\alpha}} f {\alpha'}} \]
\item Given a homomorphism $m$ from $\text{Hom}(C)$ to $R$, define the
  element as
  \[ \pendlambdaXdotY \alpha {\homrappXtoYatZ {\pendappXtoY {m_f} \alpha} {\pendappXtoY {C_r}\alpha} \alpha}  \]

  We should be able to show this is an isomorphism in the logic TODO!
\end{enumerate}

%% This last definition is probably unfamiliar to the reader that is not
%% experienced with profunctors.
%% %
%% However, we can show that the traditional definition is equivalent to
%% a natural transformation $\alpha : \cat C_0 \pipe \cdot \vdash
%% t_{\alpha} : F(\alpha) \to_{\cat D} G(\alpha')$.

%% \begin{definition}[Traditional Natural Transformation]
%%   A traditional natural transformation from $\alpha : \cat C_0 \vdash
%%   F(\alpha) : \cat D_0$ to $\alpha : \cat \vdash G(\alpha') : \cat
%%   D_0$ consists of
%%   \begin{enumerate}
%%   \item A term $\cdot \vdash t_{\alpha} : F(\alpha) \to_{\cat D} G(\alpha)$
%%   \item Such that
%%     \[ \phi : \alpha \to_{\cat C} \alpha' \vdash t_{\alpha'} \circ F_1(\phi) \equiv G_1(\phi) \circ t_{\alpha} : F(\alpha) \to_{\cat D} G(\alpha')
%%     \]
%%   \end{enumerate}
%% \end{definition}

%% \begin{theorem}[Equivalent Presentations of Natural Transformations]
%%   There is a bijection between traditional natural transformations $t
%%   : F \to G$ and profunctor homomorphisms $h : \alpha \to_{\cat C}
%%   \alpha' \vdash F(\alpha) \to_{\cat D} G(\alpha')$ given by
%%   \begin{enumerate}
%%   \item From a natural transformation $t$ we define a homomorphism
%%     $h(t)$ by composing on the $F$ side. The choice of this side is
%%     arbitrary and we could just as easily have used $G$ since they are
%%     equal by naturality.
%%     \[ \phi : \alpha \to_{\cat C} \alpha' \vdash t_{\alpha'} \circ F_1(\phi) : F(\alpha) \to_{\cat D} G_1(\alpha') \]
%%   \item From a homomorphism $\phi.h$, we define a natural transformation $t(h)$ as
%%     \[ \alpha \pipe \cdot \vdash h[{\id_{\alpha}}/\phi] \]
%%   \end{enumerate}
%% \end{theorem}
%% \begin{proof}
%%   First we show that the maps $h(\cdot),t(\cdot)$ actually produce
%%   homomorphisms and natural transformations.
%%   \begin{enumerate}
%%   \item On the left, we need to show $h(t)_{\phi} \circ F_1(\psi) =
%%     h(t)_{|phi \circ \psi}$. Expanding the definition, we just need
%%     associativity and functoriality of $F_1$:
%%     \begin{align*}
%%       h(t)_{\phi} \circ F_1(\psi) &\equiv (t_{\alpha'} \circ F_1(\phi)) \circ F_1(\psi)\\
%%       &\equiv t_{\alpha'} \circ F_1(\phi \circ \psi)\\
%%       &\equiv h(t)_{\phi \circ \psi}
%%     \end{align*}
%%     on the right, we use naturality:
%%     \begin{align*}
%%       G_1(\theta) \circ h(t)_{\phi} &\equiv G_1(\theta) \circ (t_{\alpha'} \circ F_1(\phi))\\
%%       &\equiv (t_{\alpha''} \circ F_1(\theta)) \circ F_1(\phi)\\
%%       &\equiv t_{\alpha''} \circ F_1(\theta \circ \phi)\\
%%       &\equiv t_{\theta \circ \phi}
%%     \end{align*}
%%   \item In the other direction, to prove naturality:
%%     \begin{align*}
%%       h[\id_\alpha/\phi] \circ F_1(\psi) &\equiv h[\id_\alpha \circ \psi/\phi]\\
%%       &\equiv h[\psi/\phi]\\
%%       &\equiv h[\psi \circ \id_{\alpha'}/\phi]\\
%%       &\equiv G_1(\psi) \circ h[\id_{\alpha}/\phi]
%%     \end{align*}
%%   \end{enumerate}
%%   Next we seek to prove that the functions are mutually inverse.
%%   \begin{enumerate}
%%   \item Round trip for homomorphisms:
%%     \begin{align*}
%%       h(t(h))_{\phi} &= t(h)_{\alpha} \circ F_1[\phi]\\
%%       &= h[\id_{\alpha}] \circ F_1[\phi]\\
%%       &= h[\id \circ \phi]\\
%%       &= h[\phi]
%%     \end{align*}
%%   \item Round trip for natural transformations uses the fact that
%%     functors preserve identity.
%%     \begin{align*}
%%       t(h(t))_{\alpha} &= h(t)_{\id_{\alpha}}\\
%%       &= t_{\alpha} \circ F_1[\id_{\alpha}]\\
%%       &= t_{\alpha} \circ \id_{F_0\alpha}\\
%%       &= t_{\alpha}
%%     \end{align*}
%%   \end{enumerate}
%% \end{proof}

\subsection{Internal Category Translation}



\subsection{Universal Properties}

Since we have profunctors as a primitive notion, the most natural
presentation of universal properties is given by the notion of a
functor representing a profunctor.

First, we have the ``algebraic'' notion of a functor representing a
profunctor, i.e., when a profunctor $R : C\pto D$ is equivalent to
$Hom_D(-,G=)$.
\begin{definition}[Right Representability]
  Let $C, D : \text{Cat}$ and $Q : \textrm{Prof}\,C D$ be a profunctor
  between them. Then a \emph{right representation} of $Q$ consists of
  \begin{enumerate}
  \item A functor $G : \textrm{Functor}\, D C$
  \item An isomorphism $I,I^-$ between $Q$ and $\text{Hom}(C)[Id;G]$
  \end{enumerate}
\end{definition}

This is a nice definition for applications, but a lot of the structure
(homomorphism, functoriality of G) is determined by other components
of the structure. This can be simplified to the notion of a
``universal morphism''.

\begin{definition}[Universal Morphism]
  A profunctor $Q : \textrm{Prof}\, C D$ is right representable when
  there exists
  \begin{enumerate}
  \item A function on objects $G_0 : \varr {D_0} {C_0}$
  \item A ``universal morphism'', i.e., a term $\epsilon :
    \pendallXdotY \beta {\harrapp Q {G_0 \beta} \beta}$ (not
    presupposed to be natural).
  \item An ``introduction rule'', i.e., a term $I : \pendallXdotY
    \alpha {\homrallXYtoZ \beta {\harrapp {Q_s}\alpha \beta} {\harrapp
        {C_1} \alpha {G_0\beta}}}$ (not presupposed to satisfy the
    homomorphism condition).
  \item Such that composition with $\epsilon$ is a left and right
    inverse to $I$. In more detail, we have two equalities.

    \begin{enumerate}
    \item First, the $\beta$ rule states that composing the universal
      morphism with the introduction rule applied to a morphism $q : \harrapp {Q} \alpha {\beta}$ is
      equal to the original $q$:

      \[ q = \homrappXtoYatZ {\homrappXtoYatZ {\homrappXtoYatZ {\pendappXtoY {Q_t} \alpha} {(\homrappXtoYatZ {\pendappXtoY I \alpha} {q} \beta)} {G_0 \beta}} {\pendappXtoY \epsilon \beta} {\beta}} {\pendappXtoY {D_r} \beta} {\beta} \]

    \item Second, the $\eta$ rule says that every morphism $f :
      \harrapp {C_1} \alpha {G_0 \beta}$ can be reconstructed using
      $I$ after postcomposing with $\epsilon$:

      \[ f = \homrappXtoYatZ {\pendappXtoY I \alpha} {({\homrappXtoYatZ {\homrappXtoYatZ {\homrappXtoYatZ {\pendappXtoY {Q_t} \alpha} {f} {G_0 \beta}} {\pendappXtoY \epsilon \beta} {\beta}} {\pendappXtoY {D_r} \beta} {\beta}  })} {\beta}\]

      As a special case, if $f = \pendappXtoY {C_r} {G_0 \beta}$, this shows that $\homrappXtoYatZ {\pendappXtoY I {G_0 \beta}} {\pendappXtoY \epsilon \beta} \beta = \pendappXtoY {C_r} {G_0 \beta}$
    \end{enumerate}
  \end{enumerate}
\end{definition}

These two notions turn out to be equivalent because the ``extra''
information in the hom isomorphism is in fact fully determined by the
data of a universal morphism: the functoriality of $G$ and the
naturality of $I,I_{inv}$.
%
The proof can be broken up into 3 conceptual steps:
\begin{enumerate}
\item First,, the functoriality of $G$ is derivable from $I,\epsilon$.
\item Next, $I_{inv}$ is a homomorphism automatically since composition
  with a morphism is automatically natural, where $I_{inv}$ is defined as
  \[\pendlambdaXdotY \alpha {\homrlambdaXatYdotZ f {\beta} \homrappXtoYatZ {\homrappXtoYatZ {\homrappXtoYatZ {\pendappXtoY {Q_t} \alpha} f \beta}{\pendappXtoY \epsilon \beta} {\beta}} {\pendappXtoY {D_r }\beta} {\beta}} \]

  \item Finally, $I$ is a homomorphism because the inverse of a
    natural transformation is always natural.
\end{enumerate}

%% Part of this equivalence is the \emph{Yoneda Lemma}, which we state
%% now in its internal, parameterized form.
%% The usual non-parameterized Yoneda lemma can be formed if we have a
%% unit sort.

\begin{lemma}[Universal Constructions are functorial]
  For any universal construction $(G_0,\epsilon,I)$ for $Q :
  \text{Prof}\,C\,D$, we can extend $G_0$ to a functor by defining

  \[ G_1 = \pendlambdaXdotY \beta {\homrlambdaXatYdotZ f {\beta'}
    {\homrappXtoYatZ {\pendappXtoY I {G_0\beta}} {(\homrappXtoYatZ {\homrappXtoYatZ {\homrappXtoYatZ {\pendappXtoY {Q_t} {G_0\beta}} {\pendappXtoY {C_r} {G_0\beta}} {G_0\beta}} {\pendappXtoY \epsilon {\beta}} \beta} {f} {\beta'})} {\beta'}}
  }\]

  Furthermore, $I,I^-1$ are homomorphisms with respect to the induced
  profunctor structure.
\end{lemma}
\begin{proof}
  \begin{enumerate}
  \item To prove $G_1$ preserves identity we need to show
    \[ {\homrappXtoYatZ {\pendappXtoY I {G_0\beta}} {(\homrappXtoYatZ {\homrappXtoYatZ {\homrappXtoYatZ {\pendappXtoY {Q_t} {G_0\beta}} {\pendappXtoY {C_r} {G_0\beta}} {G_0\beta}} {\pendappXtoY \epsilon {\beta}} \beta} {\pendappXtoY {D_r} {\beta}} {\beta})} {\beta}} = \pendappXtoY{C_r} {G_0\beta}\]
    which is precisely the $\eta$ principle.

  \item Next, we show $I_{inv}$ is a homomorphism from $\text{Hom}(C)[\text{Id}(C);G]$ to $Q$.
    All of this is under the binders
    \[\lambdatrinary {\alpha : C} {f : \harrapp {C_s} \alpha {\alpha'}}
    {\alpha' : C} {g : \harrapp{C_s} {\alpha'} {G_0 \beta'}}
    {\beta' : D} {h : \harrapp {D_s} {\beta'} {\beta}}
    {}\]

    % given
    % f : a -> a' in C
    % g : a' -> G b'
    % h : b' -> b

    %% I_{inv} (f ;C (g ;C G_1 h)) = Q_t f (I_{inv} g) h

    \begin{align*}
      & \appunary
      {I_{inv}}
      \alpha
      {(\appbinary {C_t} \alpha f {\alpha'} {(
          \appbinary {C_t} {\alpha'} g {G_0\beta'}
          {(\appunary {G_1} {\beta'} h \beta)} {G_0 \beta})} {G_0 \beta})}
      {\beta}\\
      &=
      \apptrinary {Q_t} \alpha
      {(\appbinary {C_t} \alpha f {\alpha'} {(
          \appbinary {C_t} {\alpha'} g {G_0\beta'}
          {(\appunary {G_1} {\beta'} h \beta)} {G_0 \beta})} {G_0 \beta})} \beta
      {\pendappXtoY \epsilon \beta} \beta
      {\pendappXtoY {D_r} \beta} {\beta}\\
      &=
      \apptrinary {Q_t} \alpha
      {(\appbinary {C_t} \alpha f {\alpha'} {(
          \appbinary {C_t} {\alpha'} g {G_0\beta'}
          {(\appunary {G_1} {\beta'} h \beta)} {G_0 \beta})} {G_0 \beta})} \beta
      {\pendappXtoY \epsilon \beta} \beta
      {(\appbinary {D_t} \beta {\pendappXtoY {D_r} \beta} \beta {\pendappXtoY {D_r} \beta} \beta)} {\beta}\\
      &=
      \apptrinary {Q_t} \alpha
      {(\appbinary {C_t} \alpha {(\appbinary {C_t} \alpha f {\alpha'} g {G_0 \beta'})} {G_0 \beta'}
        {(\appunary {G_1} {\beta'} h \beta)} {G_0 \beta})} \beta
      {\pendappXtoY \epsilon \beta} \beta
      {(\appbinary {D_t} \beta {\pendappXtoY {D_r} \beta} \beta {\pendappXtoY {D_r} \beta} \beta)} {\beta}\\
      &=
      \apptrinary {Q_t} \alpha
      {(\appbinary {C_t} \alpha f {\alpha'} g {G_0 \beta'})} {G_0 \beta'}
      {(\apptrinary {Q_t} {G_0 \beta'}
        {(\appunary {G_1} {\beta'} h {\beta})} {G_0 \beta}
        {\pendappXtoY \epsilon \beta} \beta
        {\pendappXtoY {D_r} \beta} \beta)} \beta
      {\pendappXtoY {D_r} \beta} \beta
      \\
      &=
      \apptrinary {Q_t} \alpha
      {(\appbinary {C_t} \alpha f {\alpha'} g {G_0 \beta'})} {G_0 \beta'}
      {(\appunary {I_{inv}} {G_0 \beta'} {(\appunary {G_1} {\beta'} h \beta)} \beta)} \beta
      {\pendappXtoY {D_r} \beta} \beta
      \\
      &=
      \apptrinary {Q_t} \alpha
      {(\appbinary {C_t} \alpha f {\alpha'} g {G_0 \beta'})} {G_0 \beta'}
      {(\appunary {I_{inv}} {G_0 \beta'} {(\appunary {I} {G_0\beta'} {(\apptrinary {Q_t} {G_0 \beta'} {\pendappXtoY {C_r} {G_0 \beta'}} {G_0 \beta'} {\pendappXtoY \epsilon {\beta'}} {\beta'} h \beta)} \beta)} \beta)} \beta
      {\pendappXtoY {D_r} \beta} \beta
      \\
      &=
      \apptrinary {Q_t} \alpha
      {(\appbinary {C_t} \alpha f {\alpha'} g {G_0 \beta'})} {G_0 \beta'}
      {(\apptrinary {Q_t} {G_0 \beta'} {\pendappXtoY {C_r} {G_0 \beta'}} {G_0 \beta'} {\pendappXtoY \epsilon {\beta'}} {\beta'} h \beta)} \beta
      {\pendappXtoY {D_r} \beta} \beta
      \\
      &=
      \apptrinary {Q_t} \alpha
      {(\appbinary {C_t} \alpha {(\appbinary {C_t} \alpha f {\alpha'} g {G_0 \beta'})} {G_0 \beta'}
        {\pendappXtoY {C_r} {G_0 \beta'}} {G_0 \beta'})} {G_0 \beta'}
      {\pendappXtoY \epsilon {\beta'}} {\beta'}
      {(\appbinary {D_t} {\beta'} h \beta {\pendappXtoY {D_r} \beta} \beta)} \beta\\
      &=
      \apptrinary {Q_t} \alpha
      {(\appbinary {C_t} \alpha f {\alpha'} g {G_0 \beta'})} {G_0 \beta'}
      {\pendappXtoY \epsilon {\beta'}} {\beta'}
      {(\appbinary {D_t} {\beta'} {\pendappXtoY {D_r} {\beta'}} {\beta'} h \beta )} \beta\\
      &=
      \apptrinary {Q_t} \alpha
      f {\alpha'}
      {(\apptrinary {Q_t} {\alpha'} g {G_0 \beta'} {\pendappXtoY \epsilon {\beta'}} {\beta'}{\pendappXtoY {D_r} {\beta'}} {\beta'})} {\beta'}
      h \beta\\
      &= \homrappXtoYatZ {\homrappXtoYatZ {\homrappXtoYatZ {\pendappXtoY {Q_t} \alpha} f {\alpha'}} {(\homrappXtoYatZ {\pendappXtoY {I_{inv}} {\alpha'}} {g} {\beta'})} {\beta'}} {h} {\beta}\\
    \end{align*}

    
  \item To prove $G_1$ preserves composition given $f : \harrapp {D_1} {\beta} {\beta'}$, $f' : \harrapp {D_1} {\beta'} {\beta''}$:
    \begin{align*}
      & \appunary {G_1} {\beta} {\appbinary {D_t} {\beta} f {\beta'} {f'} {\beta''}} {\beta'}\\
      &= \appunary {I} {G_0 \beta} {(\apptrinary {Q_t} {G_0 \beta} {\pendappXtoY {C_r} {G_0 \beta}} {G_0 \beta} {\pendappXtoY \epsilon {\beta}} {\beta} {({\appbinary {D_t} {\beta} f {\beta'} {f'} {\beta''}})} {\beta''})} {\beta''}\\
      &= \appunary {I} {G_0 \beta} {(\apptrinary {Q_t} {G_0 \beta} {(\appbinary {C_t} {G_0 \beta} {\pendappXtoY {C_r} {G_0 \beta}} {G_0 \beta} {\pendappXtoY {C_r} {G_0 \beta}} {G_0 \beta})} {G_0 \beta} {\pendappXtoY \epsilon {\beta}} {\beta} {({\appbinary {D_t} {\beta} f {\beta'} {f'} {\beta''}})} {\beta''})} {\beta''}\\
      &=
      \appunary {I} {G_0 \beta}
      {(\apptrinary {Q_t} {G_0 \beta}
        {\pendappXtoY {C_r} {G_0 \beta}} {G_0 \beta}
        {(\apptrinary {Q_t} {G_0 \beta}
          {\pendappXtoY {C_r} {G_0 \beta}} {G_0 \beta}
          {\pendappXtoY \epsilon {\beta}} {\beta}
          f {\beta'})} {\beta'}
        {f'} {\beta''}
        )}
      {\beta''}\\
      &=
      \appbinary {C_t} {G_0 \beta}
      {\pendappXtoY {C_r} {G_0 \beta}} {G_0 \beta}
      {(\appbinary {C_t} {G_0 \beta}
        {(\appunary {I} {G_0 \beta}
        {(\apptrinary {Q_t} {G_0 \beta}
          {\pendappXtoY {C_r} {G_0 \beta}} {G_0 \beta}
          {\pendappXtoY \epsilon {\beta}} {\beta}
          f {\beta'})} {\beta'})} {G_0 \beta'}
        {(\appunary {G_1} {\beta'} {f'} {\beta''})} {G_0 \beta''})}
      {G_0 \beta''}\\
      &=
      \appbinary {C_t} {G_0 \beta}
        {(\appbinary {C_t} {G_0 \beta}
          {\pendappXtoY {C_r} {G_0 \beta}} {G_0 \beta}
          {(\appbinary {C_t} {G_0 \beta}
            {(\appunary {I} {G_0 \beta} {\pendappXtoY \epsilon {\beta}} {\beta})} {G_0 \beta}
            {(\appunary {G_1} \beta f {\beta'})} {G_0 \beta'})} {G_0 \beta'})} {G_0 \beta'}
        {(\appunary {G_1} {\beta'} {f'} {\beta''})} {G_0 \beta''}\\
      &=
      \appbinary {C_t} {G_0 \beta}
        {(\appbinary {C_t} {G_0 \beta}
            {(\appunary {I} {G_0 \beta} {\pendappXtoY \epsilon {\beta}} {\beta})} {G_0 \beta}
            {(\appunary {G_1} \beta f {\beta'})} {G_0 \beta'})} {G_0 \beta'}
        {(\appunary {G_1} {\beta'} {f'} {\beta''})} {G_0 \beta''}\\
      &=
      \appbinary {C_t} {G_0 \beta}
        {(\appbinary {C_t} {G_0 \beta}
            {\pendappXtoY {C_r} {G_0 \beta}} {G_0 \beta}
            {(\appunary {G_1} \beta f {\beta'})} {G_0 \beta'})} {G_0 \beta'}
        {(\appunary {G_1} {\beta'} {f'} {\beta''})} {G_0 \beta''}\\
      &=
      \appbinary {C_t} {G_0 \beta}
      {(\appunary {G_1} \beta f {\beta'})} {G_0 \beta'}
        {(\appunary {G_1} {\beta'} {f'} {\beta''})} {G_0 \beta''}\\
      %% &\homrappXtoYatZ {\pendappXtoY {G_1} \beta} {(\homrappXtoYatZ {\homrappXtoYatZ {\pendappXtoY {D_t} {\beta}} {f} {\beta'}} {f'} {\beta''})} {\beta''}\\
      %% &={\homrappXtoYatZ {\pendappXtoY I {G_0\beta}} {(\homrappXtoYatZ {\homrappXtoYatZ {\homrappXtoYatZ {\pendappXtoY {Q_t} {G_0\beta}} {\pendappXtoY {C_r} {G_0\beta}} {G_0\beta}} {\pendappXtoY \epsilon {\beta}} \beta} {(\homrappXtoYatZ {\homrappXtoYatZ {\pendappXtoY {D_t} {\beta}} {f} {\beta'}} {f'} {\beta''})} {\beta''})} {\beta''}}\\
      %% &={\homrappXtoYatZ {\pendappXtoY I {G_0\beta}} {(\homrappXtoYatZ {\homrappXtoYatZ {\homrappXtoYatZ {\pendappXtoY {Q_t} {G_0\beta}} {({\pendappXtoY {C_r} {G_0\beta}})} {G_0\beta}} {\pendappXtoY \epsilon {\beta}} \beta} {(\homrappXtoYatZ {\homrappXtoYatZ {\pendappXtoY {D_t} {\beta}} {f} {\beta'}} {f'} {\beta''})} {\beta''})} {\beta''}}\\
    \end{align*}
    %% \[
    %%   {\homrappXtoYatZ {\pendappXtoY I {G_0\beta}} {(\homrappXtoYatZ {\homrappXtoYatZ {\homrappXtoYatZ {\pendappXtoY {Q_t} {G_0\beta}} {\pendappXtoY {C_r} {G_0\beta}} {G_0\beta}} {\pendappXtoY \epsilon {\beta}} \beta} {(\homrappXtoYatZ {\homrappXtoYatZ {\pendappXtoY {D_t} {\beta}} {f} {\beta'}} {f'} {\beta''})} {\beta''})} {\beta''}}
    %% \]
  \end{enumerate}
\end{proof}

\begin{lemma}[Inverse of homomorphism is a homomorphism]
  Let $\phi$ be a homomorphism from $R$ to $S : \text{Prof} C D$ and
  let $\psi : \pendallXdotY \alpha {\homrallXYtoZ \beta {\harrapp S
      \alpha \beta}{\harrapp R \alpha \beta}}$. If $\psi$ is a
  two-sided inverse to $\phi_f$ then $\psi$ is a homomorphism.
\end{lemma}
\begin{proof}
  We need to show the homomorphism condition for $\psi$:
  \[ \homrappXtoYatZ {\pendappXtoY \psi \alpha} {(\homrappXtoYatZ {\homrappXtoYatZ {\homrappXtoYatZ {\pendappXtoY {S_t} \alpha} f {\alpha'}} s {\beta'}} g {\beta})} {\beta}
    = \homrappXtoYatZ {\homrappXtoYatZ {\homrappXtoYatZ {\pendappXtoY {R_t} \alpha} f {\alpha'}} {(\homrappXtoYatZ {\pendappXtoY \psi {\alpha'}} s {\beta'})} {\beta'}} g \beta
    \]

    \begin{align*}
      \homrappXtoYatZ {\pendappXtoY \psi \alpha} {(\homrappXtoYatZ {\homrappXtoYatZ {\homrappXtoYatZ {\pendappXtoY {S_t} \alpha} f {\alpha'}} s {\beta'}} g {\beta})} {\beta}
      &= \homrappXtoYatZ {\pendappXtoY \psi \alpha} {(\homrappXtoYatZ {\homrappXtoYatZ {\homrappXtoYatZ {\pendappXtoY {S_t} \alpha} f {\alpha'}} {(\homrappXtoYatZ {\pendappXtoY {\phi_f} {\alpha'}} {(\homrappXtoYatZ {\pendappXtoY \psi {\alpha'}} s {\beta'})} {\beta'})} {\beta'}} g {\beta})} {\beta}\\
      &= \homrappXtoYatZ {\pendappXtoY \psi \alpha}
      {(\homrappXtoYatZ {\pendappXtoY {\phi_f} \alpha} {(\homrappXtoYatZ {\homrappXtoYatZ {\homrappXtoYatZ {\pendappXtoY {R_t} \alpha} f {\alpha'}} {(\homrappXtoYatZ {\pendappXtoY \psi {\alpha'}} s {\beta'})} {\beta'}} g \beta)} \beta)} {\beta}\\
      & = \homrappXtoYatZ {\homrappXtoYatZ {\homrappXtoYatZ {\pendappXtoY {R_t} \alpha} f {\alpha'}} {(\homrappXtoYatZ {\pendappXtoY \psi {\alpha'}} s {\beta'})} {\beta'}} g \beta
    \end{align*}
\end{proof}

%% \subsection{Self-Interpretation: Hom of Internal Profunctors}

%% We can define a version of the hom $\homrallXYtoZ \alpha P Q$ on
%% \emph{internal} profunctors using a combination of this connective,
%% and an \emph{equalizer} to enforce a homomorphism condition.

%% \begin{definition}[Internal Covariant Hom of Profunctors]
%%   Let $P : \textrm{Prof}\,D\,E$ and $Q : \textrm{Prof}\,C\,E$. Define
%%   the covariant Hom $P \triangleright Q : \textrm{Prof}\,C\,D$ by
%%   \[ (P \triangleright Q)_1 = \lambda (\alpha;\beta). \equalizeVofWbyXeqYatZ \phi {\homrallXYtoZ \gamma {\harrapp {P_1} \alpha \gamma} {\harrapp {Q_1} \beta \gamma}} {\psi_{dom}} {\psi_{cod}} {\homrallXYtoZ {\gamma'} {\harrapp {P_1 }\beta {\gamma'}} {\homrallXYtoZ {\gamma} {\harrapp {E_1} {\gamma'} \gamma} {\harrapp {Q_1} \alpha \gamma}}}\]
%%   where $\psi_{dom}$ and $\chi_{cod}$ are  as
%%   \[ \psi_{dom} =\homrlambdaXatYdotZ p {\gamma'} {\homrlambdaXatYdotZ e \gamma
%%     {\homrappXtoYatZ {\phi} {(\apptrinary {P_t} \beta {\pendappXtoY {D_r} \beta} \beta p {\gamma'} e \gamma)} \gamma}}
%%   \]
%%   \[
%%   \psi_{cod} = \homrlambdaXatYdotZ p {\gamma'} {\homrlambdaXatYdotZ e \gamma
%%     {\apptrinary {Q_t} \alpha {\pendappXtoY {C_r} \alpha} \alpha {(\homrappXtoYatZ \phi {p} {\gamma'})} {\gamma'} e \gamma}}
%%   \]
%% \end{definition}

%% We can easily define a contravariant hom $P \triangleleft Q$ by a dual
%% definition (TODO).


%% We can also define a left Kan extensions using the contravariant
%% cotensors.  Normally the left Kan extension is defined using tensors,
%% but the definition using profunctors doesn't look right to me. Let's
%% work backwards from the definition of colimit to see why.

%% Let $\alpha : \cat C \vdash A : \cat D$ be a functor we want the
%% colimit of. The UP of a colimit is that maps out of it are equivalent
%% to cocones from $A$. Semi-formally this looks like
%% \[ (\colim_{\alpha} A \to_{\cat D} \beta) \cong \forall \alpha:\cat C.~ A \to_{\cat D} \beta \]
%% This looks a lot like a cotensor with a trivial domain:
%% \[ (\colim_{\alpha} A \to_{\cat D} \beta) \cong \forall \alpha. (() \to_{1} \gamma) \triangleright (A \to_{\cat D} \beta) \]
%% Suggesting that we define the left Kan extension profunctor as a
%% contravariant cotensor as follows.
%% \begin{definition}
%%   Let $\cat C,\cat C',\cat D$ be cats and let $\alpha:\cat C \vdash A
%%   : \cat D$ and $\alpha : \cat C \vdash J : \cat C'$ be functors.
%%   Then the left Kan extension profunctor $\gamma^o:\cat C'; \beta:\cat
%%   D \pipe (PLan_J A)[\gamma;\beta]$ is defined as
%%   \[
%%   \gamma^o:\cat C'; \beta:\cat D \pipe (PLan_J A)[\gamma;\beta]
%%   := \forall \alpha : \cat C.~ (J \to_{\cat C'} \gamma) \triangleright (A \to_{\cat D} \beta)
%%   \]
%%   Then a left Kan extension of $A$ along $J$ is a functor $\gamma :
%%   \cat C' \vdash Lan_J A : \cat D$ that represents $PLan_J A$ in that
%%   \[ (Lan_J A \to_{\cat D} \beta) \cong (LPan_J A)[\gamma;\beta] \]
%% \end{definition}


%% \begin{definition}
  %% Let $\cat C,\cat C',\cat D$ be cats and let $\alpha:\cat C \vdash
  %% : \cat D$ and $\alpha : \cat C \vdash J : \cat C'$ be functors.
  %% Then the left Kan extension profunctor $\beta^o:\cat D; \gamma:\cat
  %% C' \pipe (PLan_J F)[\beta;\gamma]$ is defined as
%%   \[
%%   \beta^o:\cat D; \gamma:\cat C' \pipe (PLan_J A)[\beta;\gamma] :=
%%   \exists \alpha:\cat C. (\beta \to_{\cat D} A[\alpha]) \odot (J[\alpha] \to_{\cat C'} \gamma)
%%   \]
%%   And a left Kan extension is a functor $\gamma :\cat C' \vdash Lan_J
%%   A : \cat D$ that represents $PLan_J A$ in that
%%   \[ (\beta \to_{\cat D} Lan_J A[\alpha]) \cong (PLan_J A)[\beta;\alpha] \]
%% \end{definition}


%% \begin{lemma}[Parameterized Yoneda Lemma]
%%   Given categories $\cat C, \cat D$ and a profunctor $\alpha^o:\cat
%%   C;\beta : \cat D \vdash \alpha \to_{Q} \beta$ and a function
%%   $\beta : \cat D \vdash G[\beta] : \cat C$, then there is an
%%   isomorphism between
%%   \begin{enumerate}
%%   \item Terms $\cdot \vdash t : G[\beta] \to_{Q} \beta$
%%   \item left-homomorphisms $\phi : \alpha \to_{\cat C} G[\beta] \vdash
%%     u : \alpha \to_{Q} \beta$ i.e. they satisfy homomorphism on the
%%     left (the other side doesn't make sense if $G$ is not a functor):
%%     $u[\phi \circ \psi] = u[\phi] \circ \psi$
%%   \end{enumerate}
%% \end{lemma}
%% \begin{proof}
%%   First, the constructions
%%   \begin{enumerate}
%%   \item Given $t$, we define $u(t)$ to be
%%     \[\phi : \alpha \to_{\cat C} G[\beta] \vdash t \circ_{Q} \phi : \alpha \to_{Q} \beta\]

%%     Which is a left-homomorphism because:
%%     \[ (t \circ_{Q} \phi) \circ_{Q} \psi = t \circ_{Q} (\phi \circ_{\cat C} \psi)\]
%%   \item Given $u$, we define $t(u)$ to be
%%     \[ \cdot \vdash u[\id_{G[\beta]}/\phi] : G[\beta] \to_{Q} \beta \]
%%   \end{enumerate}
%%   Now we show they are inverse
%%   \begin{enumerate}
%%   \item First, the very easy case, $\cdot \vdash t(u(t)) : G[\beta] \to_{Q} \beta$:
%%     \begin{align*}
%%       t \circ_{Q} \id_{G[\beta]} &= t
%%     \end{align*}
%%     by the fact that $Q$ is a profunctor.
%%   \item Next, $\phi : \alpha \to_{\cat C} G[\beta] \vdash u(t(u)) : \alpha \to_{Q} \beta$:
%%     \begin{align*}
%%       u[\id_{G[\beta]}/\phi] \circ_{Q} \phi &= u[\id \circ_{\cat C} \phi/\phi]\\
%%       &= u[\phi/\phi]\\
%%       &= u
%%     \end{align*}
%%     which uses the fact that $u$ is a left-homomorphism.
%%   \end{enumerate}
%% \end{proof}

%% \begin{theorem}[Negative Types and Right Representables are Equivalent]
%%   For any profunctor $\alpha^o: \cat C; \beta : \cat D \pipe \alpha
%%   \to_{Q} \beta$, there is a bijection between
%%   \begin{enumerate}
%%   \item Right Represntables $\beta : \cat D \vdash G[\beta] : \cat C$ for $Q$.
%%   \item Negative Types $\beta : \cat D \vdash N[\beta] : \cat C$
%%     representing $Q$.
%%   \end{enumerate}
%% \end{theorem}
%% \begin{proof}
%%   \begin{enumerate}
%%   \item Given a right representable $G[\beta]$, we set $N(G)[\beta] =
%%     G[\beta]$, the introduction rule is the same and the elimination
%%     rule is given as in the Yoneda lemma and the inverse property
%%     follows by the Yoneda isomorphism and the fact that $I_G^{-1}$ is
%%     an inverse for $I_G$.
%%   \item Given a negative type $N[\beta]$ we set $G(N)[\beta] =
%%     N[\beta]$, the introduction rule is the same and the inverse
%%     introduction rule is given as in the Yoneda lemma.  We need to
%%     show that $N[\beta]$ has the structure of a functor and that
%%     $I_{N}$ and $I_{N}^{-1}$ are homomorphisms on the right with
%%     respect to that structure.  Define $N_1$ as follows:
%%     \begin{mathpar}
%%       \inferrule
%%           {\beta_1^o:\cat D;\beta_2:\cat D \vdash \psi \circ \epsilon : N[\beta_1/\beta] \to_{Q} \beta_2\\
%%             \phi : N[\beta_1/\beta] \to_{Q} \beta_2 \vdash I_N[\phi] : N[\beta_1/\beta] \to_{\cat C} N[\beta_2/\beta]
%%           }
%%           {\beta_1^o:\cat D;\beta_2:\cat D\pipe \psi : \beta_1 \to_{\cat D} \beta_2 \vdash I_N[\psi \circ \epsilon_{\beta_1}/\phi] : N[\beta_1/\beta] \to_{\cat C} N[\beta_2/\beta]}
%%     \end{mathpar}
%%     We need to show functoriality.
%%     For identity we have
%%     \begin{align*}
%%       I_N[\id_{\beta} \circ \epsilon_{\beta}/\phi] &= I_N[\epsilon_{\beta}]\\
%%       &= I_N[\epsilon \circ \id_{N[\beta]}]\\
%%       &= \id_{N[\beta]} \tag{$\eta$}
%%     \end{align*}
%%     For composition,
%%     \begin{align*}
%%       N_1[\theta] \circ N_1[\psi]
%%       &= I_N[\theta \circ \epsilon] \circ I_N[\psi \circ \epsilon]\\
%%       &= I_N[(\theta \circ \epsilon) \circ I_N[\psi \circ \epsilon]]\tag{left homomorphism}\\
%%       &= I_N[\theta \circ (\epsilon \circ I_N[\psi \circ \epsilon])]\\
%%       &= I_N[\theta \circ (\psi \circ \epsilon)] \tag{$\beta$}\\
%%       &= I_N[(\theta \circ \psi) \circ \epsilon]\\
%%       &= N_1[\theta \circ \psi]
%%     \end{align*}
%%     Next we show $I_N$ is a right homomorphism, the typing here is
%%     $\phi : \alpha \to_{Q} \beta_1, \theta : \beta_1 \to_{\cat D}
%%     \beta_2$.
%%     \begin{align*}
%%       N_1[\theta] \circ I_N[\phi] &= I_N[\theta \circ \epsilon] \circ I_N[\phi]\\
%%       &= I_N[\theta \circ \epsilon \circ I_N[\phi]]\tag{left homomorphism}\\
%%       &= I_N[\theta \circ \phi] \tag{$\beta$}
%%     \end{align*}
%%     and to show $I_N^{-1}$ is a right homomorphism under $\phi : \alpha \to_{\cat C}N[\beta_1], \theta : \beta_1 \to_{\cat D} \beta_2$
%%     \begin{align*}
%%       I_N^{-1}[N_1[\theta] \circ \phi] &= \epsilon \circ (N_1[\theta] \circ \phi) \tag{def}\\
%%       &= \epsilon \circ (I_N[\theta \circ \epsilon] \circ \phi)\tag{def}\\
%%       &= (\epsilon \circ I_N[\theta \circ \epsilon]) \circ \phi\tag{assoc}\\
%%       &= (\theta \circ \epsilon) \circ \phi\tag{$\beta$}\\
%%       &= \theta \circ (\epsilon \circ \phi)\tag{assoc}\\
%%       &= \theta \circ I_N^{-1}[\phi] \tag{def}
%%     \end{align*}
%%   \end{enumerate}

%%   Next we need to show that this is a \emph{bijection}.
%%   \begin{enumerate}
%%   \item Starting with a negative type $N[\beta]$, we recover the
%%     original $\epsilon$ by the Yoneda lemma.
%%   \item Starting with a right representable $G[\beta]$, we recover the
%%     original $I_N^{-1}$ by the Yoneda lemma. We need to show that the
%%     action of $G$ on arrows is the same as the one defined from $G$
%%     viewed as a negative type, which we will call $G_1'$:
%%     \begin{align*}
%%       G_1'[\theta] &= I_G[\theta \circ \epsilon]\\
%%       &= G_1[\theta] \circ I_G[\epsilon]\tag{right homomorphism}\\
%%       &= G_1[\theta] \circ I_G[\epsilon \circ \id]\\
%%       &= G_1[\theta] \circ \id\\
%%       &= G_1[\theta]
%%     \end{align*}
%%   \end{enumerate}
%% \end{proof}


%% \subsection{Adjunctions by Universal Morphism}

%% As an example of formal category theory in our syntax, let's prove the
%% equivalence between two definitions of an adjunction: one by universal
%% morphism and the other by natural hom isomorphism.
%% %
%% Though the definitions and proof will look like we are doing ordinary
%% category theory, since it is syntactic it immediately applies to all
%% models, including enriched and internal category theory.

%% We start with the usual definition of an adjunction using
%% hom-isomorphism and a profunctor.
%% \begin{definition}
%%   Let $C,D$ be categories and $c^o:C_0,d:D_0 \vdash H[c,d]$ a
%%   profunctor between them. A \emph{left adjoint} to $H$ consists of
%%   \begin{enumerate}
%%   \item A functor $c:C \vdash F[c] : D$
%%   \item A natural isomorphism $c,d \pipe h : H[c,d] \vdash
%%     F_{ind}[h]: D_1[F_0[c],d]$. Call the inverse $g. F_{ind}^{-1}[g]$.
%%   \end{enumerate}
%% \end{definition}

%% \begin{definition}
%%   Let $c:C_0,d:D_0 \vdash H[c,d]$ be a profunctor.
%%   A \emph{universal $H$-heteromorphism from $C$} consists of
%%   \begin{enumerate}
%%   \item For every $c:C_0$, an $H$-morphism $\eta_c : H(c,F_0[c])$ for
%%     some function $c:C_0 \vdash F_0[c] : D_0$.
%%   \item Such that every $h : H[c,d]$ factors through $\eta_c$ in that
%%     there is a unique $h : H[c,d] \vdash F_{ind}[h] : D[F_0[c],d]$
%%     such that $h = F_{ind}[h] \circ \eta_c$.  Uniqueness means
%%     validity of the following rule:
%%     \begin{mathpar}
%%       \inferrule
%%       {g:D[F_0[c],d] \vdash g \circ \eta = h}
%%       {g:D[F_0[c],d] \vdash g = F_{ind}[h]}
%%     \end{mathpar}
%%     (Note that we do not require $F_{ind}$ to be natural in any sense)
%%   \end{enumerate}
%% \end{definition}

%% It is easy to see that a left adjoint gives a universal heteromorphism
%% by a Yoneda-like argument:
%% \begin{construction}
%%   Let $F$ be a left adjoint to $H$. Then we define a left universal
%%   $H$-heteromorphism by
%%   \begin{enumerate}
%%   \item $c \vdash \eta_c = F_{ind}^{-1}[\id[F_0[c]]]: H[c,F_0[c]]$
%%   \item Every $h : H[c,d]$ factorizes through $\eta$ because
%%     \begin{align*}
%%       F_{ind}[h] \circ F_{ind}^{-1}[\id] &= F_{ind}^{-1}[F_{ind}[h] \circ \id] \tag{naturality}\\
%%       &= F_{ind}^{-1}[F_{ind}[h]] \tag{unit}\\
%%       &= h \tag{isomorphism}
%%     \end{align*}
%%     This factorization is unique because if $g \circ F^{-1}[\id] = h$,
%%     then
%%     \begin{align*}
%%       F_{ind}[h] &= F_{ind}[g \circ F^{-1}[\id]] \tag{congruence}\\
%%       &= g \circ F_{ind}[F^{-1}[\id]] \tag{naturality}\\
%%       &= g \circ \id \tag{iso}\\
%%       &= g \tag{unit}
%%     \end{align*}
%%   \end{enumerate}
%% \end{construction}

%% But the utility of the universal morphism definition is that the
%% universal morphism extracted from a left adjoint is enough to uniquely
%% reconstruct the rest of the structure, which includes (1)
%% functoriality of $F$ and (2) naturality of $F_{ind}$.
%% \begin{construction}
%%   Let $(F_0,\eta,F_{ind}[h])$ be a left universal $H$-morphism.
%%   We extend this to a left adjoint as follows.
%%   \begin{enumerate}
%%   \item First, $F$ acts on morphisms by
%%     \begin{mathpar}
%%       \inferrule
%%       {
%%         \inferrule
%%          {f : C[c,c'] \and \eta_{c'} : H[c',F_0[c']]}
%%          {f : C[c,c'] \vdash \eta_{c'} \circ f : H[c,F_0[c']]}
%%       }
%%       {f : C[c,c'] \vdash F_{ind}[\eta_{c'} \circ f] : D[F_0[c],F_0[c']]}
%%     \end{mathpar}
%%     This preserves identity
%%     \begin{mathpar}
%%       \inferrule
%%       {
%%         \inferrule
%%         {\id \circ \eta = \eta}
%%         {\id = F_{ind}[\eta]}
%%       }
%%       {\id = F_{ind}[\eta \circ \id]}
%%     \end{mathpar}
%%     And composition
%%     \begin{mathpar}
%%       \inferrule
%%       {F_{ind}[\eta \circ f] \circ F_{ind}[\eta \circ g] \circ \eta = \eta \circ f \circ g}
%%       {F_{ind}[\eta \circ f] \circ F_{ind}[\eta \circ g] = F_{ind}[\eta \circ f \circ g]}
%%     \end{mathpar}
%%     Which follows by a couple uses of the factorization:
%%     \begin{align*}
%%       F_{ind}[\eta \circ f] \circ F_{ind}[\eta \circ g] \circ \eta
%%       &= F_{ind}[\eta \circ f] \circ \eta \circ g \\
%%       &= \eta \circ f \circ g
%%     \end{align*}
%%   \item Next, we show that $h : H[c,d] \vdash F_{ind}[h] :
%%     D[F_0[c],d]$ is a natural transformation. As a first step, we show
%%     that $\eta_c : H[c,F_0[c]]$ is natural in $c$ with the action of
%%     $F$ on morphisms we have just defined.
%%     \[ \eta \circ f = F[f] \circ \eta \]
%%     expanding the definition, this is precisely the factorization
%%     property:
%%     \[ \eta \circ f = F_{ind}[\eta \circ f] \circ \eta \]
    
%%     Next, to show that $F_{ind}$ is natural, we need to show that
%%     (when it is well-defined):
%%     \[F_{ind}[g \circ h \circ f] = g \circ F_{ind}[h] \circ F[f] \]
%%     By the factorization property, the following sufficeth:
%%     \begin{align*}
%%      g \circ F_{ind}[h] \circ F[f] \circ \eta
%%      &= g \circ F_{ind}[h] \circ \eta \circ f \tag{naturality of $\eta$}\\
%%      &= g \circ h \circ f \tag{factorization}
%%     \end{align*}
%%   \item Finally, we define $F_{ind}^{-1}[g]$ as
%%     \[ g : D[F_0[c],d] \vdash F_{ind}^{-1}[g] = g \circ \eta : H[c,d] \]
%%     And we need to confirm that it is (1) a section of $F_{ind}$ (2) a
%%     retraction of $F_{ind}$ and (3) natural.
%%     \begin{enumerate}
%%     \item For retraction we need to show $h \vdash F_{ind}[h] \circ
%%       \eta = h$ which is precisely the factorization property.
%%     \item For section, we need to show $g \vdash F_{ind}[g \circ \eta]
%%       = g$. Immediately the uniquenes principle says its sufficient to
%%       show $g \circ \eta = g \circ \eta$.
%%     \item For naturality, we need to show
%%       \[ F_{ind}^{-1}[k \circ g \circ F[f]] = k \circ F_{ind}^{-1}[g] \circ f \]
%%       expanding the definition this is just:
%%       \[ k \circ g \circ F[f] \circ \eta = k \circ g \circ \eta \circ f \]
%%       which is just naturality of $\eta$.
%%     \end{enumerate}
%%   \end{enumerate}
%% \end{construction}

\subsection{Products and Units of Internal Categories}

%% While we have now proved a beautiful internal theorem in great
%% generality that has many useful instances in different models, we are
%% hard pressed to actually \emph{instantiate} the theorem in the syntax
%% itself because we can't really come up with any useful universal properties!
%% The reason is that Judgment Theory itself is too bare-bones.

%% For instance how would we say that a category $\cat C$ has a
%% \emph{terminal object}? Well, it would have to be in the first place a
%% \emph{single} object in $\cat C$, but so far our syntax only enables
%% us to talk about functions and functors. Well an object of $\cat C$ is
%% the same as a function from the unit sort, but so far our syntax only
%% has base sorts.

%% So we see that defining the terminal object, \emph{necessitates} some
%% notion of terminal object in our ``metatheory'': Judgment Theory.
%% This mysterious and pervasive aspect of higher category theory is
%% called the \emph{microcosm principle} because it says that in order to
%% discuss a property of a single object in a single category (the
%% microcosm), we need the same sort of object at the meta-level (the
%% macrocosm). The adage here is ``As above, so below''.

%% That might all sound very spiritual, but we'll see that it's actually
%% quite a nice heuristic for formalizing universal properties in
%% internal category theory.
%% %
%% In the spirit of this heuristic, instead of first giving a bunch of
%% new type constructions for Judgment Theory and then applying them to
%% define universal properties, we will consider them in pairs, first
%% above, and then below.

First we consider the terminal sort and then the idea of a terminal
object in an internal category.

Next, the terminal sort trivially has a category structure using the
trivial judgment as its hom set:

\begin{definition}[Terminal Category]
  The terminal category $\cat 1$ is defined as having
  \begin{enumerate}
  \item Its object sort as $1$
  \item Its hom set as $\lambda \alpha:1;\beta:1. 1 : \harr 1 1$
  \item Its identity arrow is given by $\lambda \alpha: 1. () : 1$
  \item Composition is defined by $\phi : 1, \psi : 1 \vdash () : 1$
  \item Associativity and unitality are trivial by \textsc{1-$\eta$}
  \end{enumerate}
\end{definition}

\begin{definition}[Terminal Object Specification]
  In judgment type theory with a terminal sort/judgment, we can define
  for any category $\cat C$ a profunctor $\alpha^o:\cat C;\beta : \cat
  1 \pipe 1$, which specifies a terminal object.
\end{definition}

Now let's instantiate the definition for the terminal object defined
as a negative type and as a representable and see how it reproduces
the standard type-theoretic and category-theoretic definitions.

A representation of $\alpha^o:\cat C; \beta : \cat 1 \pipe 1$ consists
of a functor $\beta : \cat 1 \vdash 1_{\cat C} : \cat C$, an introduction rule
\[ \phi : 1 \vdash I_{1}[\phi] : \alpha \to_{\cat C} 1_{\cat C}[\beta] \]
satisfying $I_{1}[\phi] \circ \gamma =I_{1}[\phi \circ \gamma]$ and an
inverse $\psi : \alpha \to_{\cat C} 1_{\cat C}[\beta] \vdash
I_{1}^{-1}[\psi] : 1$ which by \textsc{1-$\eta$} is equal to $()$ and
therefore every $t : \alpha \to_{\cat C} 1_{\cat C}[\beta]$ is equal
to $I_{1}[\phi]$. In light of this it would be appropriate to write
$I_1[\phi]$ as $()$ and then the left-homomorphism property is $()
\circ \gamma = ()$, which is precisely the definition of substitution.

This alternate syntax makes even more sense in light of the negative
type presentation. A negative type satisfying the terminal object
specification consists of a functor $1_{\cat C}$ as above and an
introduction rule as above and an elimination form $\vdash \epsilon :
1$, which is trivial.

First the product category
\begin{definition}[Product category]
  For any categories $\cat C_1$, $\cat C_2$ we can form the product
  category whose sort of objects is $\cat C_1 \times \cat C_2$, whose
  arrow judgment is $\alpha^o : \cat C_1 \times \cat C_2;\alpha \pipe
  (\pi_1 \alpha \to_{\cat C_1} \pi_1 \alpha) \times (\pi_2 \alpha
  \to_{\cat C_2} \pi_2 \alpha)$. Identity is given by $\alpha^o : \cat
  C_1 \times \cat C_2;\alpha\pipe\cdot \vdash (\id_{\pi_1 \alpha},
  \id_{\pi_2 \alpha}) : (\pi_1 \alpha \to_{\cat C_1} \pi_1 \alpha)
  \times (\pi_2 \alpha \to_{\cat C_2} \pi_2 \alpha)$ and composition
  by
  \[
  \phi %: (\pi_1 \alpha_0 \to_{\cat C_1} \pi_1 \alpha_1) \times (\pi_2 \alpha_0 \to_{\cat C_2} \pi_2 \alpha_1)
  ,
  \psi %: (\pi_1 \alpha_1 \to_{\cat C_1} \pi_1 \alpha_2) \times (\pi_2 \alpha_1 \to_{\cat C_2} \pi_2 \alpha_2)
  \vdash ((\pi_1 \phi) \circ (\pi_1 \psi), (\pi_2 \psi) \circ (\pi_2 \phi)) :
  (\pi_1 \alpha_0 \to_{\cat C_1} \pi_1 \alpha_2) \times (\pi_2 \alpha_0 \to_{\cat C_2} \pi_2 \alpha_2)  
  \]
  unitality, associativity follow from the same properties of $\cat C_1,\cat C_2$
\end{definition}

\begin{definition}[Product Functor Specification]
  For any category $\cat C$, we can specify the product functor by
  \begin{enumerate}
  \item The span is $\alpha^o:\cat C; \beta : \cat C \times \cat C \pipe (\alpha \to_{\cat C} \pi_1 \beta) \times (\alpha \to_{\cat C} \pi_2 \beta)$
  \item Left Composition is defined as
    \[ \phi, \psi \vdash (\pi_1 \phi \circ \psi, \pi_2 \phi \circ \psi)\]
  \item Left unitality:
    \begin{align*}
      (\pi_1 \phi \circ \id, \pi_2 \phi \circ \id)
      &= (\pi_1 \phi, \pi_2 \phi)\\
      &= \phi\tag{$\times-eta$}
    \end{align*}
  \item Left associativity:
    \begin{align*}
      (\phi \circ \psi) \circ \psi'
      &= (\pi_1 \phi \circ \psi, \pi_2 \phi \circ \psi) \circ \psi'\\
      &= (\pi_1 (\pi_1 \phi \circ \psi, \pi_2 \phi \circ \psi) \circ \psi',
      \pi_2 (\pi_1 \phi \circ \psi, \pi_2 \phi \circ \psi) \circ \psi')\\
      &= ((\pi_1 \phi \circ \psi) \circ \psi', (\pi_2 \phi \circ \psi) \circ \psi')\\
      &= (\pi_1 \phi \circ (\psi \circ \psi'), \pi_2 \phi \circ (\psi \circ \psi'))\\
      &= \phi \circ (\psi \circ \psi')
    \end{align*}
  \item Right composition is defined as
    \[ \phi, \theta \vdash (\pi_1 \theta \circ \pi_1 \phi, \pi_2 \theta \circ \pi_2 \phi) \]
  \item Right unitality:
    \begin{align*}
      (\pi_1 (\id,\id) \circ \pi_1 \phi, \pi_2 (\id,\id) \circ \pi_2 \phi)
      &= (\id \circ \pi_1 \phi, \id \circ \pi_2 \phi) \tag{$\times-\beta$}\\
      &= (\pi_1 \phi, \id \circ \pi_2 \phi)\\
      &= \phi \tag{$\times-\eta$}
    \end{align*}
  \item Right associativity
    \begin{align*}
      \theta' \circ (\theta \circ \phi)
      &= \theta' \circ (\pi_1 \theta \circ \pi_1 \phi, \pi_2 \theta \circ \pi_2 \phi)\\
      &= (\pi_1 \theta' \circ \pi_1(\pi_1 \theta \circ \pi_1 \phi, \pi_2 \theta \circ \pi_2 \phi),
      \pi_2 \theta' \circ \pi_2(\pi_1 \theta \circ \pi_1 \phi, \pi_2 \theta \circ \pi_2 \phi))\\
      &= (\pi_1 \theta' \circ (\pi_1 \theta \circ \pi_1 \phi), \pi_2 \theta' \circ (\pi_2 \theta \circ \pi_2 \phi))\\
      &= ((\pi_1 \theta' \circ \pi_1 \theta) \circ \pi_1 \phi, (\pi_2 \theta' \circ \pi_2 \theta) \circ \pi_2 \phi)\\
      &= (\pi_1(\theta' \circ \theta) \circ \pi_1 \phi, \pi_2(\theta' \circ \theta) \circ \pi_2 \phi)\\
      &= (\theta' \circ \theta) \circ \phi \\
    \end{align*}
  \end{enumerate}
\end{definition}

\subsection{Discrete and Codiscrete Categories}

If we have terminal judgments, then we can define the
\emph{codiscrete} category for any sort $\cat C$.
\begin{definition}
  The codiscrete category $\textrm{CoDisc} \cat C$ is defined as
  \begin{enumerate}
  \item Object sort is $\cat C$
  \item Morphisms judgment is $1$
  \item Identity and composition are given by the unique terms of
    those types, making associativity and unitality trivial to verify.
  \end{enumerate}
\end{definition}

If we have \emph{identity} objects, which we write as $\equiv_A$, then we
can define discrete categories.
\begin{definition}
  The discrete category $\textrm{Disc} \cat C$ is defined as
  \begin{enumerate}
  \item Object sort is $\cat C$
  \item Morphisms judgment is $\alpha;\alpha' \pipe \alpha \equiv \alpha'$
  \item Identity and composition are defined as in the section on
    synthetic category theory.
  \end{enumerate}
\end{definition}

\subsection{Internal Tensor, Cotensor}

We should be able to define internal \emph{tensor} using synthetic
tensor and coequalizers and also internal \emph{cotensor} using
synthetic cotensor and equalizers. These look exactly like the
set-theoretic definitions.
\begin{definition}
  Let $R$ be an internal profunctor from $C$ to $D$ and $S$ be an
  internal profunctor from $D$ to $E$.

  Then we can define the internal tensor product $R \odot S$, a
  profunctor from $C$ to $E$ as
  \begin{enumerate}
  \item The underlying span can be defined as a quotient of the synthetic tensor product
    \[ (R \odot S)[\alpha;\gamma] = \exists \beta:\cat D. R \odot S/Rcomp = Scomp \]

    Where R-comp and S-comp have as source the span
    \[ \exists \beta:\cat D.~ R[\alpha;\beta] \odot (\exists \beta':\cat D. Hom_C(\beta,\beta') \odot S[\beta';\gamma]) \]
    or the isomorphic
    \[ \exists \beta':\cat D.~ (\exists \beta:\cat D. R[\alpha;\beta] \odot Hom_C(\beta,\beta')) \odot S[\beta';\gamma] \]
    where R-comp composes the $C$-morphism with the $R$-morphism and
    S-comp the $S$-morphism.
  \item Composition is defined using composition in $R,S$.
  \end{enumerate}
\end{definition}
Note that $R$-comp and $S$-comp are each constructed using the 

\begin{definition}
  Let $R$ be an internal profunctor from $C$ to $D$ and $S$ be an
  internal profunctor from $C$ to $E$. Then the internal contravariant
  cotensor product $R \triangleright S$ is defined as
  \[ (R \triangleright S)[\beta;\gamma] = \{ p : \forall \alpha:\cat C.~ R[\alpha;\beta] \triangleright S[\alpha;\gamma] 
  \pipe R-comp = S-comp \}  \]
  R-comp and S-comp have as codomain the span
  \[ \forall \alpha:\cat C.~ R[\alpha;\beta] \triangleright \forall \alpha':\cat C.~ Hom_C(\alpha',\alpha) \triangleright S[\alpha';\gamma] \]
  or the isomorphic
  \[ \forall \alpha':\cat C.~ (\exists \alpha. Hom_C(\alpha',\alpha) \odot R[\alpha;\beta]) \triangleright S[\alpha';\gamma]
  \]
\end{definition}

We can also construct the \emph{free profunctor} from a span using tensor:

\begin{definition}
  Let $\cat C$ and $\cat D$ be internal categories and let $R[\alpha;\beta]$ be a
  span from $Ob(\cat C)$ to $Ob(\cat D)$. Then the free profunctor
  extending $R$ to be an internal profunctor from $\cat C$ to $\cat D$ is
  \[ F_{\cat C,\cat D}R[\alpha';\beta'] = \exists \alpha. Hom_C(\alpha',\alpha) \odot (\exists \beta. R[\alpha;\beta] \odot Hom_C(\beta,\beta')) \]
  or the isomorphic
  \[ F_{\cat C,\cat D}R[\alpha';\beta'] = \exists \beta. (\exists \alpha. Hom_C(\alpha',\alpha) \odot R[\alpha;\beta]) \odot Hom_C(\beta,\beta') \]
\end{definition}

And maybe we can even define the co-free profunctor from a span using cotensor?
\begin{definition}
  Let $\cat C$ and $\cat D$ be internal categories and let $R[\alpha;\beta]$ be a
  span from $Ob(\cat C)$ to $Ob(\cat D)$. Then the co-free profunctor
  extending $R$ to be an internal profunctor from $\cat C$ to $\cat D$ is
  \[ G_{\cat C,\cat D}R[\alpha';\beta'] =
  \forall \alpha. Hom_C(\alpha,\alpha') \triangleright (\forall \beta. R[\alpha;\beta] \triangleleft Hom_D(\beta',\beta)) \]
  or the isomorphic
  \[ G_{\cat C,\cat D}R[\alpha';\beta'] =
  \forall \beta. 
  (\forall \alpha. Hom_C(\alpha,\alpha') \triangleright R[\alpha;\beta]) \triangleleft Hom_D(\beta',\beta)) \]
\end{definition}

\subsection{Duality Principle}

One common source of category theoretic arguments is the duality
principle: since the opposite of any category is also a category, any
statement that quantifies over all categories is equally true of their
opposites. Furthermore in a formal language of categories, any
statement can be dualized to a statement where domains and codomains
are swapped and composition is flipped.
%
A similar duality principle holds in SCT.
%
The basic idea is to simply reverse the orientation of all \emph{set}
expressions by swapping which variable is covariant and which is
contravariant.
%
This has the effect of reversing the order of the context $\Phi$.

We can define this as a simple translation $\cdot^* : SCT \to SCT$
defined on syntactic forms of SCT.

\begin{figure}
    \begin{align*}
      (\harr P Q)^* &= \harr {P^*} {Q^*}\\
      (\alpha:\cat C)^* &= \alpha: \cat C^*\\
      (\alpha:\cat C; \beta:\cat D)^* &= \beta:\cat D^*; \alpha:\cat C^*\\
      (\Phi,x:R,\Psi)^* &= \Psi^*,x:R^*, \Phi^*\\
      (\homrallXYtoZ \alpha P Q)^* &= \homlallXYtoZ \alpha {P^*} {Q^*}\\
      (\homlallXYtoZ \alpha P Q)^* &= \homrallXYtoZ \alpha {P^*} {Q^*}\\
      (\homrlambdaXatYdotZ x \alpha s)^* &= \homllambdaXatYdotZ x \alpha s^*\\
      (\homrappXtoYatZ s t a)^* &= \homlappXtoYatZ {s^*} {t^*} {a^*}\\
      (\tensorexistsXwithYandZ \beta P Q)^* &=
      \tensorexistsXwithYandZ \beta {Q^*}{P^*}\\
      (\tensorintroatXwithYandZ b s t)^* &= \tensorintroatXwithYandZ {b^*} {t^*} {s^*}\\
      (\punitinXfromYtoZ {\cat C} {a_1}{a_2})^* &= \punitinXfromYtoZ {\cat C^*} {a_1^*}{a_2^*}
    \end{align*}
  
  \caption{Duality Involution}
\end{figure}

TODO: example of applying the principle in our calculus.

\section{Semantics}

In this section we establish our desired semantic soundness and
completeness theorems for models in virtual equipments with the
corresponding universal properties.
%
The first component of this is \emph{soundness}: every category,
object, set and transformation denote an object, vertical arrow,
horizontal arrow and 2-cell, respectively in any representable virtual
equipment, such that all equations in the theory are satisfied.
%
We combine this with a set theoretic semantics for types $A$ and
contexts $\Gamma$ to get a simple semantics of our logic over a
virtual equipment.
%
The second component of this is the \emph{completeness} theorem:
informally, the theory contains every category, object, set and
transformation that \emph{must} exist in any virtual equipment, and
further equations between syntactic forms hold if and only if they
hold in every model.
%
While sounding more ambitious, the completeness theorem is
typically simpler as it just amounts to showing that the syntax itself
forms a virtual equipment.

\subsection{Notion of Model}

\begin{definition}[Virtual Double Category]
  A virtual double category $V$ consists of
  \begin{enumerate}
    \item A set $V_0$ of objects.
    \item For each $\cat C, \cat D \in V_0$, a set $V_v(\cat C, \cat
      D)$ of vertical arrows, with composition and unit operations
      satisfying the category axioms.
    \item A set $V_h$ of horizontal arrows, with $d^+,d^- : V_h \to
      V_0$.
    \item For each composable string $\Phi$ of horizontal arrows
      starting at $\cat C$ and ending at $\cat D$, and horizontal
      arrow $R$, a set $V_2(\Phi; R)$ of 2-cells, with projection
      functions $d^{\pm} : V_2(\Phi;R) \to V_v(d^\pm \Phi, d^\pm R)$.

    \item For each $R \in V_h$, an identity arrow $\id_R \in
      V_2(d^-R,R,d^+ R; R)$ with $d^\pm \id_R = \id_{d^\pm R}$.

    \item For any $2$-cell $\alpha \in V_2(\Phi;R)$ and composable
      string $\phi$ of 2-cells into $\Phi$, a composite $\alpha \circ
      \phi$.

    \item Composition should be unital in that:
      \[ \id_R \circ [d^-\alpha, \alpha, d^+\alpha] = \alpha  \]
      and
      \[ \alpha \circ [\vec \id] = \alpha\]
    \item And associative in that
      \[ (\alpha \circ [\beta_1,\ldots]) \circ [\gamma_{11},\ldots,\gamma_{21},\ldots,] = \alpha \circ [(\beta_1 \circ \gamma_{11},\ldots), (\beta_2 \circ \gamma_{21},\ldots)]\]
  \end{enumerate}
\end{definition}

\begin{definition}[Restriction]
  Let $R \in V_h$ and $f \in V_v(\cat C, d^-R)$ and $g \in V_v(\cat D,
  d^+R)$, then a restriction of $R$ along $f,g$ is a horizontal arrow
  $R(f,g) \in V_h$ with $d^-R(f,g)= \cat C$ and $d^+R(f,g)= \cat D$
  with a (op?)-cartesian 2-cell $\alpha \in V_2(R(f,g),R)$ with
  $d^-\alpha = f$ and $d^+\alpha = g$.
\end{definition}

\begin{definition}[Unit]
  A unit $U_{\cat C}\in V_h$ with $d^\pm U_{\cat C} = \cat C$ and an op-cartesian 2-cell
  \[ r : V_h(\cat C, U_{\cat C}) \]
  with $d^\pm r = \id_{\cat C}$.
\end{definition}

\begin{definition}[Tensor]
  If $R,Q \in V_h$ with $d^+R = d^-Q$, a tensor is a horizontal arrow
  $R \odot Q \in V_h$ with $d^+ (R\odot Q) = d^+ R$ and $d^-(R \odot
  Q) = d^- Q$ with an op-cartesian 2-cell
  \[ p : V_h(R,Q;R \odot Q) \]
\end{definition}

\begin{definition}[Hom]
  
\end{definition}

\subsection{Category of Signatures and Models}

Our goal is to establish that the syntax of SCT (with whichever type
constructors we choose) presents a free virtual double category (with
the appropriate universal properties). To start we define appropriate
categories of signatures and double categories.

\begin{definition}
  Define the category Sig to have as objects well-formed contexts
  $\Gamma$ and morphisms $\gamma : \Delta \to \Gamma$ morphisms of
  contexts, with the composition as substitution and identity as a
  context morphism of all variable references.
\end{definition}

Second, there is a category of virtual double categories and functors.

\begin{definition}
  A morphism of virtual equipments with blah is a functor of virtual
  double categories that strictly preserves the universal properties.
\end{definition}

Next, we define a profunctor between the category of signatures and
the category of virtual equipments.

\begin{definition}
  Given a well-formed signature $\Gamma$ and a virtual equipment
  $\mathcal V$, we mutually define the following notions:

  \begin{itemize}
  \item The set $\sem{\Gamma}_{\mathcal V}$
  \item For any $\Gamma \vdash A$ and $\gamma \in \sem{\Gamma}_{\mathcal V}$, a set $\sem{A}_{\mathcal V} \gamma$
  \item For any $\Gamma \vdash M : A$ and $\gamma \in \sem{\Gamma}_{\mathcal V}$, an element $\sem{M}_{\mathcal V} \in \sem{A}_{\mathcal V} \gamma$
  \item categories...
  \item functors...
  \item profunctors...
  \item trans contexts,
  \item transformations
  \end{itemize}
  
  An interpretation of a signature $\Gamma$ in a virtual equipment
  $\mathcal V$, $\textrm{Interp}(\Gamma,\mathcal V)$ is an element of
  the set $\sem{\Gamma}_{\mathcal V}$ where we mutually define the set

  \begin{align*}
    \sem{\cdot}_{\mathcal V} &= 1\\
    \sem{\Gamma, x:A}_{\mathcal V} &= \sum_{\gamma \in \sem{\Gamma}_{\mathcal V}} \sem{A}_{\mathcal V}\gamma\\
  \end{align*}
  
  with an 
  \begin{align*}
    \sem{\Gamma, x:A}_{\mathcal V} &= \sem{\Gamma}_{\mathcal V} \times \sem{A}_{\mathcal V}\\
    \sem{\Gamma}_{\mathcal V} &= 1\\\\
    \sem{\Sort}_{\mathcal V} &= \mathcal V_0\\
    \sem{\varr{\cat C}{\cat D}}_{\mathcal V} &= \mathcal V_v(\sem{\cat C})
  \end{align*}

  
\end{definition}

\subsection{Completeness}

\begin{definition}
  For any $\Gamma \isadtctx$, we construct a virtual double category
  $V^\Gamma$.
  \begin{enumerate}
  \item $V^\Gamma_v$ is the set of terms $\Gamma \vdash \cat C \isaCat$.
  \item $V^\Gamma_v(\cat C, \cat D)$ is the set of ``objects'' $\Gamma
    \pipe \alpha:\cat C \vdash a : \cat D$. The category structure is
    given by substitution, with the variable $\alpha \vdash \alpha$
    the identity.
  \item $V_h$ is the set of triples $(\cat C, \cat D, \Gamma \pipe \alpha:\cat C;\beta:\cat D \vdash R \isaSet)$
    where $d^\pm$ are the projections
  \item $V^\Gamma_h(\vec R; S)$ is the set of triples $(a,b,\Gamma
    \pipe \Phi \vdash t : S[a;b])$ where $\Phi$ is the context
    corresponding to $\vec R$ and $d^\pm$ are the obvious projections.

  \item The identity morphisms are $(\cat C, \cat D, R)$ is $\Gamma\pipe\alpha:\cat C, x:R\beta:\cat D \vdash x:R$
  \item Composition $(a,b,t) \circ [(a_1,),\ldots]$ is given by
    substitution in each component: $(a[a_1],b[a_n], t[\phi])$.
  \item Associativity and unit rules for composition follow from the
    definition of substitution by a straightforward induction.
  \end{enumerate}
\end{definition}

\subsection{Soundness}

Let $$


%% \section{Conservativity of Homs, Tensors}

%% The soundness theorem is not entirely sufficient for our goals,
%% however.
%% %
%% In particular, we are interested in applying the internal category
%% constructions to models that only have units, and do not necessarily
%% admit left/right homs or tensor products of horizontal arrows.
%% %
%% To do so, we now prove a \emph{conservativity} result for the calculus
%% with homs and tensors over a simpler calculus whose only profunctor
%% constructor is the unit.
%% %
%% We must take care in defining the more restricted calculus, however.
%% %
%% If we simply remove homs and tensors from the calculus, then there is
%% no way to internalize the transformations as types, and so no way to
%% actually define internal categories!
%% %
%% For example, if we remove the hom sets, there is no way to abstract
%% over the composition transformation:
%% \[\alpha_1:\cat C, f : \harrapp {\cat C_1}{\alpha_1}{\alpha_2}, \alpha_2:\cat C, g : \harrapp{\cat C_1}{\alpha_2}{\alpha_3}, \alpha_3:\cat C \vdash \textrm{compose} : \harrapp{\cat C_1}{\alpha_1}{\alpha_3}
%% \]
%% %
%% So in addition to removing the hom sets and tensors, we add a
%% ``jumbo'' combined form of the hom set and end type, described in
%% \cref{fig:jumbo}.
%% %
%% This type directly internalizes a transformation judgment of arbitrary
%% large context, as opposed to the end type which internalized only
%% transformations from an ``empty'' context.
%% %


%% \begin{figure}
%%   \begin{mathpar}
%%     \inferrule*[right=JumboHomForm]
%%     {\Gamma \vdash \Phi \isavectx \and \Gamma \pipe \underline \Phi \vdash R \isaSet}
%%     {\Gamma \vdash \Phi \triangleright R}

%%     \inferrule*[right=JumboHomI]
%%     {\Gamma \pipe \Phi \vdash s : R}
%%     {\Gamma \vdash \lambda \Phi. s : \Phi \triangleright R}

%%     \inferrule*[right=JumboHomE]
%%     {\Gamma \vdash M : \Phi \triangleright R\and
%%       \Gamma \vdash \phi : \Psi \to \Phi}
%%     {\Gamma \pipe \Psi \vdash M \triangleright \phi : R[\underline \phi]}

%%     \inferrule*[right=Jumbo$\beta$]
%%     {}
%%     {\lambda \Phi. M \triangleright (\Phi) = M}

%%     \inferrule*[right=Jumbo$\eta$]
%%     {\Gamma \vdash M : \Phi \triangleright R}
%%     {M = \lambda \Phi. M \triangleright \Phi}
%%   \end{mathpar}
%%   \caption{Jumbo End/Homs}
%%   \label{fig:jumbo}
%% \end{figure}

%% %
%% To demonstrate this, we prove a \emph{conservativity} result for the
%% calculus with homs and tensors over a calculus containing only units,
%% and a ``jumbo'' internalization

A model of our theory is given by a virtual double category with
restrictions, units, tensors and homs. We informally describe the
semantics of each judgment now.

A context $\Gamma \isadtctx$ denotes a set.
%
A set with empty boundary $\Gamma \pipe \cdot \vdash S \isaSet$ denotes a
set for every $\gamma:\Gamma$.
%
The type $\Gamma \pipe \cdot \vdash \Sort \isaSet$ denotes the set of
objects of the virtual equipment (for any $\gamma:\Gamma$).
%
A term $\Gamma \pipe \alpha:C \vdash A : D$ denotes a vertical arrow
from $C \gamma$ to $D \gamma$ for every $\gamma:\Gamma$.
%
A term $\Gamma \pipe \alpha:C; \beta : D \vdash R \isaSet$ denotes for
each $\gamma:\Gamma$ a horizontal arrow from $C\gamma$ to $D\gamma$.
%
A term $\Gamma \pipe \alpha:C \vdash R \isaSet$ denotes for each
$\gamma$, a horizontal arrow from $C\gamma$ to itself.
%
A non-$\cdot$ context $\Gamma \vdash \Phi\isavectx$ denotes a
``composable string'' of horizontal arrows from $d^-\Phi \gamma$ to
$d^+ \Phi \gamma$.
%
A term $\Gamma \pipe \cdot \vdash M : S$ denotes a function from
$\gamma :\Gamma$ to $S \gamma$ (section of the bundle blah blah).
%
A term $\Gamma \pipe \Phi \vdash M : R$ for non-empty $\Phi$ denotes a
``globular'' 2-cell in the virtual equipment for every
$\gamma:\Gamma$.

Next, substitutions model the forms of composition in the system.
Substitution $A[B/\alpha]$ denotes composition of vertical arrows
(with variable being the identity). Substitution $R[A/\alpha;B/\beta]$
denotes restriction of a horizontal arrow $R$ along the vertical
arrows $A,B$. And substitution $M[\phi]$ denotes composition of
2-cells, where some restrictions are of course involved, and again a
variable is the identity (globular) 2-cell.

Next, we would also like to prove some kind of \emph{completeness}
theorem for our calculus, at least the part relevant to virtual
equipments. One way to do this would be to prove a kind of
parameterized completeness theorem, where we treat the $\Gamma$ like a
signature, and show that the syntax is complete for every fixed
$\Gamma$. This amounts to proving that for any fixed $\Gamma$, the
judgments above present a virtual equipment with a trivial model
(i.e., each judgment is interpreted as itself). This shows that the
syntax is closed under all constructions that are present in every
model.

Here is the outline of how we expect we should be able to prove
this. For any fixed $\Gamma$, the objects of our double category are
the terms $\Gamma \vdash C : \Sort$. A vertical arrow from $C$ to $D$
is a term $\Gamma\pipe \alpha:C \vdash A : D$. A horizontal arrow from
$C$ to $D$ is a term $\Gamma \pipe \alpha:C;\beta:D \vdash R
\isaSet$. From any composable string of such $R$s we can construct a
context $\Gamma \vdash \Phi \isavectx$ and then a 2-cell from $\Phi$ to
$R$ along $A,B$ is given by a term $\Gamma \pipe \Phi \vdash
R[A/\alpha;B/\beta]$. Compositions and restriction are modeled by the
appropriate notion of substitution and units/tensors/homs are modeled
by the corresponding type formers. There are a lot of details to be
verified, especially the equations.


\section{Discussion}

\paragraph{Directed Type Theory}

Cartesian Bicategories.

Directed Type theory

Fibrational approaches to logical relations

Double-cat-like syntax: Gradual Type Theory, System P of parametricity, cubical stuff?

String Diagrams?

\paragraph{Semantic extensions}

\paragraph{Extensions}

While we have demonstrated that many common constructions in category
theory can be expressed using the simple combination of homs, tensors
and units, there are several that are out of reach for SCT due to its
restrictive judgmental structure.

First, while we can add products to the calculus, adding in a
\emph{functor category} ${\cat C}^{\cat D}$ would necessitate adding a
syntax for functors of multiple arguments, to allow for the lambda rule:
\[ \inferrule {\alpha:\cat C, \beta:\cat D \vdash e : \cat E} {\alpha:\cat C \vdash \lambda \beta:\cat D. e : \cat E} \]
This would then require also some extension to profunctors for
allowing 0 or more covariant/contravariant dependencies as well.
%
In addition, we could consider extending our semantics with a notion
of \emph{opposite category}, with the property that a covariant
dependence of a profunctor on a variable in a category should be
equivalent to a contravariant dependence on a variable in the opposite
category.
%
This could possibly be done by introducing a modal distinction
$\alpha:^{+}\cat C$ for covariant dependence and $\alpha:^{-}\cat C$
for contravariant dependence, with the opposite category being a
modality that internalizes flipping the variance as a type
constructor.
%
With these two extensions, possibly, the category contexts for
functors and profunctors could be unified as one notion of a sequence
of object variables with variances.
%
With this unification, it would then be simpler to incorporate a third
extension: a universe object, corresponding to an object
$\textrm{Set}$, which would allow us to internalize the profunctor
judgment in the functor judgment, modeling that in classical category
theory, profunctors are the same as certain functors to the category
of Sets.
%
While all of these extensions seem plausible, they all move away to
the simple ordered type theory structure that allowed us to define
units, tensor and hom profunctors so easily. It is unclear how to
define these types in this extended structure. As a simple example, if
profunctors have many variables of mixed variance, rather than exactly
one of each variance, it is unclear what the appropriate
generalization of the transformation context $\Phi$ should be.

An orthogonal extension to these would be to consider adding a
\emph{discretization} modality $|\cat C|$ that gives the discrete
category on the underlying set of objects of $\cat C$. This would give
us the ability to formalize concepts like ``function on objects'' or
``unnatural transformation'' without the need to define internal
categories. This could be modeled by adding a modal distinction
$\alpha:^{\#}\cat C$ for ``crisp'' dependence on an object of $\cat
C$, i.e., dependence merely on the object of $\cat C$ with no
functoriality/naturality conditions. This would then be internalized
by the discretization modality.

   
%% \section{Typesetting instructions -- Summary}
%% \label{sec:typesetting-summary}

%% LIPIcs is a series of open access high-quality conference proceedings across all fields in informatics established in cooperation with Schloss Dagstuhl. 
%% In order to do justice to the high scientific quality of the conferences that publish their proceedings in the LIPIcs series, which is ensured by the thorough review process of the respective events, we believe that LIPIcs proceedings must have an attractive and consistent layout matching the standard of the series.
%% Moreover, the quality of the metadata, the typesetting and the layout must also meet the requirements of other external parties such as indexing service, DOI registry, funding agencies, among others. The guidelines contained in this document serve as the baseline for the authors, editors, and the publisher to create documents that meet as many different requirements as possible. 

%% Please comply with the following instructions when preparing your article for a LIPIcs proceedings volume. 
%% \paragraph*{Minimum requirements}

%% \begin{itemize}
%% \item Use pdflatex and an up-to-date \LaTeX{} system.
%% \item Use further \LaTeX{} packages and custom made macros carefully and only if required.
%% \item Use the provided sectioning macros: \verb+\section+, \verb+\subsection+, \verb+\subsubsection+, \linebreak \verb+\paragraph+, \verb+\paragraph*+, and \verb+\subparagraph*+.
%% \item Provide suitable graphics of at least 300dpi (preferably in PDF format).
%% \item Use BibTeX and keep the standard style (\verb+plainurl+) for the bibliography.
%% \item Please try to keep the warnings log as small as possible. Avoid overfull \verb+\hboxes+ and any kind of warnings/errors with the referenced BibTeX entries.
%% \item Use a spellchecker to correct typos.
%% \end{itemize}

%% \paragraph*{Mandatory metadata macros}
%% Please set the values of the metadata macros carefully since the information parsed from these macros will be passed to publication servers, catalogues and search engines.
%% Avoid placing macros inside the metadata macros. The following metadata macros/environments are mandatory:
%% \begin{itemize}
%% \item \verb+\title+ and, in case of long titles, \verb+\titlerunning+.
%% \item \verb+\author+, one for each author, even if two or more authors have the same affiliation.
%% \item \verb+\authorrunning+ and \verb+\Copyright+ (concatenated author names)\\
%% The \verb+\author+ macros and the \verb+\Copyright+ macro should contain full author names (especially with regard to the first name), while \verb+\authorrunning+ should contain abbreviated first names.
%% \item \verb+\ccsdesc+ (ACM classification, see \url{https://www.acm.org/publications/class-2012}).
%% \item \verb+\keywords+ (a comma-separated list of keywords).
%% \item \verb+\relatedversion+ (if there is a related version, typically the ``full version''); please make sure to provide a persistent URL, e.\,g., at arXiv.
%% \item \verb+\begin{abstract}...\end{abstract}+ .
%% \end{itemize}

%% \paragraph*{Please do not \ldots} %Do not override the \texttt{\seriesstyle}-defaults}
%% Generally speaking, please do not override the \texttt{lipics-v2021}-style defaults. To be more specific, a short checklist also used by Dagstuhl Publishing during the final typesetting is given below.
%% In case of \textbf{non-compliance} with these rules Dagstuhl Publishing will remove the corresponding parts of \LaTeX{} code and \textbf{replace it with the \texttt{lipics-v2021} defaults}. In serious cases, we may reject the LaTeX-source and expect the corresponding author to revise the relevant parts.
%% \begin{itemize}
%% \item Do not use a different main font. (For example, the \texttt{times} package is forbidden.)
%% \item Do not alter the spacing of the \texttt{lipics-v2021.cls} style file.
%% \item Do not use \verb+enumitem+ and \verb+paralist+. (The \texttt{enumerate} package is preloaded, so you can use
%%  \verb+\begin{enumerate}[(a)]+ or the like.)
%% \item Do not use ``self-made'' sectioning commands (e.\,g., \verb+\noindent{\bf My+ \verb+Paragraph}+).
%% \item Do not hide large text blocks using comments or \verb+\iffalse+ $\ldots$ \verb+\fi+ constructions. 
%% \item Do not use conditional structures to include/exclude content. Instead, please provide only the content that should be published -- in one file -- and nothing else.
%% \item Do not wrap figures and tables with text. In particular, the package \texttt{wrapfig} is not supported.
%% \item Do not change the bibliography style. In particular, do not use author-year citations. (The
%% \texttt{natbib} package is not supported.)
%% \end{itemize}

%% \enlargethispage{\baselineskip}

%% This is only a summary containing the most relevant details. Please read the complete document ``LIPIcs: Instructions for Authors and the \texttt{lipics-v2021} Class'' for all details and don't hesitate to contact Dagstuhl Publishing (\url{mailto:publishing@dagstuhl.de}) in case of questions or comments:
%% \href{http://drops.dagstuhl.de/styles/lipics-v2021/lipics-v2021-authors/lipics-v2021-authors-guidelines.pdf}{\texttt{http://drops.dagstuhl.de/styles/lipics-v2021/\newline lipics-v2021-authors/lipics-v2021-authors-guidelines.pdf}}

%% \section{Lorem ipsum dolor sit amet}

%% Lorem ipsum dolor sit amet, consectetur adipiscing elit \cite{DBLP:journals/cacm/Knuth74}. Praesent convallis orci arcu, eu mollis dolor. Aliquam eleifend suscipit lacinia. Maecenas quam mi, porta ut lacinia sed, convallis ac dui. Lorem ipsum dolor sit amet, consectetur adipiscing elit. Suspendisse potenti. Donec eget odio et magna ullamcorper vehicula ut vitae libero. Maecenas lectus nulla, auctor nec varius ac, ultricies et turpis. Pellentesque id ante erat. In hac habitasse platea dictumst. Curabitur a scelerisque odio. Pellentesque elit risus, posuere quis elementum at, pellentesque ut diam. Quisque aliquam libero id mi imperdiet quis convallis turpis eleifend. 

%% \begin{lemma}[Lorem ipsum]
%% \label{lemma:lorem}
%% Vestibulum sodales dolor et dui cursus iaculis. Nullam ullamcorper purus vel turpis lobortis eu tempus lorem semper. Proin facilisis gravida rutrum. Etiam sed sollicitudin lorem. Proin pellentesque risus at elit hendrerit pharetra. Integer at turpis varius libero rhoncus fermentum vitae vitae metus.
%% \end{lemma}

%% \begin{proof}
%% Cras purus lorem, pulvinar et fermentum sagittis, suscipit quis magna.


%% \proofsubparagraph*{Just some paragraph within the proof.}
%% Nam liber tempor cum soluta nobis eleifend option congue nihil imperdiet doming id quod mazim placerat facer possim assum. Lorem ipsum dolor sit amet, consectetuer adipiscing elit, sed diam nonummy nibh euismod tincidunt ut laoreet dolore magna aliquam erat volutpat.
%% \begin{claim}
%% content...
%% \end{claim}
%% \begin{claimproof}
%% content...
%%     \begin{enumerate}
%%         \item abc abc abc \claimqedhere{}
%%     \end{enumerate}
%% \end{claimproof}

%% \end{proof}

%% \begin{corollary}[Curabitur pulvinar, \cite{DBLP:books/mk/GrayR93}]
%% \label{lemma:curabitur}
%% Nam liber tempor cum soluta nobis eleifend option congue nihil imperdiet doming id quod mazim placerat facer possim assum. Lorem ipsum dolor sit amet, consectetuer adipiscing elit, sed diam nonummy nibh euismod tincidunt ut laoreet dolore magna aliquam erat volutpat.
%% \end{corollary}

%% \begin{proposition}\label{prop1}
%% This is a proposition
%% \end{proposition}

%% \autoref{prop1} and \cref{prop1} \ldots

%% \subsection{Curabitur dictum felis id sapien}

%% Curabitur dictum \cref{lemma:curabitur} felis id sapien \autoref{lemma:curabitur} mollis ut venenatis tortor feugiat. Curabitur sed velit diam. Integer aliquam, nunc ac egestas lacinia, nibh est vehicula nibh, ac auctor velit tellus non arcu. Vestibulum lacinia ipsum vitae nisi ultrices eget gravida turpis laoreet. Duis rutrum dapibus ornare. Nulla vehicula vulputate iaculis. Proin a consequat neque. Donec ut rutrum urna. Morbi scelerisque turpis sed elit sagittis eu scelerisque quam condimentum. Pellentesque habitant morbi tristique senectus et netus et malesuada fames ac turpis egestas. Aenean nec faucibus leo. Cras ut nisl odio, non tincidunt lorem. Integer purus ligula, venenatis et convallis lacinia, scelerisque at erat. Fusce risus libero, convallis at fermentum in, dignissim sed sem. Ut dapibus orci vitae nisl viverra nec adipiscing tortor condimentum \cite{DBLP:journals/cacm/Dijkstra68a}. Donec non suscipit lorem. Nam sit amet enim vitae nisl accumsan pretium. 

%% \begin{lstlisting}[caption={Useless code.},label=list:8-6,captionpos=t,float,abovecaptionskip=-\medskipamount]
%% for i:=maxint to 0 do 
%% begin 
%%     j:=square(root(i));
%% end;
%% \end{lstlisting}

%% \subsection{Proin ac fermentum augue}

%% Proin ac fermentum augue. Nullam bibendum enim sollicitudin tellus egestas lacinia euismod orci mollis. Nulla facilisi. Vivamus volutpat venenatis sapien, vitae feugiat arcu fringilla ac. Mauris sapien tortor, sagittis eget auctor at, vulputate pharetra magna. Sed congue, dui nec vulputate convallis, sem nunc adipiscing dui, vel venenatis mauris sem in dui. Praesent a pretium quam. Mauris non mauris sit amet eros rutrum aliquam id ut sapien. Nulla aliquet fringilla sagittis. Pellentesque eu metus posuere nunc tincidunt dignissim in tempor dolor. Nulla cursus aliquet enim. Cras sapien risus, accumsan eu cursus ut, commodo vel velit. Praesent aliquet consectetur ligula, vitae iaculis ligula interdum vel. Integer faucibus faucibus felis. 

%% \begin{itemize}
%% \item Ut vitae diam augue. 
%% \item Integer lacus ante, pellentesque sed sollicitudin et, pulvinar adipiscing sem. 
%% \item Maecenas facilisis, leo quis tincidunt egestas, magna ipsum condimentum orci, vitae facilisis nibh turpis et elit. 
%% \end{itemize}

%% \begin{remark}
%% content...
%% \end{remark}

%% \section{Pellentesque quis tortor}

%% Nec urna malesuada sollicitudin. Nulla facilisi. Vivamus aliquam tempus ligula eget ornare. Praesent eget magna ut turpis mattis cursus. Aliquam vel condimentum orci. Nunc congue, libero in gravida convallis \cite{DBLP:conf/focs/HopcroftPV75}, orci nibh sodales quam, id egestas felis mi nec nisi. Suspendisse tincidunt, est ac vestibulum posuere, justo odio bibendum urna, rutrum bibendum dolor sem nec tellus. 

%% \begin{lemma} [Quisque blandit tempus nunc]
%% Sed interdum nisl pretium non. Mauris sodales consequat risus vel consectetur. Aliquam erat volutpat. Nunc sed sapien ligula. Proin faucibus sapien luctus nisl feugiat convallis faucibus elit cursus. Nunc vestibulum nunc ac massa pretium pharetra. Nulla facilisis turpis id augue venenatis blandit. Cum sociis natoque penatibus et magnis dis parturient montes, nascetur ridiculus mus.
%% \end{lemma}

%% Fusce eu leo nisi. Cras eget orci neque, eleifend dapibus felis. Duis et leo dui. Nam vulputate, velit et laoreet porttitor, quam arcu facilisis dui, sed malesuada risus massa sit amet neque.

%% \section{Morbi eros magna}

%% Morbi eros magna, vestibulum non posuere non, porta eu quam. Maecenas vitae orci risus, eget imperdiet mauris. Donec massa mauris, pellentesque vel lobortis eu, molestie ac turpis. Sed condimentum convallis dolor, a dignissim est ultrices eu. Donec consectetur volutpat eros, et ornare dui ultricies id. Vivamus eu augue eget dolor euismod ultrices et sit amet nisi. Vivamus malesuada leo ac leo ullamcorper tempor. Donec justo mi, tempor vitae aliquet non, faucibus eu lacus. Donec dictum gravida neque, non porta turpis imperdiet eget. Curabitur quis euismod ligula. 


%% %%
%% %% Bibliography
%% %%

%% %% Please use bibtex, 

\bibliography{lipics-v2021-sample-article}

\appendix

TODO: complete substitution
\begin{figure}
  \begin{mathpar}
      (\homrallXYtoZ \alpha R P)[\xi] = \homrallXYtoZ \alpha {R[d^+\xi;\alpha/\alpha]} {P[\alpha/\alpha;d^-\xi]}

     (\homrlambdaXatYdotZ {x} {\alpha} M)[\phi]
      = \homrlambdaXatYdotZ {x} {\alpha} M[\phi,x/x,\gamma/\gamma]

     (\homllambdaXatYdotZ {x} {\alpha} M)[\phi]
      = \homllambdaXatYdotZ {x} {\alpha} M[\phi,x/x,\gamma/\gamma]


  \end{mathpar}
  \caption{Substitution}
  
\end{figure}

\begin{figure}
  \begin{mathpar}
    \inferrule
    {}
    {\Gamma\pipe \cdot \vdash \Sort \isaSet}

    \inferrule
    {x:S \in \Gamma}
    {\Gamma \pipe \cdot \vdash x : S}

    \inferrule
    {\Gamma\pipe\cdot \vdash C : \Sort}
    {\Gamma \pipe \alpha:\cat C \vdash \alpha : \cat C}

    \inferrule
    {\Gamma \pipe \alpha:C,\beta:D \vdash R \isaSet}
    {\Gamma\pipe \alpha:C, x:R, \beta: D \vdash x : R}

    \inferrule{\forall x:A \in \Gamma. \Delta \vdash \gamma(x) : A}{\gamma : \Delta \to \Gamma}\\

    \inferrule*%% [right=NullaryBoundaryMap]
               {\Gamma \isadtctx}{\Gamma \vdash \cdot : \cdot \to \cdot}\and
    \inferrule*{\Gamma \pipe \alpha':C' \vdash A : C}{\Gamma \vdash A/\alpha : \alpha':C' \to \alpha:C}

    \inferrule*
    {\Gamma \pipe \alpha':C' \vdash A : C\and
      \Gamma \pipe \beta':D' \vdash B : D
    }
    {\Gamma \vdash (A/\alpha;B/\beta) : (\alpha':C';\beta':D') \to (\alpha:C;\beta':D')}\\

    \inferrule*%% [right=NullCtxMap]
               {\Gamma \isadtctx}{\Gamma \vdash \cdot : \cdot \to \cdot}

    \inferrule*{\Gamma \pipe \alpha':C' \vdash A : C}{\Gamma \vdash A/\alpha : \alpha':C' \to \alpha:C}

    \inferrule
    {\Psi = \Psi_1 \jnctx \Psi_M \jnctx \Psi_2\and
      \Gamma \vdash \phi_1 : \Psi_1 \to \Phi_1\and
      \Gamma \pipe \Psi_M \vdash M : R[d^+\phi_1;d^-\phi_2] \and
      \Gamma \vdash \phi_2 : \Psi_2 \to \Phi_2
      }
    {\Gamma \vdash \phi_1,M/x,\phi_2 : \Psi \to \Phi_1,x:R,\Phi_2}
  \end{mathpar}
  \caption{Substitutions in SCT}
  \label{fig:basic-rules}
\end{figure}

\begin{figure}
  \begin{mathpar}
    \inferrule
    {}
    {\cdot : \cdot \to \cdot}

    \inferrule
    {\gamma : \Delta \to \Gamma \and \Delta \vdash M : A[\gamma]}
    {\gamma,M/X : \Delta \to \Gamma,X:A}
    
    \inferrule
    {\Gamma \vdash \Phi \isavectx \and
    \Gamma \vdash \Psi \isavectx \and
    d^+\Phi = d^- \Psi}
    {\Gamma \vdash \Phi \jnctx \Psi \isavectx}

    \inferrule
    {\Gamma \vdash \phi : \Phi' \to \Phi \and
     \Gamma \vdash \psi : \Psi' \to \Psi\and
     d^+\phi = d^- \psi}
    {\Gamma \vdash \phi \jnctx \psi : \Phi' \jnctx \Psi' \to \Phi \jnctx \Psi}

    \inferrule*[right=ObSubst]
    {\alpha : \cat C \vdash B : \cat D\and
    \beta : \cat D \vdash A : \cat E}
    {\alpha : \cat C \vdash A[B/\beta] : \cat E}\\

    \inferrule*[right=Everything Subst]{\gamma : \Delta \to \Gamma \and \Gamma J}{\Delta J}

    \inferrule{\Gamma \vdash \xi : Z \to \Xi\and \Gamma \pipe \Xi \vdash R \isaSet}{\Gamma \pipe Z \vdash R[\xi] \isaSet}

    \inferrule{\Gamma \vdash \phi : \Psi \to \Phi\and \Gamma \pipe \Phi \vdash M : R}{\Gamma \pipe \Psi \vdash M[\phi] : R[\underline \phi]}
  \end{mathpar}
  \begin{align*}
    (\Phi \jnctx \Psi) \jnctx \Sigma &= \Phi \jnctx (\Psi \jnctx \Sigma)\\
    \Phi \jnctx \alpha:C &= \Phi\\
    \alpha:C \jnctx \Psi &= \Psi\\\\
    d^\pm(A/\alpha) &= A/\alpha\\
    d^+(\phi_1,M/x,\phi_2) &= d^+ \phi_2\\
    d^-(\phi_1,M/x,\phi_2) &= d^- \phi_1\\\\
    \underline \cdot &= \cdot\\
    \underline {A/\alpha} &= A/\alpha\\
    \underline {\phi_1,M/x,\phi_2} &= d^-\phi_1;d^+\phi_2
  \end{align*}
  \caption{Admissible Rules}
\end{figure}

\begin{figure}
  \begin{mathpar}
    \begin{array}{rcl}
      x[\gamma] &=& \gamma(x)\\
      J[\gamma][\delta] & = & J[\gamma[\delta]]\\
      \gamma[\delta](x) &=& \gamma(x)[\delta]\\
      
      A[\alpha/\alpha] &=& A\\
      \alpha[A/\alpha] &=& A\\
      A[B/\beta][C/\gamma] & = & A[B[C/\gamma]/\beta]\\
      A[B/\beta][\gamma] &=& A[\gamma][B[\gamma]/\beta]\\\\
      
      R[\cdot] &=& R\\
      R[\alpha/\alpha] &=& R\\
      R[\alpha/\alpha,\beta/\beta] &=& R\\\\
      R[\xi][\xi'] & = & R[\xi[\xi']]\\
      R[\xi][\gamma] &=& R[\gamma][\xi[\gamma]]\\\\

      \cdot[\xi] &=& \cdot\\
      (A/\alpha)[B/\beta] &=& A[B/\beta]/\alpha\\
      (A/\alpha;B/\beta)[\xi] &=& A[d^-\xi]/\alpha,B[d^+\xi]/\beta\\
      \cdot[\gamma] &=& \cdot\\
      (A/\alpha)[\gamma] &=& A[\gamma]/\alpha\\
      (A/\alpha;B/\beta)[\gamma] &=& A[\gamma]/\alpha;B[\gamma]/\beta\\\\

      \cdot[\gamma] &=& \cdot\\
      (A/\alpha)[\gamma] &=& A[\gamma]/\alpha\\
      (\phi_1,M/x,\phi_2)[\gamma] &=& \phi_1[\gamma],M[\gamma]/x,\phi_2[\gamma]

      TODO: cat substitution id/assoc
    \end{array}
  \end{mathpar}  
  \caption{Algebraic Properties}
  \label{fig:equations}
\end{figure}

\section{Additional Rules}

The key property that the hom allows is that given
any variable in context of a term $\Phi,\alpha,x,\beta,\Psi \vdash M :
Q$ we can represent this term as a ``single-variable term'' by moving
$\Phi$ and $\Psi$ to the right of the turnstile:

\[ \alpha, x, \beta \vdash \lambda \Phi. \lambda \Psi M. : \Phi \triangleright Q \triangleleft \Psi \]

This allows us to describe restricted positive eliminations that only
work on single variable contexts. We give an admissible ``syntax
sugar'' in Figure~\ref{fig:multi-hom}.


Any $\Phi_l,\beta:C,\Phi_r \vdash M : R$ can be represented as a term
$\beta:C \vdash M' : \Phi_r \triangleright_{\beta:C} R \triangleleft_{\beta:C} $


Say we have a term in context
\[ \alpha,x:P,\beta,y:Q,\gamma \vdash M : R \]
Using $\triangleright\triangleleft$, we can produce an equivalent term
of a different output type whose context is one of the single
variables $\alpha,\beta$ or $\gamma$ in the context.

We can make it $\alpha$ by using $\triangleleft$ twice:
\[ \alpha \vdash \lambda^\triangleleft (x:P,\beta). \lambda^\triangleleft (y:Q, \gamma). M : \forall \beta. (\forall \gamma. R \triangleleft Q) \triangleleft P \]

Or $\gamma$ by using $\triangleright$ twice:
\[ \gamma \vdash\lambda^\triangleright (\beta, y:Q).  \lambda^\triangleright (\alpha, x:P). M : \forall \gamma. Q \triangleright \forall \alpha. P \triangleright R \]

And we have two different ways to make the context $\beta$, either we use $\triangleright$ and then $\triangleleft$:
\[ \beta \vdash\lambda^\triangleright (\alpha, x:P).  \lambda^\triangleleft (y:Q, \gamma). M : \forall \alpha. P \triangleright (\forall \gamma. R \triangleleft Q) \]
or we do the opposite order:
\[ \beta \vdash \lambda^\triangleleft (y:Q, \gamma). \lambda^\triangleright (\alpha, x:P). M : \forall \gamma. (\forall \alpha. P \triangleright R) \triangleleft Q \]

In general we can iterate this process. That is given any $\Phi,\Psi$
with $d^+\Phi = d^-\Psi = \beta:D$, we can take a term $\Phi \jnctx
\Psi \vdash M : R$ and produce an equivalent term $\beta:D \vdash M :
\Phi \triangleright R \triangleleft \Psi$. To formalize this, we show
how we can create ``multi-hom'' jumbo connectives $\Phi \triangleright
R$ and $R \triangleleft \Psi$, where the boundary conditions on
$\Phi,\Psi$ are analogous to the ordinary case. These are given in
Figure\ref{multi-hom}.

\begin{figure}
  \begin{align*}
    P \triangleleft \gamma:E &= P\\
    P \triangleleft (\Psi,x:R,\gamma:E) &= (\forall \gamma: E. P \triangleleft R) \triangleleft \Psi\\
    %% \cdot \triangleright_{\alpha : C} P &= P\\
    %% (x:R,\beta:D,\Phi_r)\triangleright_{\alpha:C} P  &= \Phi_R \triangleright_{\beta:D} \forall \beta. R \triangleright P
  \end{align*}
  \begin{mathpar}
    \inferrule
    {\Gamma \pipe d^+\Xi; \gamma: E \vdash P \isaSet\and
     \Gamma \pipe d^-\Xi; \gamma:E \vdash \Psi \isavectx}
    {\Gamma \pipe \Xi \vdash P \triangleleft \Psi}

    \inferrule*[right=Covariant Multi-Hom introduction]
    {\Gamma\pipe\Phi\jnctx \Psi \vdash M : P}
    {\Gamma\pipe\Phi \vdash \lambda^\triangleleft \Psi. M : P \triangleleft \Psi}\and

    \inferrule*[right=Covar Multi-Hom elim]
    {~}
    {\Gamma \pipe f:P\triangleleft \Psi,\Psi \vdash \textrm{app}\triangleleft_\Psi : P}

    %%   (\lambda^\triangleleft \Psi. M)[\phi] = \lambda^\triangleleft \Psi. M[\phi,\Psi]\\
    %%   \lambda^\triangleleft \gamma. M = M\\
    %%   \lambda^\triangleleft (\Psi,x:R,\gamma:\cat E). M = \lambda^\triangleleft \Psi. \lambda^\triangleleft (x:R,\gamma:\cat E). M\\

    %%   (\lambda^\triangleleft \gamma. M)[\phi] = M[\phi] = \lambda^\triangleleft \gamma. M[\phi]\\
    %%   (\lambda^\triangleleft (\Psi,x:r,\gamma:\cat E). M)[\phi] =
    %%   (\lambda^\triangleleft \Psi. \lambda^\triangleleft (x:R,\gamma:\cat E). M)[\phi]
    %%   = \lambda^\triangleleft \Psi. (\lambda^\triangleleft (x:R,\gamma:\cat E). M)[\phi, \Psi]
    %%   = \lambda^\triangleleft \Psi. \lambda^\triangleleft (x:R,\gamma:\cat E). M[\phi,\Psi,x:R,\gamma]      

    %%   \inferrule*[right=Covariant Multi-Hom elimination]
    %%   {\Phi_1 \vdash M : P\triangleleft \Phi_2 \and \phi_2 : \Psi \vdash \Phi_2 \subst(\xi)}
    %%   {\Phi_1,\Psi \vdash M\triangleleft \phi_2 : P[d^+\xi]}\\

    %%   M \triangleleft (A/\gamma) = M[A/\gamma]\\
    %%   M \triangleleft (\phi, N/x,A/\gamma) =  M \triangleleft (N, A) \triangleleft \phi

    %%   (M \triangleleft \phi_2)[\phi_1,\psi] = M[\phi_1] \triangleleft [\phi_2[\psi]]\\
    %%   TODO: prove

    %%   TODO: (\lambda^\triangleleft \Phi. M) \triangleleft \phi = M[\phi]
    %%   TODO: M = \lambda^\triangleleft \Phi. M\triangleleft \Phi

    %%   TODO: contravariant by duality
    %%   %% \inferrule*[right=Contra Multi-Hom formation]
    %%   %% {\Phi \isavectx(\alpha:\cat C; d^-\Xi) \and P \prof(\alpha : \cat C; d^+\Xi)}
    %%   %% {\Phi \triangleright P \prof(\xi)}

    %%   %% \alpha \triangleright P = P\\
    %%   %% \alpha:\cat C, x:R, \Phi \triangleright P = \forall \alpha. R \triangleright \Phi \triangleright P\\
  \end{mathpar}
  \begin{align*}
    \lambda^\triangleleft \gamma:E. M &= M\\
    \lambda^\triangleleft (\Psi,x:R,\gamma:E). M &= \lambda^\triangleleft \Psi. \lambda^\triangleleft (x:R,\gamma:E). M\\\\
    \textrm{app}\triangleleft_{\gamma:E} &= f\\
    \textrm{app}\triangleleft_{\Psi,x:R\gamma:E} &= \textrm{app}\triangleleft_\Psi \triangleleft (x,\gamma) \\
  \end{align*}
  \caption{Multi-Hom}
  \label{multi-hom}
\end{figure}

\section{Unit and Tensor}
we can define the
\emph{unit} , which in set-theoretic settings we
might call the \emph{identity/equality judgment} $A \equiv_{\cat C}
A'$.
%
First the formation rule says given two objects $A,A'$ in the same
category $\cat C$, we can form the hom-set from $A$ to $A'$ $A
\to_{\cat C} A'$.
%
The introduction rule says that given any object $A$, we can define
the identity arrow from $A$ to itself $\id(A) : A \to_{\cat C} A$.
%
Without knowing more about the category in question, these are the
only arrows we know exist.
%

The elimination form is trickiest to define in a natural deduction style.
%
First, let's consider the universal property of the Hom. From
Crutwell-Shulman, we want a 2-cell

\[% https://tikzcd.yichuanshen.de/#N4Igdg9gJgpgziAXAbVABwnAlgFyxMJZARgBoAGAXVJADcBDAGwFcYkQBBEAX1PU1z5CKcqQBM1Ok1bsuvftjwEiY8ZIYs2iTjz4gMioUTLF10rSABOMAGaMekmFADm8IqBuWIAWyRkQOBBIqlKa7AASProeXr6IogFBiP6MWGAWUBDMAEaMbDQAFjD0UOyQ6SA0OPRY9trlbPIgnlHxVUkhqRXamTl5lSBFJWUE+QE1deCjDtxAA
\begin{tikzcd}
                    & A \arrow[ldd, no head, Rightarrow] \arrow[rdd, no head, Rightarrow] &   \\
                    & refl                                                                &   \\
A \arrow[rr, "Hom"] &                                                                     & A
\end{tikzcd}\]

That is universal/opcartesian in that for any 2-cell $t$:

\[ % https://tikzcd.yichuanshen.de/#N4Igdg9gJgpgziAXAbVABwnAlgFyxMJZABgBpiBdUkANwEMAbAVxiRACEQBfU9TXfIRQBGclVqMWbbrxAZseAkQBMY6vWatEIAIIy+CwUQAsaiZrZ6eBgUpQA2Mxqnb2Acn1z+ioclXLxZy0QAGFPeVtfUwD1SWCAEXDvIxQAZlJhQLi2HA8ucRgoAHN4IlAAMwAnCABbJDIQHAgkYWsQKtqW6iakVRAGLDBgqDo4AAtCkGoJuig2SCHPDrrEPp7EdP7B4LgIAbnpmFmkMCYGBmoGOgAjGAYABWS7fphynCnzFxAACU625aQm3Wpi2i20I3Gk0Ox20C1Y-2qKwa6wArAjOogQet7OiVijus1EI5PsEAErcChcIA
\begin{tikzcd}
B \arrow[r] \arrow[rrdd] & {} \arrow[r, no head, dashed] & A \arrow[rr, "Hom"] &    & A \arrow[rr, no head, dashed] &  & B' \arrow[lldd] \\
                         &                               &                     & t' &                               &  &                 \\
                         &                               & C \arrow[rr, "R"]   &    & D                             &  &                
\end{tikzcd}\]

there exists a $t'$ which extends $t$ in that the following is equal
to $t$:
\[% https://tikzcd.yichuanshen.de/#N4Igdg9gJgpgziAXAbVABwnAlgFyxMJZABgBoAmAXVJADcBDAGwFcYkQAhEAX1PU1z5CKAIwVqdJq3Y8+IDNjwEi5cTQYs2iEAEFZ-RUKIAWNZM3s9vA4OUoAbGY3TtHAOT75ApcOSrjEs5aIADCngq2vqYB6lLBACLh3kYoAMykqYFx7Dge1l6GdsjpxFkW2lZyET5EZKWx5SBJhb5i9eYuTfnVKciO7UEy3clF6SJlnQBOMABmjDwSMFAA5vBEoDOTEAC2SGQgOBBIIvmbO8c0h0iqIIxYYMFQ9HAAFksgNG-0UOyQD55nXaIG5XRDpW73YJwCB3H6fGDfJBgZiMRg0Rj0ABGMEYAAURsJbrMcB8OsEABLnU5bIHg0GmCH-bRPV7veGI7R-NjU86IfaggCsPKBDNB9mFSAFlyOiEcZPYACUATSkAAOaXXdGQ9hQCDMTGMNjsn6cghGg70LDzU3-CWIdUHGXgu5MkC6-WG0lfE3gM2knCW62+21yQFIACcGr5dsjjuO+xdUJhWDhIG9SJRaItVt+frtIn5MrEjKTsK9CJNyNRl0DuZDGxViALUYdiZ1zzeqfTNu5ocbDtBImLbeZHbZaYrdd7Dd5Q6jDJHbr1BvN3eD5oDOZ7C24QA
\begin{tikzcd}
{} \arrow[dd] \arrow[r]  & {} \arrow[dd] \arrow[rr, no head, dashed] &                     & A \arrow[ldd, no head, Rightarrow] \arrow[rdd, no head, Rightarrow] \arrow[rrr, no head, dashed] &                               &  & {} \arrow[dd, no head, Rightarrow] \\
                         &                                           &                     & refl                                                                                             &                               &  &                                    \\
B \arrow[r] \arrow[rrdd] & {} \arrow[r, no head, dashed]             & A \arrow[rr, "Hom"] &                                                                                                  & A \arrow[rr, no head, dashed] &  & B' \arrow[lldd]                    \\
                         &                                           &                     & t'                                                                                               &                               &  &                                    \\
                         &                                           & C \arrow[rr, "R"]   &                                                                                                  & D                             &  &                                   
\end{tikzcd}\]

Clearly, our introduction rule models this first 2-cell. Furthermore,
the construction in the second condition is modeled faithfully by the
following sequent rule in the figure. This is a J-like induction rule:
to construct a term using an input of the Hom type, it is sufficient
to consider the case in which the input is the identity morphism.
%
However, building in a substitution the usual way we would (pushing
the substitution into the continuation $N$) doesn't work properly
because we can't necessarily extract a substitution for the
continuation from the substitution for the the whole term. A
substitution for the whole term would be of the form
\[\phi,D_1/\delta_1,M/f,D_2/\delta_2,\psi\]
The the candidate for the continuation's substitution would be
$\phi,?,\psi$ but there is not necessarily something to go in the
middle here since $D_1,D_2$ do not necessarily have the same
\emph{input variable}. However, if we assume the existence already of
the cotensor products the problem can be simplified. Instead of
allowing for a general elimination of a hom term anywhere in the
context, we can instead make a remarkably restricted rule: we can only
eliminate a Hom term when it is the \emph{sole} input in the context:

\[ \inferrule*[right=Hom Elim Very Simple]{\alpha:\cat C \vdash N : Q[\alpha/\alpha_1,\alpha/\alpha_2]}{\alpha_1:\cat C, f : \alpha_1 \to_{\cat C} \alpha_2, \alpha_2:\cat C \vdash J(\alpha_1.\alpha_2. Q, \alpha. N, \alpha_1, f, \alpha_2) : Q}\]

While this rule looks very impoverished at first sight, it has the
virtue of easily supporting a built-in cut:

\[ \inferrule*[right=Hom Elim VSimple Subst]{
  Q \prof(\alpha_1:\cat C; \alpha_2:\cat C)\and
  \Phi \vdash M : A_1 \to_{\cat C} A_2\and
  \alpha:\cat C \vdash N : Q[\alpha/\alpha_1,\alpha/\alpha_2]}{\Phi \vdash J(\alpha_1.\alpha_2. Q, \alpha. N, A_1, M, A_2) : Q[A_1/\alpha_1, A_2/\alpha_2]}\]

However, this rule on its own clearly is not enough to support our
desired universal property shown above diagrammatically. To prove our
completeness theorem, we will need to show that this rule is enough to
give us the desired universal property \emph{assuming the presence of
  left and right cotensors}.

So first, we must show that the ``proper'' sequent rule

\[ \inferrule*[right=Hom-Left (Sequent Style)]
    {\Phi_1[\alpha/\alpha_1], \alpha:\cat C, \Phi_2[\alpha/\alpha_2]\vdash Q[\alpha/\alpha_1,\alpha/\alpha_2]}
    {\Phi_1, \alpha_1 : \cat C, f : \alpha_1 \to_{\cat C} \alpha_2, \alpha_2 : \cat C, \Phi_2\vdash : Q}
\]

is admissible.

Which we can sketch as follows:

\[
\inferrule*
    {\inferrule*{\inferrule*{\Phi_1[\alpha/\alpha_1],\alpha:\cat C, \Phi_2[\alpha/\alpha_1] \vdash Q[\alpha/\alpha_1;\alpha/\alpha_2]}{\alpha:\cat C \vdash \Phi_1[\alpha/\alpha_1] \triangleright Q[\alpha/\alpha_1;\alpha/\alpha_2] \triangleleft \Phi_2[\alpha/\alpha_2]}}{\alpha_1 : \cat C, f : \alpha_1 \to_{\cat C} \alpha_2, \alpha_2 : \cat C\vdash \Phi_1 \triangleright Q \triangleleft \Phi_2}}
    {\Phi_1, \alpha_1 : \cat C, f : \alpha_1 \to_{\cat C} \alpha_2, \alpha_2 : \cat C, \Phi_2\vdash Q}
\]

Which using our syntax sugar is given by the term:
\begin{mathpar}
  \inferrule*
    {\inferrule*
      {\inferrule*
        {\Phi_1[\alpha/\alpha_1],\alpha:\cat C, \Phi_2[\alpha/\alpha_1] \vdash N : Q[\alpha/\alpha_1;\alpha/\alpha_2]}
        {\alpha:\cat C \vdash \lambda^\triangleright \Phi_1[\alpha/\alpha_1]. \lambda^\triangleleft \Phi_2[\alpha/\alpha_2]. N : \Phi_1[\alpha/\alpha_1] \triangleright Q[\alpha/\alpha_1;\alpha/\alpha_2] \triangleleft \Phi_2[\alpha/\alpha_2]}}
      {\alpha_1 : \cat C, f : \alpha_1 \to_{\cat C} \alpha_2, \alpha_2 : \cat C\vdash J(\alpha_1.\alpha_2.\Phi_1 \triangleright Q \triangleleft \Phi_2, \alpha. \lambda^\triangleright \Phi_1[\alpha/\alpha_1]. \lambda^\triangleleft \Phi_2[\alpha/\alpha_2]. N, \alpha_1,f,\alpha_2) : \Phi_1 \triangleright Q \triangleleft \Phi_2}}
    {\Phi_1, \alpha_1 : \cat C, f : \alpha_1 \to_{\cat C} \alpha_2, \alpha_2 : \cat C, \Phi_2\vdash
      \Phi_1 \triangleright (J(\alpha_1.\alpha_2.\Phi_1 \triangleright Q \triangleleft \Phi_2, \alpha. \lambda^\triangleright \Phi_1[\alpha/\alpha_1]. \lambda^\triangleleft \Phi_2[\alpha/\alpha_2]. N, \alpha_1,f,\alpha_2)) \triangleleft \Phi_2 : Q}
\end{mathpar}

We need to show $\beta$ and $\eta$ hold.

First, $\beta$.

\begin{align*}
  &(\Phi_1 \triangleright (J_\to(\alpha_1.\alpha_2.\Phi_1 \triangleright Q \triangleleft \Phi_2, \alpha. \lambda^\triangleright \Phi_1[\alpha/\alpha_1]. \lambda^\triangleleft \Phi_2[\alpha/\alpha_2]. N, \alpha_1,f,\alpha_2)) \triangleleft \Phi_2) [\Phi_1[A/\alpha_1],A/\alpha_1,\id(A),A/\alpha_2,\Phi_2[A/\alpha_2]] \\
  &=
  \Phi_1[A/\alpha_1] \triangleright (J_\to(\alpha_1.\alpha_2.\Phi_1 \triangleright Q \triangleleft \Phi_2, \alpha. \lambda^\triangleright \Phi_1[\alpha/\alpha_1]. \lambda^\triangleleft \Phi_2[\alpha/\alpha_2]. N, A,\id(A), A)) \triangleleft \Phi_2[A/\alpha_2]\\
  &=
  \Phi_1[A/\alpha_1] \triangleright (\lambda^\triangleright \Phi_1[A/\alpha_1]. \lambda^\triangleleft \Phi_2[A/\alpha_2]. N[A/\alpha]) \triangleleft \Phi_2[A/\alpha_2]\\
  &=  (\lambda^\triangleleft \Phi_2[A/\alpha_2]. N[A/\alpha]) \triangleleft \Phi_2[A/\alpha_2]\\
  &=  N[A/\alpha]
\end{align*}

\begin{figure}
  \begin{mathpar}
    \inferrule*[right=Full Unit Elim*]
    {\Phi_1[\alpha/\alpha_1], \alpha:\cat C, \Phi_2[\alpha/\alpha_2]\vdash N : Q[\alpha/\alpha_1,\alpha/\alpha_2]}
    {\Phi_1, \alpha_1 : \cat C, f : \alpha_1 \to_{\cat C} \alpha_2, \alpha_2 : \cat C, \Phi_2\vdash J_\to'(\Phi_1, \alpha_1 : \cat C, f : \alpha_1 \to_{\cat C} \alpha_2, \alpha_2 : \cat C, \Phi_2, Q, \alpha. N) : Q}

    J_\to'(\Phi_1, \alpha_1 : \cat C, f : \alpha_1 \to_{\cat C} \alpha_2, \alpha_2 : \cat C, \Phi_2, Q, \alpha. N) =
    \Phi_1 \triangleright (J(\alpha_1.\alpha_2.\Phi_1 \triangleright Q \triangleleft \Phi_2, \alpha. \lambda^\triangleright \Phi_1[\alpha/\alpha_1]. \lambda^\triangleleft \Phi_2[\alpha/\alpha_2]. N, \alpha_1,f,\alpha_2)) \triangleleft \Phi_2

    \inferrule*[right=Full Unit $\beta$*]
    {}
    {J_\to'(\Phi_1, \alpha_1 : \cat C, f : \alpha_1 \to_{\cat C} \alpha_2, \alpha_2 : \cat C, \Phi_2, Q, \alpha. N)[\Phi_1[\alpha/\alpha_1],\alpha/\alpha_1, \id(\alpha)/f, \alpha/\alpha_2, \Phi_2[\alpha/\alpha_2]]
    = N}

    (\Phi_1 \triangleright J(\alpha_1.\alpha_2.\Phi_1 \triangleright Q \triangleleft \Phi_2, \alpha. \lambda^\triangleright \Phi_1[\alpha/\alpha_1]. \lambda^\triangleleft \Phi_2[\alpha/\alpha_2]. N, \alpha_1,f,\alpha_2) \triangleleft \Phi_2)[\Phi_1[\alpha/\alpha_1],\alpha/\alpha_1, \id(\alpha)/f, \alpha/\alpha_2, \Phi_2[\alpha/\alpha_2]]\\
    = \Phi_1[\alpha/\alpha_1] \triangleright J(\alpha_1.\alpha_2.\Phi_1 \triangleright Q \triangleleft \Phi_2, \alpha. \lambda^\triangleright \Phi_1[\alpha/\alpha_1]. \lambda^\triangleleft \Phi_2[\alpha/\alpha_2]. N, \alpha,\id(\alpha),\alpha) \triangleleft \Phi_2[\alpha/\alpha_2]\\
    = \Phi_1[\alpha/\alpha_1] \triangleright
    (\lambda^\triangleright \Phi_1[\alpha/\alpha_1]. \lambda^\triangleleft \Phi_2[\alpha/\alpha_2]. N)
    \triangleleft \Phi_2[\alpha/\alpha_2]\\
    = N

    \inferrule*[right=Unit$\eta$]
    {\Phi_1, \alpha_1 : \cat C, f : \alpha_1 \to_{\cat C} \alpha_2, \alpha_2 : \cat C, \Phi_2 \vdash M : Q}
    {M =
      J_\to'(\Phi_1, \alpha_1 : \cat C, f : \alpha_1 \to_{\cat C} \alpha_2, \alpha_2 : \cat C, \Phi_2, Q, \alpha. M[\Phi_1[\alpha/\alpha_1],\alpha/\alpha_1,\id(\alpha), \alpha/\alpha_2,\Phi_2[\alpha/\alpha_2]])}

    M
    =(\triangleleft\beta\triangleright\beta)
    \Phi_1 \triangleright
    (\lambda^\triangleright \Phi_1. \lambda^\triangleleft \Phi_2. M)
    \triangleleft \Phi_2
    = \Phi_1 \triangleright
    J_\to(\alpha_1.\alpha_2.Q, \alpha. \lambda^\triangleright \Phi_1[\alpha/\alpha_1]. \lambda^\triangleleft \Phi_2[\alpha/\alpha_2]. M[\alpha/\alpha_1, \id(\alpha)/f,\alpha/\alpha_2], \alpha_1, f, \alpha_2)
    \triangleleft \Phi_2
    =
    J_\to'(\Phi_1, \alpha_1 : \cat C, f : \alpha_1 \to_{\cat C} \alpha_2, \alpha_2 : \cat C, \Phi_2, Q, \alpha. N)
  \end{mathpar}  
  \caption{Admissible Unit Universal Property Rules}
\end{figure}


First, the syntax for tensor is based on the coend formula that gives
the definition in $Cat$, but using logical syntax $\exists$ rather
than integrals because I don't like integral syntax and the syntax
matches existential quantification in type theory.
%
Note that $\exists \beta. ~Q \odot R$ is the application of a single
connective to $Q$ and $R$, and not separable into a quantifier
$\exists$ connective and a product connective $\odot$.
%
Introduction, elimination and $\beta\eta$ are similar to a combination
of existential quantification and the tensor product in linear logic.
%
The introduction rule is justified by the fact that semantically what
we have is a transformation
\[ \textrm{pair} : (\alpha : \cat C,Q,\beta:\cat D,R,\gamma:\cat E) \Rightarrow \exists \beta. R \odot Q \]
And the introduction rule is just building in a substitution to this
transformation.
%

For the elim rule we can pull a similar trick as with the Hom
type. First, the rule where there's only one variable.
\[ \inferrule*[right=Tensor elim Simple]
  {\alpha : \cat C, q : Q, \beta : \cat D, r: R, \gamma : \cat E \vdash N : S}
  {\alpha : \cat C, x : \exists \beta. Q \odot R, \gamma : \cat E \vdash J_\odot(x, q.\beta.r. N) : S}
\]

\begin{figure}
\begin{mathpar}
    \inferrule*[right=Full Tensor Elim*]
    {\Phi_1, \alpha:\cat C, q:Q, \beta:\cat D, r:R, \gamma:\cat E \Phi_2\vdash N : S}
    {\Phi_1, \alpha : \cat C, x : \exists \beta:\cat D. Q \odot R, \gamma : \cat E, \Phi_2
      \vdash J_\odot'(\Phi_1, \alpha : \cat C, x : \exists \beta:\cat D. Q \odot R, \gamma : \cat E, \Phi_2; S;q.\beta.r.N) : S}

    J_\odot'(\Phi_1, \alpha : \cat C, x : \exists \beta:\cat D. Q \odot R, \gamma : \cat E, \Phi_2;S; q.\beta.r.N)
    = \Phi_1 \triangleright J_\odot(\alpha.\gamma.\Phi_1 \triangleright S \triangleleft \Phi_2, \alpha.q.\beta.r.\gamma. \lambda^\triangleleft \Phi_1. \lambda^\triangleright\Phi_2. N, \alpha, x, \gamma) \triangleleft \Phi_2

    J_\odot'(\Phi_1, \alpha : \cat C, x : \exists \beta:\cat D. Q \odot R, \gamma : \cat E, \Phi_2; q.\beta.r.N)[\Phi_1,\alpha/\alpha,(q,\beta,r)/x,\gamma/\gamma,\Phi_2] = N
    

  \inferrule*[right=Full Tensor elim*]{ S \prof(\alpha:\cat C; \gamma : \cat E)\and
    \Phi \vdash M : \exists \beta. Q[A/\alpha] \odot R[G/\gamma]\and
    \alpha : \cat C, q : Q, \beta : \cat D, r: R, \gamma : \cat E \vdash N : S}
  {\Phi \vdash J_\odot(\alpha. \gamma. S, \alpha.q.\beta.r.\gamma.N, A, M, G) : S[A/\alpha, G/\gamma]}

  J_\odot(\alpha. \gamma. S, \alpha.q.\beta.r.\gamma.N, A, M, G)[\phi] =
  J_\odot(\alpha. \gamma. S, \alpha.q.\beta.r.\gamma.N, A[d^-\phi], M[\phi], G[d^+\phi])

  %% \inferrule*[right=Tensor elim Sequent]
  %% {\Phi,\alpha : \cat C, q : Q, \beta : \cat D, r: R, \gamma : \cat E,\Psi \vdash N : S}
  %% {\Phi, \alpha : \cat C, x : \exists \beta. Q \odot R, \gamma : \cat E,\Psi \vdash J_\odot(x;q.\beta.r. N) : S}

  %% \inferrule*[right=Tensor elim Subst]
  %% {\Phi,\alpha : \cat C, q : Q, \beta : \cat D, r: R, \gamma : \cat E,\Phi' \vdash N : S
  %% \and
  %% \phi : \Psi \to \Phi,\alpha : \cat C, q : Q, \beta : \cat D, r: R, \gamma : \cat E,\Phi'}
  %% {\Psi  \vdash J_\odot(x;q.\beta.r. N)[\phi] : S}

  \inferrule*[right=Tensor $\beta$]
  {}{J_\odot(\alpha.\gamma.S, \alpha.q.\beta.r.\gamma.N,A,(M,B,N),G) = N[A/\alpha,M/q,B/\beta,N/r,G/\gamma]}

  \inferrule*[right=Tensor$\eta$]
  {\alpha : \cat C, x : \exists \beta. Q \odot R, \gamma : \cat E \vdash M : S}
  {\alpha : \cat C, x : \exists \beta. Q \odot R, \gamma : \cat E \vdash M =
    J(\alpha.\gamma.S, \alpha.q.\beta.r.\gamma. M[\alpha/\alpha,(q,\beta,r)/x,\gamma/\gamma])
    : S}
\end{mathpar}
\caption{Tensor Admissible Universal Property Rules}
\label{fig:tensor}
\end{figure}



\end{document}

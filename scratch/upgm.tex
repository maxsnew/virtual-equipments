\documentclass{article}

\usepackage[margin=1in]{geometry}
\usepackage{tikz-cd}
\usepackage{amssymb}
\usepackage{amsmath}

\begin{document}

\newtheorem{theorem}{Theorem}
\newtheorem{definition}{Definition}
\newcommand{\pto}{\nrightarrow}
\newcommand{\pfrom}{\nleftarrow}
\newcommand{\vcat}{\mathcal}
\newcommand{\cat}{\mathbb}
\newcommand{\vtkmnd}{\mathbb{K}\text{Mod}(\vcat{V},T)}
\newcommand{\rmod}{\text{RMod}}
\newcommand{\lmod}{\text{LMod}}

\title{Universal Properties in Generalized Multicategories}
\author{Max S. New}

\maketitle

In this note, we seek to develop some theory of right and left
universal properties at the abstract level of generalized
multicategories.
%
Our goal is to ease the development of new type theories with sound
and complete categorical semantics based on Martin-Lof's notion of
judgmental structure.
%
In effect we are abstracting the Curry-Howard-Lambek correspondence as
follows:

\begin{tabular}{|c|c|}
  \hline
  Syntax/Type Theory & Semantics/Category Theory\\ \hline \hline
  Judgmental Structure & Generalized Multicategory\\ \hline
  Negative Type & Representable Generic Right Universal Property\\ \hline
  Positive Type & Representable Generic Left Universal Property\\ \hline  
\end{tabular}

In the process we will provide definitions and theorems that should
free the working semanticist from proving some tedious theorems and
provide guidance for the development of new type theories.
%
In particular, our notions of generic right and left-universal
property lead to a transparent definition of the category of models in
which the type theory presents the initial model.
% TODO: actually see if we *can* prove the following LOL
Further, we provide a quite abstract ``presentation parsimony''
theorem that shows that type theories presenting functorial
constructions can \emph{derive} the functoriality from the
introduction and elimination rules.

\section{Horizontal Profunctors and Representability}

In ordinary category theory there is no need for a separate notion of
right- and left-universal property since categories satisfy perfect
duality: the opposite of a category is a category and so a
left-universal property is defined as a right-universal property in
the opposite category.
%
However multicategories are assymetric by default, the source is given
by a context and the target by a single object, so the opposite of a
multicategory is not a multicategory but a poycategory.
%
It is conceivable that the more general framework of polycategories
will restore this duality, but for now we will work with the
asymmetric notion, which will help to develop intuition in the worst
case.

Despite the lack of perfect duality, we can abstract over some of the
commonalities between right and left universal properties.
%
Right universal properties are right modules and left universal
properties are left modules, but rather than being modules between
categories or between multicategories, the modules will have one side
be a category and the other a multicategory.
%
This structure is generally modelled by a horizontal profunctor
between the virtual equipments of categories and multicategories.

\begin{definition}[Horizontal Profunctor]
  Given virtual double categories $\vcat{V},\vcat{U}$, a
  \emph{horizontal profunctor} $H : \vcat V \pfrom \vcat U$ is for
  each $v \in \vcat V, u \in \vcat U$ a set $H(v,u)$ of ``horizontal
  heteromorphisms'' from $u$ to $v$, so written $v \pfrom u$.
  Additionally a notion of ``hetero-2-cell'' with shapes:

  \begin{tikzcd}
v_n \arrow[dd, "f"] & \cdots \arrow[l, "R_n"] & v_0 \arrow[l, "R_1"] & u_0 \arrow[l, "h"] & \cdots \arrow[l, "Q_1"] & u_m \arrow[l, "Q_m"] \arrow[dd, "g"] \\
 &  &  &  &  &  \\
v' &  &  &  &  & u' \arrow[lllll, "h'"]
  \end{tikzcd}

  that can be composed with other hetero-2-cells and 2-cells in $\vcat
  V, \vcat U$ in the evident way, satisfying the evident unitality and
  associativity laws.
\end{definition}

We draw diagrams for horizontal profunctors the same way we draw them
for virtual double categories, this is justified by the following
definition, in which they are just diagrams in a virtual double category:

\begin{definition}[Cograph of a Horizontal Profunctor]
  For any horizontal profunctor $H : \vcat V \pfrom \vcat U$ we
  construct the cograph $V +_{H} U$ to have as objects $V_0 + U_0$,
  vertical arrows only within $V,U$ and horizontal morphisms defined
  as:
  \begin{align*}
    (V +_{H} U)(v,v') &= V(v,v')\\
    (V +_{H} U)(u,u') &= U(u,u')\\
    (V +_{H} U)(v,u) &= H(v,u)\\
    (V +_{H} U)(u,v) &= \emptyset
  \end{align*}
  With 2-cells inherited from $V,U,H$.
\end{definition}

I'm not sure if it's possible to define a virtual equipment of virtual
double categories, double functors and \emph{horizontal} profunctors,
but we will need at least the following very simple kind of 2-cell in
this imagined equipment of the following shape:

\begin{tikzcd}
 & W \arrow[ldd, "F_V"] \arrow[rdd, "F_U"] &  \\
 & F &  \\
V &  & U \arrow[ll, "H"]
\end{tikzcd}

\begin{definition}[Parameterized Element of a Profunctor]
  Let $\vcat {V,U}$ be virtual double categories, $\cat W$ be a
  category and let $H \vcat V \pfrom \vcat U$ be a profunctor.

  Then a $\cat W$-parameterized element of $H$ consists of
  \begin{enumerate}
  \item Functors $F_U : \cat W \to (\vcat U)_v$ and $F_{\vcat V} :
    \cat W \to (\vcat V)_v$.
  \item For each $w \in W$, a heteromorphism $F(w) : F_{\vcat V} w
    \pfrom F_{\vcat W} w$
  \item For each $f : w \to w'$, a 2-cell
    \begin{tikzcd}
F_{V} w \arrow[dd, "F_{V} f"] &  & F_{U} w \arrow[ll, "Fw"] \arrow[dd, "F_{U}f"] \\
 & Ff &  \\
F_{V} w' &  & F_{U} w' \arrow[ll, "Fw'"]
    \end{tikzcd}
    such that the assignment is functorial (preserve identity,
    composition).
  \end{enumerate}
\end{definition}

Now we can define a general notion of representability in a
profunctor.
%
Representability is a formulation of a (parameterized) type having a
universal property: the horizontal morphism is the universal property
and the 2-cell means ``maps into/out of this object look like the
universal property''.
%
In a virtual double category representability is a relationship
between horizontal and vertical morphisms with the same boundary, but
our horizontal profunctor doesn't give us a notion of vertical
morphism.
%
It might be possible to develop a more general definition of
profunctor, but instead for now we will limit to the following case:

\begin{definition}{Representable Heteromorphism}
  Let $H : \vcat V \pfrom \vcat U$ be a horizontal profunctor and
  $\bar H : \vcat U \to \vcat V$ be a functor with a $(\vcat U)_v$
  parameterized element of $H$:
  \[
  \begin{tikzcd}
 & U_v \arrow[ldd, "\bar H"'] \arrow[rdd, "id"] &  \\
 & \bar H &  \\
V &  & U \arrow[ll, "H"]
\end{tikzcd}
  \]

  Then a representation of a heteromorphism $h : u \pfrom v$ consists
  of a vertical arrow $\bar h : v \to \bar H u$ and a 2-cell
  \[
  \begin{tikzcd}
u \arrow[dd, "id"] &  & v \arrow[dd, "\bar h"] \arrow[ll, "h"] \\
 & \bar h &  \\
u &  & \bar H u \arrow[ll, "\bar H u"]
  \end{tikzcd}
  \]
  that is cartesian in $\vcat V +_{H} \vcat U$.
\end{definition}

Below, we think of all heteromorphisms as expressing
\emph{specifications} for vertical morphisms in this way, which is why
it is thought of as a ``property'' that an object may or may not have.

\section{Concrete and Generic Right Universal Properties}

A right universal property is a specification for a parameterized
object by defining what maps \emph{into} the object look like.
%
These are simpler to formalize for generalized multicategories because
we have less interaction with the context as given by the monad $T$.
%
If our generalized multicategories are defined as kleisli monoids in
$\vcat V$ with monad $T$, then a right universal property on a
``multicategory'' $\vcat M \in \vtkmnd$ will be parameterized by a
``category'' $\cat C \in \vcat V$ and is defined as an element of the
following profunctor:

\begin{definition}{Profunctor of Right Modules}
  The profunctor of category-parameterized right modules on
  multicategories $\rmod : \vtkmnd \pfrom \vcat V$ has as elements
  $\vcat M \pfrom \cat C$ \emph{right modules} which consist of
  \begin{enumerate}
  \item A profunctor $Q : T \vcat M_0 \pfrom \cat C$ in $\vcat V$.
  \item An action
    \[
\begin{tikzcd}
T^2M_0 \arrow[dd, "\mu"] & T M_0 \arrow[l, "T(M_1)"] & C \arrow[dd, "id"] \arrow[l, "Q"] \\
 & Qcut &  \\
TM_0 &  & C \arrow[ll, "Q"]
\end{tikzcd}
\]
  \item That is associative:
    \[
    \begin{tikzcd}
T^3M_0 \arrow[dd, "\mu"] &  & T^2M_0 \arrow[dd, "\mu"] \arrow[ll, "T^2(M_1)"] & T M_0 \arrow[l, "T(M_1)"] & C \arrow[dd, "id"] \arrow[l, "Q"] &  & T^3M_0 \arrow[dd, "\mu"] & T^2M_0 \arrow[l] & TM_0 \arrow[l] \arrow[dd, "id"] & C \arrow[l, "Q"] \arrow[dd, "id"] \\
 & \mu &  & Qcut &  &  &  & T(M_1) &  &  \\
T^2M_0 \arrow[dd, "\mu"] &  & TM_0 \arrow[ll, "T(M_1)"] &  & C \arrow[ll, "Q"] \arrow[dd, "id"] & = & T^2M_0 \arrow[dd, "\mu"] &  & TM_0 \arrow[ll, "T(M_1)"] & C \arrow[l, "Q"] \arrow[dd, "id"] \\
 &  & Qcut &  &  &  &  &  & Qcut &  \\
TM_0 &  &  &  & C \arrow[llll, "Q"] &  & TM_0 &  &  & C \arrow[lll, "Q"]
\end{tikzcd}
    \]
  \item And unital:
    \[\begin{tikzcd}
 & TM_0 \arrow[ld, "T\eta"] \arrow[d, "id"] &  & C \arrow[ll, "Q"] \arrow[d, "id"] &  &  &  \\
T^2M_0 \arrow[dd, "\mu"] & TM_0 \arrow[l, "M_1"] &  & C \arrow[ll, "Q"] \arrow[dd, "id"] & = & TM_0 \arrow[d, "id"] & C \arrow[d, "id"] \arrow[l, "Q"] \\
 &  & Qcut &  &  & TM_0 & C \arrow[l, "Q"] \\
TM_0 &  &  & C \arrow[lll, "Q"] &  &  & 
    \end{tikzcd}\]
  \end{enumerate}

  And a 2-cell with shape 
  \[\begin{tikzcd}
{\vcat M}_n \arrow[dd, "f"] & \cdots \arrow[l, "R_n"] & {\vcat M}_0 \arrow[l, "R_1"] & {\cat C}_0 \arrow[l, "Q"] & \cdots \arrow[l, "P_1"] &  {\cat C}_m \arrow[l, "P_m"] \arrow[dd, "g"] \\
 &  &  &  &  &  \\
{\vcat M}' &  &  &  &  & {\cat C}' \arrow[lllll, "Q'"]
  \end{tikzcd}\]
  is given by a 2-cell in $\vcat V$ with shape:
  \[\begin{tikzcd}
T^{n+1}\vcat{M}_n \arrow[d, "\mu^n"] & \cdots \arrow[l, "T^nR_n"] & T\vcat{M}_0 \arrow[l, "TR_1"] & \cat C_0 \arrow[l, "Q"] & \cdots \arrow[l, "P_1"] & \cat C_m \arrow[l, "P_m"] \arrow[dd, "g"] \\
T\vcat{M}_n \arrow[d, "Tf"] &  &  &  &  &  \\
T\vcat{M}' &  &  &  &  & \cat C' \arrow[lllll, "Q'"]
\end{tikzcd}
  \]

  with composition defined in the obvious way and unitality and
  associativity properties of 2-cells following from the unitality and
  associativity properties of the actions.
\end{definition}

\subsection{Examples}

The most basic is the identity module which is given by $M_1 : TM_0
\pfrom M_0$ itself. We will use this for representability next.

If $\vcat V$ is cartesian or even virtually cartesian, we can define
the \emph{product} property as $\Delta^{*}(M_1\times M_1) : TM_0
\pfrom M_0\times M_0$ and the \emph{cartesian unit} as $!^*(Id_{1}) :
TM_0 \pfrom 1$.

We will need some technology from the next section to define a general
notion of \emph{closed} multicategory, but we can define the property
of $A \to B$ concretely in many different multicategories.

\subsection{Representability}

As noted above, the hom of a generalized multicategory can be seen as
a module $M_1 : T M_0 \pfrom M_0$ and we can make this into a
parameterized element of $\rmod$ of shape:
  \[
  \begin{tikzcd}
 & \vtkmnd_v \arrow[ldd,"id"] \arrow[rdd, "-_{0}"] &  \\
 & -_1 &  \\
\vtkmnd &  & \vcat V \arrow[ll, "\rmod"]
\end{tikzcd}
  \]
and then we can define a representable negative universal property as
just a representable right module with respect to $-_1$.

In more detail a representation of $Q : \vcat M \pfrom \cat C$ is a
2-cell:
  \[
  \begin{tikzcd}
TM_0 \arrow[dd, "id"] &  & C \arrow[dd, "G_Q"] \arrow[ll, "Q"] \\
 & I_Q &  \\
TM_0 &  & M_1 \arrow[ll, "M_1"]
  \end{tikzcd}
  \]
that is cartesian in the appropriate sense and a module
\emph{homomorphism} in that the introduction rule turns $Q$-cuts into
$M$-cuts.
%
Returning to type-theoretic intuition, this square is the
\emph{introduction rule}: above the line we have the ``formal''
product or functin etc, and below the line we have the
``internalized'' notion.
%
The homomorphism property gives the definition of substitution for the
introductin rule.
%
We can define an ``elimination'' rule using the cartesian property of
$I_Q$, except \emph{we can't we need a notion of vertical morphism from cat to multicat(!)}:

\[\begin{tikzcd}
C \arrow[d, "F"] &  & C \arrow[ll, "\to_C"] \arrow[dd, "id"] &  & C \arrow[dd, "F"] &  & C \arrow[dd, "F"] \arrow[ll, "=_C"] \\
M_0 \arrow[d, "\eta"] & \epsilon &  &  &  & F &  \\
T(M_0) \arrow[dd, "id", two heads, tail] &  & C \arrow[ll, "|-^P"] \arrow[dd, "F"] & = & M_0 \arrow[dd, "\eta"] &  & M_0 \arrow[dd, "id"] \arrow[ll, "=_{M_0}"] \\
 & \iota &  &  &  & Var &  \\
T(M_0) &  & M_0 \arrow[ll, "\vdash"'] &  & T(M_0) &  & M_0 \arrow[ll, "|-"]
\end{tikzcd}
\]


\section{Left Universal Properties}

For left-universal properties, consider sums as an illustrative
example.
%
In a category $\cat C$, a sum/coproduct $A + B$ is defined by the property that
\[ \cat C(A + B, C) \equiv \cat C(A,C) \times \cat C(B,C) \]
%
However, a sum in a multicategory $\vcat C$ should \emph{not} just be
a sum in the underlying category (unlike for right universal
properties).
%
Instead it is what's called a \emph{distributive} coproduct, which can
be defined manually by some laws relating compositions of the
coproduct with the monoidal structure.
%
We want to find an abstract setting that captures these distributivity
properties, which already have a type-theoretic explanation: the
equivalence happens ``in an arbitrary context'':
\[ \vcat C(\Gamma,A + B; C) \equiv \vcat C(\Gamma, A; C) \times C(\Gamma,B;C)\]
the usual approach is to use fibrations, and our approach is similar,
but more general and we hope more conceptually direct.

We define left universal properties as saying that maps out of a
context with the object are given by a certain module. For this to
make sense we need a concept of ``context with an object missing'',
and after McBride and Combinatorialists we will call this the
``derivative'' of the context. Since the shape of contexts is given by
a monad, we would like a concept that applies to an arbitrary monad,
not just polynomial functors. I haven't nailed it down but it looks
like we need the following at least.

The microcoms principle is here inescapable and so we need an extra
condition of $\vcat V$ being a \emph{monoidal} virtual double
category, which we will here weaken to a \emph{virtually monoidal}
double category which might be called a ``virtual multi-double
category'' or something like that.

\begin{definition}[Derivative of a Monad]
  Let $T$ be a monad on a virtually monoidal virtual equipment $\vcat
  V$. A \emph{derivative} of $T$ is an endofunctor $dT : \vcat V \to
  \vcat V$ such that/with
  \begin{enumerate}
  \item $dT$ is a vertical monoid in $\vcat V$, i.e. there are natural
    monoidal product $m_A : dT(A),dT(A) \to dT(A)$ and unit $e_A :
    \cdot \to dT(A)$.
  \item $dT$ is a left-module of $T$, i.e. there is a vertical action
    $d\mu : dT \circ T \Rightarrow dT$ that interacts with $\mu,\eta$
    in the obvious way.
  \item A ``plugging in'' morphism $dT,Id \Rightarrow T$, i.e., $l :
    dT(A),A \to TA$.
  \item $\lambda l : dT \to (Id \to T)$ is a monoid homomorphism
    where $(Id \to T)$ uses kleisli composition
  \item The two ways of plugging $dT(T A), A \to T A$ are equal:
    \[\begin{tikzcd}
 & dT(TA),A \arrow[ld, "{dT(\eta),id}"'] \arrow[rd, "{id,\eta}"] &  \\
dT(A),A \arrow[rdd, "l_A"] &  & dT(TA),TA \arrow[d, "l_{TA}"] \\
 &  & T^2A \arrow[ld, "\mu"] \\
 & TA & 
    \end{tikzcd}\]
  \item Possibly other laws...
  \end{enumerate}
\end{definition}
What other conditions we have no clue unfortunately. For instance all
of the above are satisfied by the trivial functor $dT(A) = 1$

Given a derivative $dT$ of $T$, we define the profunctor of
\emph{left} modules as follows:

\begin{definition}[Profunctor of Left Modules]
  The profunctor of category-parameterized left modules on
  multicategories $\lmod : \vcat V \pfrom \vtkmnd$ has as elements
  $\cat C \pfrom \vcat M$ \emph{left modules} which consist of
  \begin{enumerate}
  \item A profunctor $P : dT(\vcat M_0) \otimes \cat C \pfrom \vcat M_0$ in $\vcat V$
  \item An action \[\begin{tikzcd}
dT(TM_0) \times (dT(M_0) \times C) \arrow[dd, "l"] &  & dT(M_0) \times C \arrow[ll, "dT(M_1) \times P"] &  & M_0 \arrow[dd, "id"] \arrow[ll, "P"] \\
 &  & Pcut &  &  \\
dT(M_0) \times C &  &  &  & M_0 \arrow[llll, "P"]
  \end{tikzcd}\]
  \item That is associative
  \item Unital
  \item 2-cells are defined as...
  \end{enumerate}
\end{definition}

\section{Polynomial Multicategories}

$T$-multicategories for an arbitrary monad on a virtual equipment are
maybe too general. Things get a bit more concrete if we consider the
case where $T$ is a \emph{polynomial monad}, and we can get a toy
theory by considering polynomials in \emph{one} variable, i.e., spans.

In this case, we consider the double category of polynomials in some
rich category $E$, whose 2-cells are given by cartesian
morphisms. Then we can define a \emph{flavor} of multicategory to just
be some (horizontal) monad $T : I \pto I$ in that double category, and
can define a $T$-multicategory to be a monad in the slice over
$T$. This makes our monad $T$ much more concrete: the object $I$ is an
object of \emph{sorts}, each ``shape'' gives us a \emph{judgment} of
our multicategory, and the positions in the shape tell us what the
free variables are. The monad structure gives us identity judgments
and the different cuts available between judgments (i.e., all of
them).



\end{document}

%% Local Variables:
%% compile-command: "pdflatex upgm.tex"
%% End:

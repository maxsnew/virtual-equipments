\documentclass{article}

\usepackage[margin=1in]{geometry}
\usepackage{tikz-cd}
\usepackage{amssymb}
\usepackage{amsmath}
\usepackage{mathpartir}

\begin{document}

\newtheorem{theorem}{Theorem}
\newtheorem{definition}{Definition}
\newcommand{\pto}{\nrightarrow}
\newcommand{\pfrom}{\nleftarrow}
\newcommand{\vcat}{\mathcal}
\newcommand{\cat}{\mathbb}
\newcommand{\vtkmnd}{\mathbb{K}\text{Mod} (\vcat{V},T)}
\newcommand{\rmod}{\text{RMod}}
\newcommand{\lmod}{\text{LMod}}

\newcommand{\id}{\text{id}}
\newcommand{\when}{\text{when}\,}
\newcommand{\sort}{\,\text{sort}}
\newcommand{\ctx}{\,\text{context}}
\newcommand{\pipe}{\,|\,}

\title{Judgment Theory: An Syntax for Virtual Equipments}
\author{Max S. New}

\maketitle

In what follows we present an internal language for virtual equipments
(TODO: cite Crutwell-Shulman, Leinster, Hermida)
%
Despite sounding slightly obscure, virtual equipments are quite useful
as ``a good place to do category theory'', that is a virtual equipment
has enough structure to define internal notions of category,
multicategory and polycategory.
%
From a type-theoretic angle and a sprinkle of Curry-Howard-Lambek
ideology, we can also see a virtual equipment as a ``good place to
study type theories'' and so we call the type theory ``judgment
theory''.

The following figures present the basic formation, primitive and
admissible and equality rules of a type theory we call judgment
theory, an internal language for virtual equipments.
%
Note that contra Crutwell-Shulman, we define a virtual equipment to be
a virtual double category with all restrictions, whereas they
additionally require the category to have all \emph{units}.
%
Judgment theory has 4 central concepts, which we will refer to
sometimes using type theoretic and at other times category-theoretic
terminology.
%
In type theoretic terminology, judgment theory has \emph{sorts} $\cat
C, \cat D, \cat E$, and \emph{types} $A,B,C$ which have a given sort
and are also \emph{parameterized} by an object of some other sort, and
so can also be thought of as \emph{functions} with input and output
sorts.
%
Next there are \emph{judgments} $P,Q,R,S$ which are parameterized by
two variables.
%
In $\alpha^o:\cat C;\beta : \cat D \pipe R$, we say $R$ depends on
$\alpha$ \emph{contravariantly} and in $\beta$ \emph{covariantly}.
%
Finally, we have \emph{terms} $s,t,u$ that are elements of some
judgment $R$ but are also parameterized by elements of some other
judgments which are given by a \emph{context} $\Phi$, which is a
``compatible string'' of judgments where the covariant variable of one
judgment is the same as the contravariant variable of the next.
%
Since these terms are parameterized by other terms, we can see these
as \emph{inference rules} for the judgments.

At other times we will want to think of these using set-theoretic
terminology, in which case we have sets $\cat C,\cat D, \cat E$,
functions/elements $A,B,C$, \emph{spans} $R,\ldots$ and morphisms of
spans $s,t,u$.
%
Finally when we develop multicategory theory we will use
category-theoretic terminology, in which the structure is most
familiar and we have categories, objects/functors,
\emph{profunctors}/bimodules and 2-cells/homomorphisms of bimodules.


\begin{figure}
  \begin{mathpar}
    \cat C \sort\\
    \inferrule*[right=Object/Type-formation]
    {\cat C \sort \and \cat D \sort}
    {\alpha : \cat C \vdash A : \cat D}\\

    \inferrule*[right=Relation/Judgment/Profunctor-formation]
    {\cat C \sort \and \cat D \sort}
    {\alpha^{o}:\cat C; \beta : \cat D \pipe R}

    \inferrule*[right=Context-formation]
    {\cat C \sort \and \cat D \sort}
    {\alpha^{o}:\cat C ; \beta:\cat D\sort \pipe \Phi \ctx}

    \inferrule*[right=Element/Term/Arrow-formation]
    {\alpha^o:\cat C; \beta : \cat D \pipe \Phi\and
      \alpha^o:\cat C; \beta : \cat D \pipe R}
    {\alpha^{o}:\cat C;\beta : \cat D \pipe \Phi \vdash t : R}

    \inferrule*[right=Substitution-formation]
    {\alpha : \cat C \vdash A : \cat C'\\
      \beta : \cat D \vdash B : \cat D'\\
      \alpha^o : \cat C; \beta : \cat D \pipe \Phi\\
      \alpha'^o : \cat C'; \beta' : \cat D' \pipe \Psi}
    {A;B \pipe \Phi\vdash \vec t : \Psi}
  \end{mathpar}
  \caption{Formation Rules of Judgment Theory}
\end{figure}

\begin{figure}
  \begin{mathpar}
    \inferrule*[right=]
    {~}
    {\alpha : \cat C \vdash \alpha : \cat C}

    \inferrule*[right=*]
    {\alpha : \cat C \vdash B : \cat D\and
    \beta : \cat D \vdash A : \cat E}
    {\alpha : \cat C \vdash A[B/\beta] : \cat E}\\

    \inferrule*[right=Restriction*]
    {\alpha^o : \cat C;\beta : \cat D \pipe R\and
      \alpha' : \cat C' \vdash A : \cat C\and
      \beta' : \cat D' \vdash B : \cat D'}
    {\alpha'^o : \cat C';\beta':\cat D' \pipe R[A/\alpha,B/\beta]}\\

    \inferrule
    {~}
    {\alpha^o : \cat C ; \alpha : \cat C \pipe \cdot}

    \inferrule
    {\alpha^o : \cat C; \beta : \cat D \pipe \Phi\and
      \beta^o : \cat D; \gamma : \cat E\pipe R}
    {\alpha^o : \cat C : \gamma : \cat E \pipe \Phi, \phi : R}

    \inferrule*[right=*]
    {\alpha^o : \cat C; \beta : \cat D \pipe \Phi\and
    \beta^o : \cat D; \gamma : \cat E \pipe \Psi}
    {\alpha^o : \cat C; \gamma : \cat E \pipe \Phi, \Psi}\\

    \inferrule*[right=]
    {\Phi \vdash t : R[A/\alpha,B/\beta]}
    {A;B \pipe \Phi \vdash t/\phi : (\phi : R)}

    \inferrule*[right=]
    {A;B \pipe \Psi \vdash \vec t : \Phi \and
      B;C \pipe \Psi' \vdash s : R[B/\beta, C/\gamma]}
    {A;C \pipe \Psi,\Psi' \vdash \vec t, s/\phi : (\Phi, \phi : R)}

    \inferrule*[right=*]
    {A;B \pipe \Psi \vdash \vec t : \Phi \and
    B;C \pipe \Psi' \vdash \vec s : \Phi'}
    {A;C \pipe \Psi,\Psi' \vdash \vec t, \vec s : \Phi,\Phi'}
    \\

    \inferrule*[right=]
    {~}
    {\alpha^o;\beta\pipe \phi : R \vdash \phi : R}
    
    \inferrule*[right=*]
    {A;B \pipe \Psi \vdash \vec s : \Phi\and
      \alpha'^o : \cat C'; \beta' : \cat D' \pipe \Phi \vdash t : R
    }
    {\alpha^o:\cat C; \beta : \cat D \pipe \Psi \vdash t[\vec s] : R[A/\alpha',B/\beta']}
  \end{mathpar}
  \caption{Basic Rules of Judgment Theory, starred rules are
    admissible, non-starred are primitive}
\end{figure}
\begin{figure}
  \begin{mathpar}
    \begin{array}{rcl}
      A[\alpha/\alpha] &=& A\\
      A[B/\beta][C/\gamma] & = & A[B[C/\gamma]/\beta]\\\\
      
      R[\alpha/\alpha,\beta/\beta] &=& R\\
      R[\alpha'.A/\alpha,\beta'.B/\beta][A'/\alpha',B'/\beta'] & = &
      R[A[A'/\alpha']/\alpha,B[B'/\beta']/\beta]\\\\
      \Phi,\cdot &=& \Phi\\
      \cdot,\Phi &=& \Phi\\
      (\Phi,\Psi),\Theta &=& \Phi,(\Psi,\Theta)\\\\
      (\vec t,\vec s),\vec u &=& \vec t, (\vec s, \vec u)\\\\
      t[\phi_1/\phi_1,\ldots\phi_n/\phi_n] &=& t\\
      t[\Psi_1.s_1/\phi_1,\ldots,\Psi_n.s_n/\phi_n][\vec u_1/\Psi_1,\ldots,\vec u_n/\Psi_n] &=& t[s_1[\vec u_1/\Psi_1]/\phi_1,\ldots,s_n[\vec u_n/\Psi_n]/\phi_n]\\
    \end{array}
  \end{mathpar}  
  \caption{Algebraic Properties of Basic Rules}
\end{figure}

\begin{figure}
  \begin{mathpar}
    \inferrule*[right=Hom-Formation]
    {\beta : \cat D \vdash A : \cat C\and
    \gamma : \cat E \vdash A' : \cat C}
    {\beta^o : \cat D;\gamma : \cat E \pipe A \to_{\cat C} A'}

    \inferrule*[right=Hom-Introduction]
    {~}
    {\alpha^o : \cat C, \alpha : \cat C | \cdot \vdash \id_{\alpha} : \alpha \to_{\cat C} \alpha}

    \inferrule*[right=Hom-Left (Sequent Style)]
    {\alpha^o,\beta|\Phi,\phi : R[\delta/\delta_1], \psi : S[\delta/\delta_2],\Psi \vdash Q }
    {\alpha^o,\beta|\Phi,\phi : R, \chi : \delta_1 \to \delta_2, \psi : S,\Psi \vdash Q}
    
    \inferrule*[right=Hom-Elim]
    {\beta_1^o;\beta_2\pipe\Psi \vdash f : \beta_1 \to_{C} \beta_2\and
      \alpha^o;\beta\pipe \Phi\and
      \beta^o;\gamma\pipe \Phi'\and
      \Phi,\Phi' \vdash g : R}
    {\Phi[\cdot;\beta_1/\beta],\Psi,\Phi'[\beta_2/\beta;\cdot]\vdash \when f = \id_{\beta}.~ g[\beta_1/\beta;\beta_2/\beta] : R[\beta_1/\beta;\beta_2/\beta]}
  \end{mathpar}
  \caption{Hom Sets}
\end{figure}



\end{document}

%% Local Variables:
%% compile-command: "pdflatex syntax.tex"
%% End:
